\documentclass[thmsb,11pt]{article}
\usepackage{amsfonts}
\usepackage{appendix}
\usepackage[pagewise,displaymath, mathlines]{lineno}
\usepackage{amssymb}
\usepackage{amsmath}
\usepackage{graphicx}
\usepackage{color}
\usepackage{refcount}
\usepackage{natbib}
\usepackage{bm}
\usepackage{hyperref}
\usepackage{epstopdf}
\setcounter{MaxMatrixCols}{10}
\newtheorem{theorem}{Theorem}
\newtheorem{acknowledgement}[theorem]{Acknowledgement}
\newtheorem{algorithm}[theorem]{Algorithm}
\newtheorem{assumption}{Assumption}
\newtheorem{axiom}{Axiom}
\newtheorem{case}[theorem]{Case}
\newtheorem{claim}[theorem]{Claim}
\newtheorem{conclusion}[theorem]{Conclusion}
\newtheorem{condition}[theorem]{Condition}
\newtheorem{conjecture}{Conjecture}
\newtheorem{corollary}{Corollary}
\newtheorem{criterion}[theorem]{Criterion}
\newtheorem{definition}{Definition}
\newtheorem{lemma}{Lemma}
\newtheorem{problem}[theorem]{Problem}
\newtheorem{proposition}{Proposition}
\newtheorem{solution}[theorem]{Solution}
\newtheorem{summary}[theorem]{Summary}
\newtheorem{example}{Example}
\newtheorem{exercise}{Exercise}
\newtheorem{notation}{Notation}
\newtheorem{remark}{Remark}
\newcommand{\bmat}{\begin{matrix}}
\newcommand{\emat}{\end{matrix}}
\newcommand{\ov}{\overline}
\newcommand{\un}{\underline}
\newcommand{\EE}{\mathbb E}
\newcommand{\var}{\mathrm{var}}
\newcommand{\cov}{\mathrm{cov}}
\newcommand{\corr}{\mathrm{corr}}
\newcommand{\dd}{\displaystyle}
\newcommand{\ZZ}{\mathbb{Z}}
\newcommand{\RR}{\mathbb{R}}
\newcommand{\FF}{\mathbb F}
\newcommand{\LL}{\mathbb L}
\newcommand{\MM}{\mathbb M}
\newcommand{\KK}{\mathbb K}
\newcommand{\HH}{\mathbb H}
\newcommand{\QQ}{\mathbb Q}
\newcommand{\CC}{\mathbb{C}}
\newcommand{\Pin}{P_{\text{in}}}
\newcommand{\bx}{\mathbf{x}}
\newcommand{\bp}{\mathbf{p}}
\newcommand{\by}{\mathbf{y}}
\newcommand{\cP}{{\cal P}}
\newcommand{\bB}{\mathbf B}
\newcommand{\bM}{\mathbf M}
\newcommand{\bS}{\mathbf S}
\newcommand{\phis}{\varphi}
\newcommand{\barphis}{\overline\phis}
\newcommand{\se}{\text{se}}
\newcommand{\daga}{a^\dagger}
\newcommand{\devides}{\bigl |}
\newcommand{\eval}{\biggl |}
\newcommand{\ybar}{\overline y}
\newcommand{\bWhat}{\hat{\mathbf W}}
\newcommand{\bW}{\mathbf W}
\newcommand{\bz}{\mathbf z}
\newcommand{\bs}{\mathbf s}
\newcommand{\rightas}{\stackrel{a.s.}{\rightarrow}}
\newcommand{\rightp}{\stackrel{p}{\rightarrow}}
\newcommand{\rightd}{\stackrel{d}{\rightarrow}}
\newcommand{\bI}{\mathbf I}
\newcommand{\barB}{{\overline B}}
\newcommand{\barC}{{\overline C}}
\newcommand{\pbar}{{\overline p}}
\newcommand{\bbar}{{\overline b}}
\newcommand{\mubar}{{\overline \mu}}

\newenvironment{proof}[1][Proof]{\noindent \textbf{#1.} }{\  \rule{0.5em}{0.5em}}
\topmargin=-1cm
\oddsidemargin=-0cm
\textheight=22.2cm
\textwidth=16cm
\setcounter{secnumdepth}{2}
\pagestyle{plain}
\setcounter{figure}{0}
%\setpagewiselinenumbers
%\linenumbers
% Macros for Scientific Word 3.5 documents saved with the LaTeX filter.
% Copyright (C) 2000 Mackichan Software, Inc.

\typeout{TCILATEX Macros for Scientific Word 3.5 <19 July 2000>.}
\typeout{NOTICE:  This macro file is NOT proprietary and may be 
freely copied and distributed.}
%
\makeatletter

%%%%%%%%%%%%%%%%%%%%%
% FMTeXButton
% This is used for putting TeXButtons in the 
% frontmatter of a document. Add a line like
% \QTagDef{FMTeXButton}{101}{} to the filter 
% section of the cst being used. Also add a
% new section containing:
%     [f_101]
%     ALIAS=FMTexButton
%     TAG_TYPE=FIELD
%     TAG_LEADIN=TeX Button:
%
% It also works to put \defs in the preamble after 
% the \input tcilatex
\def\FMTeXButton#1{#1}
%
%%%%%%%%%%%%%%%%%%%%%%
% macros for time
\newcount\@hour\newcount\@minute\chardef\@x10\chardef\@xv60
\def\tcitime{
\def\@time{%
  \@minute\time\@hour\@minute\divide\@hour\@xv
  \ifnum\@hour<\@x 0\fi\the\@hour:%
  \multiply\@hour\@xv\advance\@minute-\@hour
  \ifnum\@minute<\@x 0\fi\the\@minute
  }}%

%%%%%%%%%%%%%%%%%%%%%%
% macro for hyperref
%%% \@ifundefined{hyperref}{\def\hyperref#1#2#3#4{#2\ref{#4}#3}}{}

\def\x@hyperref#1#2#3{%
   % Trun off various catcodes before reading parameter 4
   \catcode`\~ = 12
   \catcode`\$ = 12
   \catcode`\_ = 12
   \catcode`\# = 12
   \catcode`\& = 12
   \y@hyperref{#1}{#2}{#3}%
}

\def\y@hyperref#1#2#3#4{%
   #2\ref{#4}#3
   \catcode`\~ = 13
   \catcode`\$ = 3
   \catcode`\_ = 8
   \catcode`\# = 6
   \catcode`\& = 4
}

\@ifundefined{hyperref}{\let\hyperref\x@hyperref}{}


% macro for external program call
\@ifundefined{qExtProgCall}{\def\qExtProgCall#1#2#3#4#5#6{\relax}}{}
%%%%%%%%%%%%%%%%%%%%%%
%
% macros for graphics
%
\def\FILENAME#1{#1}%
%
\def\QCTOpt[#1]#2{%
  \def\QCTOptB{#1}
  \def\QCTOptA{#2}
}
\def\QCTNOpt#1{%
  \def\QCTOptA{#1}
  \let\QCTOptB\empty
}
\def\Qct{%
  \@ifnextchar[{%
    \QCTOpt}{\QCTNOpt}
}
\def\QCBOpt[#1]#2{%
  \def\QCBOptB{#1}%
  \def\QCBOptA{#2}%
}
\def\QCBNOpt#1{%
  \def\QCBOptA{#1}%
  \let\QCBOptB\empty
}
\def\Qcb{%
  \@ifnextchar[{%
    \QCBOpt}{\QCBNOpt}%
}
\def\PrepCapArgs{%
  \ifx\QCBOptA\empty
    \ifx\QCTOptA\empty
      {}%
    \else
      \ifx\QCTOptB\empty
        {\QCTOptA}%
      \else
        [\QCTOptB]{\QCTOptA}%
      \fi
    \fi
  \else
    \ifx\QCBOptA\empty
      {}%
    \else
      \ifx\QCBOptB\empty
        {\QCBOptA}%
      \else
        [\QCBOptB]{\QCBOptA}%
      \fi
    \fi
  \fi
}
\newcount\GRAPHICSTYPE
%\GRAPHICSTYPE 0 is for TurboTeX
%\GRAPHICSTYPE 1 is for DVIWindo (PostScript)
%%%(removed)%\GRAPHICSTYPE 2 is for psfig (PostScript)
\GRAPHICSTYPE=\z@
\def\GRAPHICSPS#1{%
 \ifcase\GRAPHICSTYPE%\GRAPHICSTYPE=0
   \special{ps: #1}%
 \or%\GRAPHICSTYPE=1
   \special{language "PS", include "#1"}%
%%%\or%\GRAPHICSTYPE=2
%%%  #1%
 \fi
}%
%
\def\GRAPHICSHP#1{\special{include #1}}%
%
% \graffile{ body }                                  %#1
%          { contentswidth (scalar)  }               %#2
%          { contentsheight (scalar) }               %#3
%          { vertical shift when in-line (scalar) }  %#4

\def\graffile#1#2#3#4{%
%%% \ifnum\GRAPHICSTYPE=\tw@
%%%  %Following if using psfig
%%%  \@ifundefined{psfig}{\input psfig.tex}{}%
%%%  \psfig{file=#1, height=#3, width=#2}%
%%% \else
  %Following for all others
  % JCS - added BOXTHEFRAME, see below
    \bgroup
	   \@inlabelfalse
       \leavevmode
       \@ifundefined{bbl@deactivate}{\def~{\string~}}{\activesoff}%
        \raise -#4 \BOXTHEFRAME{%
           \hbox to #2{\raise #3\hbox to #2{\null #1\hfil}}}%
    \egroup
}%
%
% A box for drafts
\def\draftbox#1#2#3#4{%
 \leavevmode\raise -#4 \hbox{%
  \frame{\rlap{\protect\tiny #1}\hbox to #2%
   {\vrule height#3 width\z@ depth\z@\hfil}%
  }%
 }%
}%
%
\newcount\draft
\draft=\z@
\let\nographics=\draft
\newif\ifwasdraft
\wasdraftfalse

%  \GRAPHIC{ body }                                  %#1
%          { draft name }                            %#2
%          { contentswidth (scalar)  }               %#3
%          { contentsheight (scalar) }               %#4
%          { vertical shift when in-line (scalar) }  %#5
\def\GRAPHIC#1#2#3#4#5{%
   \ifnum\draft=\@ne\draftbox{#2}{#3}{#4}{#5}%
   \else\graffile{#1}{#3}{#4}{#5}%
   \fi
}
%
\def\addtoLaTeXparams#1{%
    \edef\LaTeXparams{\LaTeXparams #1}}%
%
% JCS -  added a switch BoxFrame that can 
% be set by including X in the frame params.
% If set a box is drawn around the frame.

\newif\ifBoxFrame \BoxFramefalse
\newif\ifOverFrame \OverFramefalse
\newif\ifUnderFrame \UnderFramefalse

\def\BOXTHEFRAME#1{%
   \hbox{%
      \ifBoxFrame
         \frame{#1}%
      \else
         {#1}%
      \fi
   }%
}


\def\doFRAMEparams#1{\BoxFramefalse\OverFramefalse\UnderFramefalse\readFRAMEparams#1\end}%
\def\readFRAMEparams#1{%
 \ifx#1\end%
  \let\next=\relax
  \else
  \ifx#1i\dispkind=\z@\fi
  \ifx#1d\dispkind=\@ne\fi
  \ifx#1f\dispkind=\tw@\fi
  \ifx#1t\addtoLaTeXparams{t}\fi
  \ifx#1b\addtoLaTeXparams{b}\fi
  \ifx#1p\addtoLaTeXparams{p}\fi
  \ifx#1h\addtoLaTeXparams{h}\fi
  \ifx#1X\BoxFrametrue\fi
  \ifx#1O\OverFrametrue\fi
  \ifx#1U\UnderFrametrue\fi
  \ifx#1w
    \ifnum\draft=1\wasdrafttrue\else\wasdraftfalse\fi
    \draft=\@ne
  \fi
  \let\next=\readFRAMEparams
  \fi
 \next
 }%
%
%Macro for In-line graphics object
%   \IFRAME{ contentswidth (scalar)  }               %#1
%          { contentsheight (scalar) }               %#2
%          { vertical shift when in-line (scalar) }  %#3
%          { draft name }                            %#4
%          { body }                                  %#5
%          { caption}                                %#6


\def\IFRAME#1#2#3#4#5#6{%
      \bgroup
      \let\QCTOptA\empty
      \let\QCTOptB\empty
      \let\QCBOptA\empty
      \let\QCBOptB\empty
      #6%
      \parindent=0pt
      \leftskip=0pt
      \rightskip=0pt
      \setbox0=\hbox{\QCBOptA}%
      \@tempdima=#1\relax
      \ifOverFrame
          % Do this later
          \typeout{This is not implemented yet}%
          \show\HELP
      \else
         \ifdim\wd0>\@tempdima
            \advance\@tempdima by \@tempdima
            \ifdim\wd0 >\@tempdima
               \setbox1 =\vbox{%
                  \unskip\hbox to \@tempdima{\hfill\GRAPHIC{#5}{#4}{#1}{#2}{#3}\hfill}%
                  \unskip\hbox to \@tempdima{\parbox[b]{\@tempdima}{\QCBOptA}}%
               }%
               \wd1=\@tempdima
            \else
               \textwidth=\wd0
               \setbox1 =\vbox{%
                 \noindent\hbox to \wd0{\hfill\GRAPHIC{#5}{#4}{#1}{#2}{#3}\hfill}\\%
                 \noindent\hbox{\QCBOptA}%
               }%
               \wd1=\wd0
            \fi
         \else
            \ifdim\wd0>0pt
              \hsize=\@tempdima
              \setbox1=\vbox{%
                \unskip\GRAPHIC{#5}{#4}{#1}{#2}{0pt}%
                \break
                \unskip\hbox to \@tempdima{\hfill \QCBOptA\hfill}%
              }%
              \wd1=\@tempdima
           \else
              \hsize=\@tempdima
              \setbox1=\vbox{%
                \unskip\GRAPHIC{#5}{#4}{#1}{#2}{0pt}%
              }%
              \wd1=\@tempdima
           \fi
         \fi
         \@tempdimb=\ht1
         %\advance\@tempdimb by \dp1
         \advance\@tempdimb by -#2
         \advance\@tempdimb by #3
         \leavevmode
         \raise -\@tempdimb \hbox{\box1}%
      \fi
      \egroup%
}%
%
%Macro for Display graphics object
%   \DFRAME{ contentswidth (scalar)  }               %#1
%          { contentsheight (scalar) }               %#2
%          { draft label }                           %#3
%          { name }                                  %#4
%          { caption}                                %#5
\def\DFRAME#1#2#3#4#5{%
 \begin{center}
     \let\QCTOptA\empty
     \let\QCTOptB\empty
     \let\QCBOptA\empty
     \let\QCBOptB\empty
	 \vbox\bgroup
        \ifOverFrame 
           #5\QCTOptA\par
        \fi
        \GRAPHIC{#4}{#3}{#1}{#2}{\z@}
        \ifUnderFrame 
           \par#5\QCBOptA
        \fi
	 \egroup
 \end{center}%
 }%
%
%Macro for Floating graphic object
%   \FFRAME{ framedata f|i tbph x F|T }              %#1
%          { contentswidth (scalar)  }               %#2
%          { contentsheight (scalar) }               %#3
%          { caption }                               %#4
%          { label }                                 %#5
%          { draft name }                            %#6
%          { body }                                  %#7
\def\FFRAME#1#2#3#4#5#6#7{%
 %If float.sty loaded and float option is 'h', change to 'H'  (gp) 1998/09/05
  \@ifundefined{floatstyle}
    {%floatstyle undefined (and float.sty not present), no change
     \begin{figure}[#1]%
    }
    {%floatstyle DEFINED
	 \ifx#1h%Only the h parameter, change to H
      \begin{figure}[H]%
	 \else
      \begin{figure}[#1]%
	 \fi
	}
  \let\QCTOptA\empty
  \let\QCTOptB\empty
  \let\QCBOptA\empty
  \let\QCBOptB\empty
  \ifOverFrame
    #4
    \ifx\QCTOptA\empty
    \else
      \ifx\QCTOptB\empty
        \caption{\QCTOptA}%
      \else
        \caption[\QCTOptB]{\QCTOptA}%
      \fi
    \fi
    \ifUnderFrame\else
      \label{#5}%
    \fi
  \else
    \UnderFrametrue%
  \fi
  \begin{center}\GRAPHIC{#7}{#6}{#2}{#3}{\z@}\end{center}%
  \ifUnderFrame
    #4
    \ifx\QCBOptA\empty
      \caption{}%
    \else
      \ifx\QCBOptB\empty
        \caption{\QCBOptA}%
      \else
        \caption[\QCBOptB]{\QCBOptA}%
      \fi
    \fi
    \label{#5}%
  \fi
  \end{figure}%
 }%
%
%
%    \FRAME{ framedata f|i tbph x F|T }              %#1
%          { contentswidth (scalar)  }               %#2
%          { contentsheight (scalar) }               %#3
%          { vertical shift when in-line (scalar) }  %#4
%          { caption }                               %#5
%          { label }                                 %#6
%          { name }                                  %#7
%          { body }                                  %#8
%
%    framedata is a string which can contain the following
%    characters: idftbphxFT
%    Their meaning is as follows:
%             i, d or f : in-line, display, or floating
%             t,b,p,h   : LaTeX floating placement options
%             x         : fit contents box to contents
%             F or T    : Figure or Table. 
%                         Later this can expand
%                         to a more general float class.
%
%
\newcount\dispkind%

\def\makeactives{
  \catcode`\"=\active
  \catcode`\;=\active
  \catcode`\:=\active
  \catcode`\'=\active
  \catcode`\~=\active
}
\bgroup
   \makeactives
   \gdef\activesoff{%
      \def"{\string"}
      \def;{\string;}
      \def:{\string:}
      \def'{\string'}
      \def~{\string~}
      %\bbl@deactivate{"}%
      %\bbl@deactivate{;}%
      %\bbl@deactivate{:}%
      %\bbl@deactivate{'}%
    }
\egroup

\def\FRAME#1#2#3#4#5#6#7#8{%
 \bgroup
 \ifnum\draft=\@ne
   \wasdrafttrue
 \else
   \wasdraftfalse%
 \fi
 \def\LaTeXparams{}%
 \dispkind=\z@
 \def\LaTeXparams{}%
 \doFRAMEparams{#1}%
 \ifnum\dispkind=\z@\IFRAME{#2}{#3}{#4}{#7}{#8}{#5}\else
  \ifnum\dispkind=\@ne\DFRAME{#2}{#3}{#7}{#8}{#5}\else
   \ifnum\dispkind=\tw@
    \edef\@tempa{\noexpand\FFRAME{\LaTeXparams}}%
    \@tempa{#2}{#3}{#5}{#6}{#7}{#8}%
    \fi
   \fi
  \fi
  \ifwasdraft\draft=1\else\draft=0\fi{}%
  \egroup
 }%
%
% This macro added to let SW gobble a parameter that
% should not be passed on and expanded. 

\def\TEXUX#1{"texux"}

%
% Macros for text attributes:
%
\def\BF#1{{\bf {#1}}}%
\def\NEG#1{\leavevmode\hbox{\rlap{\thinspace/}{$#1$}}}%
%
%%%%%%%%%%%%%%%%%%%%%%%%%%%%%%%%%%%%%%%%%%%%%%%%%%%%%%%%%%%%%%%%%%%%%%%%
%
%
% macros for user - defined functions
\def\limfunc#1{\mathop{\rm #1}}%
\def\func#1{\mathop{\rm #1}\nolimits}%
% macro for unit names
\def\unit#1{\mathop{\rm #1}\nolimits}%

%
% miscellaneous 
\long\def\QQQ#1#2{%
     \long\expandafter\def\csname#1\endcsname{#2}}%
\@ifundefined{QTP}{\def\QTP#1{}}{}
\@ifundefined{QEXCLUDE}{\def\QEXCLUDE#1{}}{}
\@ifundefined{Qlb}{\def\Qlb#1{#1}}{}
\@ifundefined{Qlt}{\def\Qlt#1{#1}}{}
\def\QWE{}%
\long\def\QQA#1#2{}%
\def\QTR#1#2{{\csname#1\endcsname #2}}%(gp) Is this the best?
\long\def\TeXButton#1#2{#2}%
\long\def\QSubDoc#1#2{#2}%
\def\EXPAND#1[#2]#3{}%
\def\NOEXPAND#1[#2]#3{}%
\def\PROTECTED{}%
\def\LaTeXparent#1{}%
\def\ChildStyles#1{}%
\def\ChildDefaults#1{}%
\def\QTagDef#1#2#3{}%

% Constructs added with Scientific Notebook
\@ifundefined{correctchoice}{\def\correctchoice{\relax}}{}
\@ifundefined{HTML}{\def\HTML#1{\relax}}{}
\@ifundefined{TCIIcon}{\def\TCIIcon#1#2#3#4{\relax}}{}
\if@compatibility
  \typeout{Not defining UNICODE  U or CustomNote commands for LaTeX 2.09.}
\else
  \providecommand{\UNICODE}[2][]{\protect\rule{.1in}{.1in}}
  \providecommand{\U}[1]{\protect\rule{.1in}{.1in}}
  \providecommand{\CustomNote}[3][]{\marginpar{#3}}
\fi

%
% Macros for style editor docs
\@ifundefined{StyleEditBeginDoc}{\def\StyleEditBeginDoc{\relax}}{}
%
% Macros for footnotes
\def\QQfnmark#1{\footnotemark}
\def\QQfntext#1#2{\addtocounter{footnote}{#1}\footnotetext{#2}}
%
% Macros for indexing.
%
\@ifundefined{TCIMAKEINDEX}{}{\makeindex}%
%
% Attempts to avoid problems with other styles
\@ifundefined{abstract}{%
 \def\abstract{%
  \if@twocolumn
   \section*{Abstract (Not appropriate in this style!)}%
   \else \small 
   \begin{center}{\bf Abstract\vspace{-.5em}\vspace{\z@}}\end{center}%
   \quotation 
   \fi
  }%
 }{%
 }%
\@ifundefined{endabstract}{\def\endabstract
  {\if@twocolumn\else\endquotation\fi}}{}%
\@ifundefined{maketitle}{\def\maketitle#1{}}{}%
\@ifundefined{affiliation}{\def\affiliation#1{}}{}%
\@ifundefined{proof}{\def\proof{\noindent{\bfseries Proof. }}}{}%
\@ifundefined{endproof}{\def\endproof{\mbox{\ \rule{.1in}{.1in}}}}{}%
\@ifundefined{newfield}{\def\newfield#1#2{}}{}%
\@ifundefined{chapter}{\def\chapter#1{\par(Chapter head:)#1\par }%
 \newcount\c@chapter}{}%
\@ifundefined{part}{\def\part#1{\par(Part head:)#1\par }}{}%
\@ifundefined{section}{\def\section#1{\par(Section head:)#1\par }}{}%
\@ifundefined{subsection}{\def\subsection#1%
 {\par(Subsection head:)#1\par }}{}%
\@ifundefined{subsubsection}{\def\subsubsection#1%
 {\par(Subsubsection head:)#1\par }}{}%
\@ifundefined{paragraph}{\def\paragraph#1%
 {\par(Subsubsubsection head:)#1\par }}{}%
\@ifundefined{subparagraph}{\def\subparagraph#1%
 {\par(Subsubsubsubsection head:)#1\par }}{}%
%%%%%%%%%%%%%%%%%%%%%%%%%%%%%%%%%%%%%%%%%%%%%%%%%%%%%%%%%%%%%%%%%%%%%%%%
% These symbols are not recognized by LaTeX
\@ifundefined{therefore}{\def\therefore{}}{}%
\@ifundefined{backepsilon}{\def\backepsilon{}}{}%
\@ifundefined{yen}{\def\yen{\hbox{\rm\rlap=Y}}}{}%
\@ifundefined{registered}{%
   \def\registered{\relax\ifmmode{}\r@gistered
                    \else$\m@th\r@gistered$\fi}%
 \def\r@gistered{^{\ooalign
  {\hfil\raise.07ex\hbox{$\scriptstyle\rm\text{R}$}\hfil\crcr
  \mathhexbox20D}}}}{}%
\@ifundefined{Eth}{\def\Eth{}}{}%
\@ifundefined{eth}{\def\eth{}}{}%
\@ifundefined{Thorn}{\def\Thorn{}}{}%
\@ifundefined{thorn}{\def\thorn{}}{}%
% A macro to allow any symbol that requires math to appear in text
\def\TEXTsymbol#1{\mbox{$#1$}}%
\@ifundefined{degree}{\def\degree{{}^{\circ}}}{}%
%
% macros for T3TeX files
\newdimen\theight
\@ifundefined{Column}{\def\Column{%
 \vadjust{\setbox\z@=\hbox{\scriptsize\quad\quad tcol}%
  \theight=\ht\z@\advance\theight by \dp\z@\advance\theight by \lineskip
  \kern -\theight \vbox to \theight{%
   \rightline{\rlap{\box\z@}}%
   \vss
   }%
  }%
 }}{}%
%
\@ifundefined{qed}{\def\qed{%
 \ifhmode\unskip\nobreak\fi\ifmmode\ifinner\else\hskip5\p@\fi\fi
 \hbox{\hskip5\p@\vrule width4\p@ height6\p@ depth1.5\p@\hskip\p@}%
 }}{}%
%
\@ifundefined{cents}{\def\cents{\hbox{\rm\rlap/c}}}{}%
\@ifundefined{miss}{\def\miss{\hbox{\vrule height2\p@ width 2\p@ depth\z@}}}{}%
%
\@ifundefined{vvert}{\def\vvert{\Vert}}{}%  %always translated to \left| or \right|
%
\@ifundefined{tcol}{\def\tcol#1{{\baselineskip=6\p@ \vcenter{#1}} \Column}}{}%
%
\@ifundefined{dB}{\def\dB{\hbox{{}}}}{}%        %dummy entry in column 
\@ifundefined{mB}{\def\mB#1{\hbox{$#1$}}}{}%   %column entry
\@ifundefined{nB}{\def\nB#1{\hbox{#1}}}{}%     %column entry (not math)
%
\@ifundefined{note}{\def\note{$^{\dag}}}{}%
%
\def\newfmtname{LaTeX2e}
% No longer load latexsym.  This is now handled by SWP, which uses amsfonts if necessary
%
\ifx\fmtname\newfmtname
  \DeclareOldFontCommand{\rm}{\normalfont\rmfamily}{\mathrm}
  \DeclareOldFontCommand{\sf}{\normalfont\sffamily}{\mathsf}
  \DeclareOldFontCommand{\tt}{\normalfont\ttfamily}{\mathtt}
  \DeclareOldFontCommand{\bf}{\normalfont\bfseries}{\mathbf}
  \DeclareOldFontCommand{\it}{\normalfont\itshape}{\mathit}
  \DeclareOldFontCommand{\sl}{\normalfont\slshape}{\@nomath\sl}
  \DeclareOldFontCommand{\sc}{\normalfont\scshape}{\@nomath\sc}
\fi

%
% Greek bold macros
% Redefine all of the math symbols 
% which might be bolded	 - there are 
% probably others to add to this list

\def\alpha{{\Greekmath 010B}}%
\def\beta{{\Greekmath 010C}}%
\def\gamma{{\Greekmath 010D}}%
\def\delta{{\Greekmath 010E}}%
\def\epsilon{{\Greekmath 010F}}%
\def\zeta{{\Greekmath 0110}}%
\def\eta{{\Greekmath 0111}}%
\def\theta{{\Greekmath 0112}}%
\def\iota{{\Greekmath 0113}}%
\def\kappa{{\Greekmath 0114}}%
\def\lambda{{\Greekmath 0115}}%
\def\mu{{\Greekmath 0116}}%
\def\nu{{\Greekmath 0117}}%
\def\xi{{\Greekmath 0118}}%
\def\pi{{\Greekmath 0119}}%
\def\rho{{\Greekmath 011A}}%
\def\sigma{{\Greekmath 011B}}%
\def\tau{{\Greekmath 011C}}%
\def\upsilon{{\Greekmath 011D}}%
\def\phi{{\Greekmath 011E}}%
\def\chi{{\Greekmath 011F}}%
\def\psi{{\Greekmath 0120}}%
\def\omega{{\Greekmath 0121}}%
\def\varepsilon{{\Greekmath 0122}}%
\def\vartheta{{\Greekmath 0123}}%
\def\varpi{{\Greekmath 0124}}%
\def\varrho{{\Greekmath 0125}}%
\def\varsigma{{\Greekmath 0126}}%
\def\varphi{{\Greekmath 0127}}%

\def\nabla{{\Greekmath 0272}}
\def\FindBoldGroup{%
   {\setbox0=\hbox{$\mathbf{x\global\edef\theboldgroup{\the\mathgroup}}$}}%
}

\def\Greekmath#1#2#3#4{%
    \if@compatibility
        \ifnum\mathgroup=\symbold
           \mathchoice{\mbox{\boldmath$\displaystyle\mathchar"#1#2#3#4$}}%
                      {\mbox{\boldmath$\textstyle\mathchar"#1#2#3#4$}}%
                      {\mbox{\boldmath$\scriptstyle\mathchar"#1#2#3#4$}}%
                      {\mbox{\boldmath$\scriptscriptstyle\mathchar"#1#2#3#4$}}%
        \else
           \mathchar"#1#2#3#4% 
        \fi 
    \else 
        \FindBoldGroup
        \ifnum\mathgroup=\theboldgroup % For 2e
           \mathchoice{\mbox{\boldmath$\displaystyle\mathchar"#1#2#3#4$}}%
                      {\mbox{\boldmath$\textstyle\mathchar"#1#2#3#4$}}%
                      {\mbox{\boldmath$\scriptstyle\mathchar"#1#2#3#4$}}%
                      {\mbox{\boldmath$\scriptscriptstyle\mathchar"#1#2#3#4$}}%
        \else
           \mathchar"#1#2#3#4% 
        \fi     	    
	  \fi}

\newif\ifGreekBold  \GreekBoldfalse
\let\SAVEPBF=\pbf
\def\pbf{\GreekBoldtrue\SAVEPBF}%
%

\@ifundefined{theorem}{\newtheorem{theorem}{Theorem}}{}
\@ifundefined{lemma}{\newtheorem{lemma}[theorem]{Lemma}}{}
\@ifundefined{corollary}{\newtheorem{corollary}[theorem]{Corollary}}{}
\@ifundefined{conjecture}{\newtheorem{conjecture}[theorem]{Conjecture}}{}
\@ifundefined{proposition}{\newtheorem{proposition}[theorem]{Proposition}}{}
\@ifundefined{axiom}{\newtheorem{axiom}{Axiom}}{}
\@ifundefined{remark}{\newtheorem{remark}{Remark}}{}
\@ifundefined{example}{\newtheorem{example}{Example}}{}
\@ifundefined{exercise}{\newtheorem{exercise}{Exercise}}{}
\@ifundefined{definition}{\newtheorem{definition}{Definition}}{}


\@ifundefined{mathletters}{%
  %\def\theequation{\arabic{equation}}
  \newcounter{equationnumber}  
  \def\mathletters{%
     \addtocounter{equation}{1}
     \edef\@currentlabel{\theequation}%
     \setcounter{equationnumber}{\c@equation}
     \setcounter{equation}{0}%
     \edef\theequation{\@currentlabel\noexpand\alph{equation}}%
  }
  \def\endmathletters{%
     \setcounter{equation}{\value{equationnumber}}%
  }
}{}

%Logos
\@ifundefined{BibTeX}{%
    \def\BibTeX{{\rm B\kern-.05em{\sc i\kern-.025em b}\kern-.08em
                 T\kern-.1667em\lower.7ex\hbox{E}\kern-.125emX}}}{}%
\@ifundefined{AmS}%
    {\def\AmS{{\protect\usefont{OMS}{cmsy}{m}{n}%
                A\kern-.1667em\lower.5ex\hbox{M}\kern-.125emS}}}{}%
\@ifundefined{AmSTeX}{\def\AmSTeX{\protect\AmS-\protect\TeX\@}}{}%
%

% This macro is a fix to eqnarray
\def\@@eqncr{\let\@tempa\relax
    \ifcase\@eqcnt \def\@tempa{& & &}\or \def\@tempa{& &}%
      \else \def\@tempa{&}\fi
     \@tempa
     \if@eqnsw
        \iftag@
           \@taggnum
        \else
           \@eqnnum\stepcounter{equation}%
        \fi
     \fi
     \global\tag@false
     \global\@eqnswtrue
     \global\@eqcnt\z@\cr}


\def\TCItag{\@ifnextchar*{\@TCItagstar}{\@TCItag}}
\def\@TCItag#1{%
    \global\tag@true
    \global\def\@taggnum{(#1)}}
\def\@TCItagstar*#1{%
    \global\tag@true
    \global\def\@taggnum{#1}}
%
%%%%%%%%%%%%%%%%%%%%%%%%%%%%%%%%%%%%%%%%%%%%%%%%%%%%%%%%%%%%%%%%%%%%%
%
\def\tfrac#1#2{{\textstyle {#1 \over #2}}}%
\def\dfrac#1#2{{\displaystyle {#1 \over #2}}}%
\def\binom#1#2{{#1 \choose #2}}%
\def\tbinom#1#2{{\textstyle {#1 \choose #2}}}%
\def\dbinom#1#2{{\displaystyle {#1 \choose #2}}}%
\def\QATOP#1#2{{#1 \atop #2}}%
\def\QTATOP#1#2{{\textstyle {#1 \atop #2}}}%
\def\QDATOP#1#2{{\displaystyle {#1 \atop #2}}}%
\def\QABOVE#1#2#3{{#2 \above#1 #3}}%
\def\QTABOVE#1#2#3{{\textstyle {#2 \above#1 #3}}}%
\def\QDABOVE#1#2#3{{\displaystyle {#2 \above#1 #3}}}%
\def\QOVERD#1#2#3#4{{#3 \overwithdelims#1#2 #4}}%
\def\QTOVERD#1#2#3#4{{\textstyle {#3 \overwithdelims#1#2 #4}}}%
\def\QDOVERD#1#2#3#4{{\displaystyle {#3 \overwithdelims#1#2 #4}}}%
\def\QATOPD#1#2#3#4{{#3 \atopwithdelims#1#2 #4}}%
\def\QTATOPD#1#2#3#4{{\textstyle {#3 \atopwithdelims#1#2 #4}}}%
\def\QDATOPD#1#2#3#4{{\displaystyle {#3 \atopwithdelims#1#2 #4}}}%
\def\QABOVED#1#2#3#4#5{{#4 \abovewithdelims#1#2#3 #5}}%
\def\QTABOVED#1#2#3#4#5{{\textstyle 
   {#4 \abovewithdelims#1#2#3 #5}}}%
\def\QDABOVED#1#2#3#4#5{{\displaystyle 
   {#4 \abovewithdelims#1#2#3 #5}}}%
%
% Macros for text size operators:
%
\def\tint{\mathop{\textstyle \int}}%
\def\tiint{\mathop{\textstyle \iint }}%
\def\tiiint{\mathop{\textstyle \iiint }}%
\def\tiiiint{\mathop{\textstyle \iiiint }}%
\def\tidotsint{\mathop{\textstyle \idotsint }}%
\def\toint{\mathop{\textstyle \oint}}%
\def\tsum{\mathop{\textstyle \sum }}%
\def\tprod{\mathop{\textstyle \prod }}%
\def\tbigcap{\mathop{\textstyle \bigcap }}%
\def\tbigwedge{\mathop{\textstyle \bigwedge }}%
\def\tbigoplus{\mathop{\textstyle \bigoplus }}%
\def\tbigodot{\mathop{\textstyle \bigodot }}%
\def\tbigsqcup{\mathop{\textstyle \bigsqcup }}%
\def\tcoprod{\mathop{\textstyle \coprod }}%
\def\tbigcup{\mathop{\textstyle \bigcup }}%
\def\tbigvee{\mathop{\textstyle \bigvee }}%
\def\tbigotimes{\mathop{\textstyle \bigotimes }}%
\def\tbiguplus{\mathop{\textstyle \biguplus }}%
%
%
%Macros for display size operators:
%
\def\dint{\displaystyle \int}%
\def\diint{\displaystyle \iint}%
\def\diiint{\displaystyle \iiint}%
\def\diiiint{\mathop{\displaystyle \iiiint }}%
\def\didotsint{\mathop{\displaystyle \idotsint }}%
\def\doint{\mathop{\displaystyle \oint}}%
\def\dsum{\mathop{\displaystyle \sum }}%
\def\dprod{\mathop{\displaystyle \prod }}%
\def\dbigcap{\mathop{\displaystyle \bigcap }}%
\def\dbigwedge{\mathop{\displaystyle \bigwedge }}%
\def\dbigoplus{\mathop{\displaystyle \bigoplus }}%
\def\dbigodot{\mathop{\displaystyle \bigodot }}%
\def\dbigsqcup{\mathop{\displaystyle \bigsqcup }}%
\def\dcoprod{\mathop{\displaystyle \coprod }}%
\def\dbigcup{\mathop{\displaystyle \bigcup }}%
\def\dbigvee{\mathop{\displaystyle \bigvee }}%
\def\dbigotimes{\mathop{\displaystyle \bigotimes }}%
\def\dbiguplus{\mathop{\displaystyle \biguplus }}%

%%%%%%%%%%%%%%%%%%%%%%%%%%%%%%%%%%%%%%%%%%%%%%%%%%%%%%%%%%%%%%%%%%%%%%%
% NOTE: The rest of this file is read only if amstex has not been
% loaded.  This section is used to define amstex constructs in the
% event they have not been defined.
%
%


\def\ExitTCILatex{\makeatother\endinput}

\bgroup
\ifx\ds@amstex\relax
   \message{amstex already loaded}\aftergroup\ExitTCILatex
\else
   \@ifpackageloaded{amsmath}%
      {\message{amsmath already loaded}\aftergroup\ExitTCILatex}
      {}
   \@ifpackageloaded{amstex}%
      {\message{amstex already loaded}\aftergroup\ExitTCILatex}
      {}
   \@ifpackageloaded{amsgen}%
      {\message{amsgen already loaded}\aftergroup\ExitTCILatex}
      {}
\fi
\egroup


%%%%%%%%%%%%%%%%%%%%%%%%%%%%%%%%%%%%%%%%%%%%%%%%%%%%%%%%%%%%%%%%%%%%%%%%
%%
%
%
%  Macros to define some AMS LaTeX constructs when 
%  AMS LaTeX has not been loaded
% 
% These macros are copied from the AMS-TeX package for doing
% multiple integrals.
%
\typeout{TCILATEX defining AMS-like constructs}
\let\DOTSI\relax
\def\RIfM@{\relax\ifmmode}%
\def\FN@{\futurelet\next}%
\newcount\intno@
\def\iint{\DOTSI\intno@\tw@\FN@\ints@}%
\def\iiint{\DOTSI\intno@\thr@@\FN@\ints@}%
\def\iiiint{\DOTSI\intno@4 \FN@\ints@}%
\def\idotsint{\DOTSI\intno@\z@\FN@\ints@}%
\def\ints@{\findlimits@\ints@@}%
\newif\iflimtoken@
\newif\iflimits@
\def\findlimits@{\limtoken@true\ifx\next\limits\limits@true
 \else\ifx\next\nolimits\limits@false\else
 \limtoken@false\ifx\ilimits@\nolimits\limits@false\else
 \ifinner\limits@false\else\limits@true\fi\fi\fi\fi}%
\def\multint@{\int\ifnum\intno@=\z@\intdots@                          %1
 \else\intkern@\fi                                                    %2
 \ifnum\intno@>\tw@\int\intkern@\fi                                   %3
 \ifnum\intno@>\thr@@\int\intkern@\fi                                 %4
 \int}%                                                               %5
\def\multintlimits@{\intop\ifnum\intno@=\z@\intdots@\else\intkern@\fi
 \ifnum\intno@>\tw@\intop\intkern@\fi
 \ifnum\intno@>\thr@@\intop\intkern@\fi\intop}%
\def\intic@{%
    \mathchoice{\hskip.5em}{\hskip.4em}{\hskip.4em}{\hskip.4em}}%
\def\negintic@{\mathchoice
 {\hskip-.5em}{\hskip-.4em}{\hskip-.4em}{\hskip-.4em}}%
\def\ints@@{\iflimtoken@                                              %1
 \def\ints@@@{\iflimits@\negintic@
   \mathop{\intic@\multintlimits@}\limits                             %2
  \else\multint@\nolimits\fi                                          %3
  \eat@}%                                                             %4
 \else                                                                %5
 \def\ints@@@{\iflimits@\negintic@
  \mathop{\intic@\multintlimits@}\limits\else
  \multint@\nolimits\fi}\fi\ints@@@}%
\def\intkern@{\mathchoice{\!\!\!}{\!\!}{\!\!}{\!\!}}%
\def\plaincdots@{\mathinner{\cdotp\cdotp\cdotp}}%
\def\intdots@{\mathchoice{\plaincdots@}%
 {{\cdotp}\mkern1.5mu{\cdotp}\mkern1.5mu{\cdotp}}%
 {{\cdotp}\mkern1mu{\cdotp}\mkern1mu{\cdotp}}%
 {{\cdotp}\mkern1mu{\cdotp}\mkern1mu{\cdotp}}}%
%
%
%  These macros are for doing the AMS \text{} construct
%
\def\RIfM@{\relax\protect\ifmmode}
\def\text{\RIfM@\expandafter\text@\else\expandafter\mbox\fi}
\let\nfss@text\text
\def\text@#1{\mathchoice
   {\textdef@\displaystyle\f@size{#1}}%
   {\textdef@\textstyle\tf@size{\firstchoice@false #1}}%
   {\textdef@\textstyle\sf@size{\firstchoice@false #1}}%
   {\textdef@\textstyle \ssf@size{\firstchoice@false #1}}%
   \glb@settings}

\def\textdef@#1#2#3{\hbox{{%
                    \everymath{#1}%
                    \let\f@size#2\selectfont
                    #3}}}
\newif\iffirstchoice@
\firstchoice@true
%
%These are the AMS constructs for multiline limits.
%
\def\Let@{\relax\iffalse{\fi\let\\=\cr\iffalse}\fi}%
\def\vspace@{\def\vspace##1{\crcr\noalign{\vskip##1\relax}}}%
\def\multilimits@{\bgroup\vspace@\Let@
 \baselineskip\fontdimen10 \scriptfont\tw@
 \advance\baselineskip\fontdimen12 \scriptfont\tw@
 \lineskip\thr@@\fontdimen8 \scriptfont\thr@@
 \lineskiplimit\lineskip
 \vbox\bgroup\ialign\bgroup\hfil$\m@th\scriptstyle{##}$\hfil\crcr}%
\def\Sb{_\multilimits@}%
\def\endSb{\crcr\egroup\egroup\egroup}%
\def\Sp{^\multilimits@}%
\let\endSp\endSb
%
%
%These are AMS constructs for horizontal arrows
%
\newdimen\ex@
\ex@.2326ex
\def\rightarrowfill@#1{$#1\m@th\mathord-\mkern-6mu\cleaders
 \hbox{$#1\mkern-2mu\mathord-\mkern-2mu$}\hfill
 \mkern-6mu\mathord\rightarrow$}%
\def\leftarrowfill@#1{$#1\m@th\mathord\leftarrow\mkern-6mu\cleaders
 \hbox{$#1\mkern-2mu\mathord-\mkern-2mu$}\hfill\mkern-6mu\mathord-$}%
\def\leftrightarrowfill@#1{$#1\m@th\mathord\leftarrow
\mkern-6mu\cleaders
 \hbox{$#1\mkern-2mu\mathord-\mkern-2mu$}\hfill
 \mkern-6mu\mathord\rightarrow$}%
\def\overrightarrow{\mathpalette\overrightarrow@}%
\def\overrightarrow@#1#2{\vbox{\ialign{##\crcr\rightarrowfill@#1\crcr
 \noalign{\kern-\ex@\nointerlineskip}$\m@th\hfil#1#2\hfil$\crcr}}}%
\let\overarrow\overrightarrow
\def\overleftarrow{\mathpalette\overleftarrow@}%
\def\overleftarrow@#1#2{\vbox{\ialign{##\crcr\leftarrowfill@#1\crcr
 \noalign{\kern-\ex@\nointerlineskip}$\m@th\hfil#1#2\hfil$\crcr}}}%
\def\overleftrightarrow{\mathpalette\overleftrightarrow@}%
\def\overleftrightarrow@#1#2{\vbox{\ialign{##\crcr
   \leftrightarrowfill@#1\crcr
 \noalign{\kern-\ex@\nointerlineskip}$\m@th\hfil#1#2\hfil$\crcr}}}%
\def\underrightarrow{\mathpalette\underrightarrow@}%
\def\underrightarrow@#1#2{\vtop{\ialign{##\crcr$\m@th\hfil#1#2\hfil
  $\crcr\noalign{\nointerlineskip}\rightarrowfill@#1\crcr}}}%
\let\underarrow\underrightarrow
\def\underleftarrow{\mathpalette\underleftarrow@}%
\def\underleftarrow@#1#2{\vtop{\ialign{##\crcr$\m@th\hfil#1#2\hfil
  $\crcr\noalign{\nointerlineskip}\leftarrowfill@#1\crcr}}}%
\def\underleftrightarrow{\mathpalette\underleftrightarrow@}%
\def\underleftrightarrow@#1#2{\vtop{\ialign{##\crcr$\m@th
  \hfil#1#2\hfil$\crcr
 \noalign{\nointerlineskip}\leftrightarrowfill@#1\crcr}}}%
%%%%%%%%%%%%%%%%%%%%%

\def\qopnamewl@#1{\mathop{\operator@font#1}\nlimits@}
\let\nlimits@\displaylimits
\def\setboxz@h{\setbox\z@\hbox}


\def\varlim@#1#2{\mathop{\vtop{\ialign{##\crcr
 \hfil$#1\m@th\operator@font lim$\hfil\crcr
 \noalign{\nointerlineskip}#2#1\crcr
 \noalign{\nointerlineskip\kern-\ex@}\crcr}}}}

 \def\rightarrowfill@#1{\m@th\setboxz@h{$#1-$}\ht\z@\z@
  $#1\copy\z@\mkern-6mu\cleaders
  \hbox{$#1\mkern-2mu\box\z@\mkern-2mu$}\hfill
  \mkern-6mu\mathord\rightarrow$}
\def\leftarrowfill@#1{\m@th\setboxz@h{$#1-$}\ht\z@\z@
  $#1\mathord\leftarrow\mkern-6mu\cleaders
  \hbox{$#1\mkern-2mu\copy\z@\mkern-2mu$}\hfill
  \mkern-6mu\box\z@$}


\def\projlim{\qopnamewl@{proj\,lim}}
\def\injlim{\qopnamewl@{inj\,lim}}
\def\varinjlim{\mathpalette\varlim@\rightarrowfill@}
\def\varprojlim{\mathpalette\varlim@\leftarrowfill@}
\def\varliminf{\mathpalette\varliminf@{}}
\def\varliminf@#1{\mathop{\underline{\vrule\@depth.2\ex@\@width\z@
   \hbox{$#1\m@th\operator@font lim$}}}}
\def\varlimsup{\mathpalette\varlimsup@{}}
\def\varlimsup@#1{\mathop{\overline
  {\hbox{$#1\m@th\operator@font lim$}}}}

%
%Companion to stackrel
\def\stackunder#1#2{\mathrel{\mathop{#2}\limits_{#1}}}%
%
%
% These are AMS environments that will be defined to
% be verbatims if amstex has not actually been 
% loaded
%
%
\begingroup \catcode `|=0 \catcode `[= 1
\catcode`]=2 \catcode `\{=12 \catcode `\}=12
\catcode`\\=12 
|gdef|@alignverbatim#1\end{align}[#1|end[align]]
|gdef|@salignverbatim#1\end{align*}[#1|end[align*]]

|gdef|@alignatverbatim#1\end{alignat}[#1|end[alignat]]
|gdef|@salignatverbatim#1\end{alignat*}[#1|end[alignat*]]

|gdef|@xalignatverbatim#1\end{xalignat}[#1|end[xalignat]]
|gdef|@sxalignatverbatim#1\end{xalignat*}[#1|end[xalignat*]]

|gdef|@gatherverbatim#1\end{gather}[#1|end[gather]]
|gdef|@sgatherverbatim#1\end{gather*}[#1|end[gather*]]

|gdef|@gatherverbatim#1\end{gather}[#1|end[gather]]
|gdef|@sgatherverbatim#1\end{gather*}[#1|end[gather*]]


|gdef|@multilineverbatim#1\end{multiline}[#1|end[multiline]]
|gdef|@smultilineverbatim#1\end{multiline*}[#1|end[multiline*]]

|gdef|@arraxverbatim#1\end{arrax}[#1|end[arrax]]
|gdef|@sarraxverbatim#1\end{arrax*}[#1|end[arrax*]]

|gdef|@tabulaxverbatim#1\end{tabulax}[#1|end[tabulax]]
|gdef|@stabulaxverbatim#1\end{tabulax*}[#1|end[tabulax*]]


|endgroup
  

  
\def\align{\@verbatim \frenchspacing\@vobeyspaces \@alignverbatim
You are using the "align" environment in a style in which it is not defined.}
\let\endalign=\endtrivlist
 
\@namedef{align*}{\@verbatim\@salignverbatim
You are using the "align*" environment in a style in which it is not defined.}
\expandafter\let\csname endalign*\endcsname =\endtrivlist




\def\alignat{\@verbatim \frenchspacing\@vobeyspaces \@alignatverbatim
You are using the "alignat" environment in a style in which it is not defined.}
\let\endalignat=\endtrivlist
 
\@namedef{alignat*}{\@verbatim\@salignatverbatim
You are using the "alignat*" environment in a style in which it is not defined.}
\expandafter\let\csname endalignat*\endcsname =\endtrivlist




\def\xalignat{\@verbatim \frenchspacing\@vobeyspaces \@xalignatverbatim
You are using the "xalignat" environment in a style in which it is not defined.}
\let\endxalignat=\endtrivlist
 
\@namedef{xalignat*}{\@verbatim\@sxalignatverbatim
You are using the "xalignat*" environment in a style in which it is not defined.}
\expandafter\let\csname endxalignat*\endcsname =\endtrivlist




\def\gather{\@verbatim \frenchspacing\@vobeyspaces \@gatherverbatim
You are using the "gather" environment in a style in which it is not defined.}
\let\endgather=\endtrivlist
 
\@namedef{gather*}{\@verbatim\@sgatherverbatim
You are using the "gather*" environment in a style in which it is not defined.}
\expandafter\let\csname endgather*\endcsname =\endtrivlist


\def\multiline{\@verbatim \frenchspacing\@vobeyspaces \@multilineverbatim
You are using the "multiline" environment in a style in which it is not defined.}
\let\endmultiline=\endtrivlist
 
\@namedef{multiline*}{\@verbatim\@smultilineverbatim
You are using the "multiline*" environment in a style in which it is not defined.}
\expandafter\let\csname endmultiline*\endcsname =\endtrivlist


\def\arrax{\@verbatim \frenchspacing\@vobeyspaces \@arraxverbatim
You are using a type of "array" construct that is only allowed in AmS-LaTeX.}
\let\endarrax=\endtrivlist

\def\tabulax{\@verbatim \frenchspacing\@vobeyspaces \@tabulaxverbatim
You are using a type of "tabular" construct that is only allowed in AmS-LaTeX.}
\let\endtabulax=\endtrivlist

 
\@namedef{arrax*}{\@verbatim\@sarraxverbatim
You are using a type of "array*" construct that is only allowed in AmS-LaTeX.}
\expandafter\let\csname endarrax*\endcsname =\endtrivlist

\@namedef{tabulax*}{\@verbatim\@stabulaxverbatim
You are using a type of "tabular*" construct that is only allowed in AmS-LaTeX.}
\expandafter\let\csname endtabulax*\endcsname =\endtrivlist

% macro to simulate ams tag construct


% This macro is a fix to the equation environment
 \def\endequation{%
     \ifmmode\ifinner % FLEQN hack
      \iftag@
        \addtocounter{equation}{-1} % undo the increment made in the begin part
        $\hfil
           \displaywidth\linewidth\@taggnum\egroup \endtrivlist
        \global\tag@false
        \global\@ignoretrue   
      \else
        $\hfil
           \displaywidth\linewidth\@eqnnum\egroup \endtrivlist
        \global\tag@false
        \global\@ignoretrue 
      \fi
     \else   
      \iftag@
        \addtocounter{equation}{-1} % undo the increment made in the begin part
        \eqno \hbox{\@taggnum}
        \global\tag@false%
        $$\global\@ignoretrue
      \else
        \eqno \hbox{\@eqnnum}% $$ BRACE MATCHING HACK
        $$\global\@ignoretrue
      \fi
     \fi\fi
 } 

 \newif\iftag@ \tag@false
 
 \def\TCItag{\@ifnextchar*{\@TCItagstar}{\@TCItag}}
 \def\@TCItag#1{%
     \global\tag@true
     \global\def\@taggnum{(#1)}}
 \def\@TCItagstar*#1{%
     \global\tag@true
     \global\def\@taggnum{#1}}

  \@ifundefined{tag}{
     \def\tag{\@ifnextchar*{\@tagstar}{\@tag}}
     \def\@tag#1{%
         \global\tag@true
         \global\def\@taggnum{(#1)}}
     \def\@tagstar*#1{%
         \global\tag@true
         \global\def\@taggnum{#1}}
  }{}
% Do not add anything to the end of this file.  
% The last section of the file is loaded only if 
% amstex has not been.



\makeatother
\endinput


\begin{document}

\title{\textbf{Outline: BEGS theory }}
\date{}
\maketitle


\section{Results}

\subsection{Representative agent}
\label{sec rep agent}
This section describes the representative agent environment with risky debt and no transfers. The household values consumption and leisure using a quasi-linear utility function and
solves
\begin{equation}
W_0(b_{-1})\max_{\{c_t,l_t,b_{t}\}_t}\mathbb{E}_0\beta^t\left \{c_t-\frac{l_t^{1+\gamma}}{1+\gamma}\right\}
\end{equation}
subject to
\begin{equation}
c_t+b_{t}=(1-\tau_t)\theta l_t+R_tP_tb_{t-t}
\end{equation}
Using the optimality condition for labor and savings we can summarize the set of implementability constraints for the government as follows
\begin{equation}
\label{rep agent QL seq implementability}
b_{t-1} \frac{P_t}{E_{t-1}P_t}=\mathbb{E}_t\sum_{j}\beta^{t+j}[c_{t+j}-l_{t+j}^{1+\gamma}] \quad \forall t
\end{equation}
We also have the feasibility constraint
\begin{equation}
\label{feasibility}
c_t+g_t\leq\theta l_t,
\end{equation}
and the market clearing for bonds $b_t+B_t=0$.





\noindent The optimal Ramsey allocation solves  $\max_{\{c_t,l_t\}_t}W_0(b_{-1})$ subject to \eqref{rep agent QL seq implementability}, feasibility \eqref{feasibility} and natural debt limits for the government $\underline{B}$\footnote{These will be explicitly derived for the examples we solve in this section.}. For the rest of the note we assume i.i.d exogenous shocks to expenditures denote expenditure and payoffs by $g_t=g(s_t)$and $P_t=P(s_t)$ with the normalization $\mathbb{E}P(s)=1$


\begin{theorem}
\label{rep agent general payoffs}

Suppose $P(s)$ satisfies \[P(s)-P(s')>\beta \frac{g(s)-g(s')}{\underline{B}} \quad \forall \quad s,s, \] there exists an invariant distribution of assets with unbounded support. Further for large enough assets (or debt) there is a drift towards the interior region. 

In particular $V$ is strictly concave and there exists $\underline{B}>B_1>B_2>-\infty$ such that

\[\mathbb{E}V'(B(s))>V'(B\_) \quad B\_>B_2 \]
and
\[\mathbb{E}V'(B(s))<V'(B\_) \quad B\_<B_1 \]

\end{theorem}


\begin{theorem}
\label{rep agent policy}
For the two special payoff structures, the optimal tax and asset policies $\{\tau_t,B_t\}_t$ are characterized as follows,
\begin{enumerate}
\item If $P(s)=1$ 
\[\lim_t \tau_t=-\infty, \quad \lim_t B_t=\infty    \quad a.s\]
\item If $P(s) = 1+ \frac{\beta}{ B^*}(g(s) - \mathbb{E} g)$ \quad for some $B^*\geq\underline{B}$
\[\lim_t \tau_t=\tau^*>\infty, \quad \lim_tB_t=  B^*\quad a.s \quad \forall B_{-1} \]

\end{enumerate}
 
\end{theorem}
%When the risky payoff on the bond $P(s)$ perfectly spans $g(s)$,
\begin{remark}
In the case where payoffs are perfectly correlated with expenditure shocks (case 2), we can express the long run assets \[B^*=\beta \frac{\var(g(s))}{\cov(P(s),g(s))}\]Thus, if the pays off are more in the low expenditure states, the government issues debt in the long run and vice versa. The long run tax rate is inversely related to $B^*$ with the following limits,\[\lim_{B^*\to\underline{B}}\tau^*=\frac{\gamma}{1+\gamma}\quad \lim_{B^*\to\infty}\tau^*=-\infty\] 
 
\end{remark}


\begin{corollary}
Let $\underline{B}>-\infty$ be the natural debt limit for the government. Suppose we impose an upper bound on assets $\overline{B}<\infty$,

\begin{enumerate}
\item If $P(s) = 1+ \frac{\beta}{ B^*}(g(s) - \mathbb{E} g)$ \quad for some $\overline{B}>B^*\geq\underline{B}$
\[\exists \epsilon>0, \quad \Pr\{B_t<\underline{B}+\epsilon \text{ or } B_t>\overline{B}-\epsilon\}=0\]

\item For all other payoffs
\[\forall \epsilon>0, \quad \Pr\{B_t<\underline{B}+\epsilon \text{ and } B_t>\overline{B}-\epsilon \}=1\]
\end{enumerate}

\end{corollary}




  

\begin{theorem}
\label{rep agent ergodic distribution}
Consider a orthogonal decomposition of $P(s)$ as follows
\[P(s)=\hat{P}(s)+P^*(s)\] where
\[
  P^*(s) = 1+\frac{\beta}{B^*}( g - \mathbb{E} g) \text{ for some } B^*\geq \underline{B}
\]  
and $\hat{P}(s)$ is orthogonal to $g(s)$. Expanding the policy rules around the steady state of the $P^*(s)$ economy we have the following characterization,

\begin{itemize}
 \item The ergodic distribution of debt of the policy rules linearized around $(B^*, P^*(s))$ will have mean $B^*$, 
 \item The coefficient of variation is given by
  \[
    \frac{\sigma(B)}{\mathbb E(B)} = \sqrt\frac{\var(P(s)) - |\cov(g(s),P(s))|}{(1+|\cov(g(s),P(s))|)|\cov(g(s),P(s))|}\leq\sqrt\frac{\var(P(s)) - |\cov(g(s),P(s))|}{|\cov(g(s),P(s))|}
  \]
  \item The speed of convergence to the ergodic distribution described by
  \[
    \frac{\EE_{t-1}(B_t-B^*)}{(B_{t-1} - B^*)} = \frac1{1+|\cov(P(s),g(s))|}
  \]

\end{itemize}
  

\end{theorem}





\subsection{Heterogeneous agent:Quasi-linear unproductive agent}
Suppose the unproductive agent has quasi-linear preferences and we additionally impose a non negativity constraint on his consumption.
\begin{theorem}
Let $\omega,n$ be the Pareto weight and mass of the productive agent with $n<\frac{\gamma}{1+\gamma}$. The government trades a risky bond with payoffs $P(s)$ and faces i.i.d expenditure shocks.  The optimal tax, transfer and asset policies $\{\tau_t,T_t,B_t\}$ are characterized as follows,


\begin{enumerate}
 \item For $\omega\geq n \left(\frac{1+\gamma}{\gamma}\right)$ we have  $T_t=0$ and the optimal policy is same as in a representative agent economy studied in Theorems \ref{rep agent general payoffs}, \ref{rep agent policy} and \ref{rep agent ergodic distribution}
 \item For $\omega< n \left(\frac{1+\gamma}{\gamma}\right)$, suppose we further assume that $\min_{s}\{P(s)\}>\beta$. We have two parts:
 
 There exists $\mathcal{B}(\omega)$ and $\tau^*(\omega)$ with $\mathcal{B}'(\omega)>0$ and $\lim_{\omega\to 0}\mathcal{B}(\omega)<0$ such that 
 \begin{enumerate}
  \item $B\_>\mathcal{B(\omega)}$
\[T_t>0, \quad \tau_t=\tau^*(\omega), \textit{ and } B_t=B\_ \quad \forall t\] 
\item $B\_\leq \mathcal{B(\omega)}$, the policies depend on the structure of $P(s)$. 
\begin{enumerate}
 \item For a risky bond with $P(s) = 1+ \frac{\beta}{ B^*}(g(s) - \mathbb{E} g)$ \quad for some $B^*<\mathcal{B}(\omega)$ we have two cases depending on $B\_$
\begin{enumerate}
 \item For $B\_\leq B^*$ 
 \[T_t = 0, \quad \lim_t \tau_t=\tau^{**} (\omega), \textit{ and } \lim_tB_t=  B^*\quad a.s \]
\item For $\mathcal{B}(\omega)>B\_>B^*$
\small
\[\Pr\{\lim_t T_t = 0, \lim_t \tau_t=\tau^{**}(\omega),\lim_tB_t=  B^* \textit{ or } \lim_t T_t>0 \text{ i.o.},\lim_t\tau_t=\tau^*(\omega), \lim_tB_t=\mathcal{B}(\omega) \}>0 \]
 \normalsize 
\end{enumerate}

\item For all other payoffs 
   \[\lim_t T_t=0 \text{ i.o.},\quad \lim_t\tau_t=\tau^*(\omega) \text{ and } \lim_tB_t=\mathcal{B}(\omega)\quad \textit{a.s}\]

 \end{enumerate}

 \end{enumerate}

 \end{enumerate}


\end{theorem}




% \begin{theorem}
% \label{main theorem}
% Suppose the Pareto weight of the agents are interior and we normalize the assets of the unproductive agent to zero. 
% Let $n$ and $\omega$ denote the mass and the Pareto weight of the productive agents. The optimal tax, transfer and debt policy $\{\tau_t,T_t,B_t\}_t$ is characterized as follows,
% \begin{enumerate}
% \item As $n\to 0$,
% \[\lim_t T_t=0,\quad \lim_t\tau_t=-\infty \text{ and } \lim_t B_t=\infty\]
% \item As $n\to 1$,
% \[\lim_t T_t=0 \text{ i.o.},\quad \lim_t\tau_t=\frac{-\gamma(\omega-n)}{\omega-(\omega-n)(1+\gamma)} \text{ and } \lim_tB_t=\mathcal{B}(\omega)<\infty\]
% \end{enumerate}
% 
% Furthermore $\mathcal{B}'(\omega)>0$ and $\lim_{\omega\to 0}\mathcal{B}(\omega)<0$
% \end{theorem}
% 
% \begin{corollary}
% Let $\underline{B}>-\infty$ be the natural debt limit for the government. Suppose we impose an upper bound on assets such that, $\mathcal{B}(\omega)<\overline{B}<\infty$. 
% \begin{enumerate}
% \item As $n\to 0$,
% \[\forall \epsilon>0, \quad \Pr\{B_t<\underline{B}+\epsilon \text{ and } B_t>\overline{B}-\epsilon \}=1\]
% \item As $n\to 1$,
% \[\exists \epsilon>0, \quad \Pr\{B_t<\underline{B}+\epsilon \text{ or } B_t>\underline{B}-\epsilon\}=0\]
% \end{enumerate}
% 
% 
% \end{corollary}


\section{Old Discussion}


The purpose of this note is to study the roles of transfers and debt as tools to hedge aggregate shocks in presence of  incomplete markets. 

We begin with the representative agent model studied in AMSS and make two changes:  first, the government trades a ``risky'' bond and second, the government  is prohibited 
from using transfers. The ``risky'' bond is a security whose payoffs $P_t(s^t)$ are correlated with the aggregate shocks. This feature allows us to vary the government's ability to span aggregate shocks while keeping markets incomplete. Below we will characterize two polar cases analytically: a) risk-free bond where $P_t(s^t)=1$ and b) perfect spanning where $\corr(P_t,g_t)=\pm 1$. Lastly we develop an approximation result for intermediate cases.

The main finding is that  the  levels and fluctuations in tax rates and government debt are both
driven by the covariance between  payoff on the bonds and the exogenous  needs for revenue  driven by exogenous government purchases.
With perfect spanning, the joint distribution of  long-run debt and taxes is degenerate, taxes and debt both  diverge in an economy with only
a risk-free bond being traded. The approximation result allows us to estimate the rates of convergence and volatility of debt and taxes for intermediate cases.

The restriction on transfers is meant to capture costs of using this instrument to hedge aggregate shocks. So far these costs are {\em ad hoc}
and assumed to be arbitrarily high. In the next section we endogenize these costs by adding concerns for redistribution and  allowing the government to choose transfers optimally. These concerns are be modeled by adding a positive
mass of unproductive but risk averse agents and assigning to  them a non-zero Pareto weight. Under the assumption that the government trades a risk free bond\footnote{The results for arbitrary payoffs extend analogously, i.e with perfect spanning the government eventually smooths fluctuations in both taxes and transfers.}
we construct two limiting cases of a representative agent economy by varying the mass of either type of agent.

The key insight is that the welfare costs of using fluctuating transfers to hedge aggregate shocks comes from 
the presence (and mass) of the unproductive risk averse agents. As their mass vanishes,
the optimal policy smooths labor taxes and all shocks are hedged by fluctuating transfers. 
In this case, asymptotically transfers diverge and tax rates approach zero. But when the mass of the productive agents is low, 
limiting transfers are constant. Eventually as their mass goes to zero, the optimal policy has unbounded labor subsidies and the level of transfers approaches zero.
In both cases the implied government assets (under the normalization that the unproductive agent holds no assets) diverge.

These limits can be contrasted with an alternative economy in which  the unproductive agents have quasi-linear preferences and 
we impose a non-negativity constraint on their consumption. As the mass of these unproductive agents approaches zero, 
limiting taxes are constant (they approach zero if their Pareto weight approaches zero too) but transfers are also zero infinitely often. The government accumulates a  finite level assets (typically positive). Thus the limiting allocation is similar to  that studied in AMSS with a non-negativity constraint on transfers where the government eventually accumulates first best level of assets. However, in our case these levels vary  inversely with the governments' preferences for redistribution and can switch signs for a low enough Pareto weight
on the productive agent. More regressive governments employ
tax subsidies and deficits that are financed by issuing debt. The optimal policy uses transfers as against accumulating assets to hedge aggregate shocks.

\section{Results left on the cutting board}

Besides the results mentioned before we have some things we proved that I could not fit. They mainly pertain to existence of steady state for more general preferences and shocks. I list them in this section


\subsection{Persistence}
\begin{theorem}
Consider a representative agent model studied in section \ref{sec rep agent} but with persistence shocks. For every complete market allocation indexed by $\mu^*$ there exists a payoff vector $P(s)$ such that there exists a  steady state in the risky debt economy that supports the same allocation. For an arbitrary payoff vector $P(s)$ (generically) there exists a steady state only if $\|S\|=2$
\end{theorem}


\subsection{Risk Aversion}
We have some scattered results for risk aversion and existence of steady state.
\begin{theorem}
Assume that $U$ is  separable and iso-elastic,	 $U(c,l) = \frac{c^{1-\sigma}}{1-\sigma} -\frac{ l^{1+\gamma}}{1+\gamma}$.
and shocks are i.i.d with $s_b$  being the ``adverse'' state (either low TFP or high govt. expenditures)
and $s_g$ begin the good state. Let $x_{fb}$ be the discounted present value of marginal utility weighted government surpluses associated with the first best allocation.
%\st{ be a value of  %\st{the state $x$ from which a government can implement first=best with complete markets}
%marginal utility weighted debt associated with the first-best allocation with complete markets.}\tjs{David: the phrase
%``first-best allocation with complete markets'' remains imprecise?  What does it mean?}
	 Let $q_{fb}(s)$ be the shadow price of government debt in state $s$ at the first best allocation.
	If
	\begin{equation}\label{eqn:prop52sufficient}
		\frac{1-q_{fb}(s_b)}{1-q_{fb}(s_g)} > \frac{g(s_b)}{g(s_g)}\geq 1 ,
	\end{equation}
		then there exists a steady state with $x_{fb}>x^*>0$
		\end{theorem}



\begin{theorem}
\label{prop: long run forces}
Consider an economy consisting of  two types of households with $%
\theta _{1,t}>\theta _{2,t}=0$. One period utilities are $\ln c-\frac{1}{2}%
l^{2}.$ The shock $s$  is i.i.d and takes  two values, $s\in \left\{
s_{L},s_{H}\right\}.$ We assume that $g\left( s\right) =g$ for all $s,$
and $\theta _{1}\left( s_{H}\right) >\theta _{1}\left( s_{L}\right) .$ We allow the discount factor $\beta(s)$ to depend on  $s.$

\smallskip Suppose that $g<\theta (s)$ for all $%
s.$ Let $R(s)$ be the gross interest rates and $x=U^2_c(s)\left[b_2(s)-b_1(s)\right]$

\begin{enumerate}
\item \textbf{Countercyclical interest rates.} If $\beta \left( s_{H}\right) =\beta \left( s_{L}\right)$, then
there exists a steady state $\left( x^{SS},\rho ^{SS}\right) $ such that $%
x^{SS}>0,\ R^{SS}\left( s_{H}\right) <R^{SS}\left( s_{L}\right) .$
\item \textbf{Acyclical interest rates.}  There exists a pair $\left\{ \beta \left( s_{H}\right) ,\beta \left( s_{L}\right)
\right\} $ such that there exists a steady state with $x^{SS}>0$ and $R^{SS}\left(
s_{H}\right) =R^{SS}\left( s_{L}\right)$.
\item \textbf{Procyclical interest rates.} There exists a pair  $\left\{ \beta \left( s_{H}\right) ,\beta \left( s_{L}\right)
\right\} $ such that there exists a steady state with $x^{SS}<0$ and  $R^{SS}\left( s_{H}\right) >R^{SS}\left( s_{L}\right) .$
\end{enumerate}
In all cases, taxes $\tau(s)=\tau^{SS}$ are independent of the realized state.
\end{theorem}

\begin{remark}
 We can extend this result to the case with payoff shocks instead of discount factor shocks
\end{remark}
		
\begin{theorem}
\label{theorem}

Suppose the Pareto weight of the agents are interior and we normalize the assets of the unproductive agent to zero. 
Let $n \in [0,1]$ denote the mass of the productive agents. The optimal tax, transfer and debt policy $\{\tau_t,T_t,B_t\}_t$ is characterized as follows:
\begin{enumerate}
\item As $n\to 0$,
\[\lim_t T_t=0,\quad \lim_t\tau_t=-\infty.\]
\item As $n\to 1$,
\[\lim_t T_t=\infty,\quad \lim_t\tau_t=0.\]
\end{enumerate}
In both cases limiting government assets, $\lim_tB_t=\infty$ 


\end{theorem}
\begin{corollary}
Let $\underline{B}>-\infty$ be the natural debt limit for the government. Suppose we impose an upper bound on assets, $B_t\leq\overline{B}<\infty$
\[\forall \epsilon>0, \quad \Pr\{B_t<\underline{B}+\epsilon \text{ and } B_t>\overline{B}-\epsilon \}=1\]
\end{corollary}

		
		
\begin{theorem}
Consider an economy with at least two agents that are strictly risk averse and a shock process such that there exists a steady state. There do no exists initial conditions i.e a distribution of asset and a choice of Pareto weights such that this allocation can be supported as a complete market Ramsey allocation.
\end{theorem}
		
		

\newpage
\section{Appendix}
\appendix
{\textbf{Theorem 1}}
\begin{proof}




The optimal Ramsey plan solves the following Bellman equation. Let $V(b\_)$ be the maximum ex-ante value the government can achieve with debt $b\_$. 

\begin{equation}
  \label{eq-QLRA obj}
    V(b\_)=\max_{c(s),l(s),b(s)} \sum_{s}\pi(s)\left\{c(s)-\frac{l(s)^{1+\gamma}}{1+\gamma}+\beta V(b(s)) \right\}
\end{equation}   
subject to

   \begin{subequations}
   \label{sys- QLRA constraint}
    \begin{equation}
    \label{rep agent implementability constraint }
    c(s)+b(s)=l(s)^{1+\gamma}+\beta^{-1} P(s)b\_
    \end{equation}
 

 
\begin{equation}
  \label{eq-resoruces}
c(s)+g(s)\leq\theta l(s)
\end{equation}   
Let $\bar{b}=-\underline{B}$
\begin{equation}
  \label{ndl}
b\leq \bar{b}
\end{equation}   


   \end{subequations}
Let $\mu(s)\pi(s),\phi(s)\pi(s)$ be the Lagrange multipliers on the respective constraints. 


\begin{lemma}
There exists  a $\overline{b}$ such that $b_t\leq\overline{b}$. This is the natural debt limit for the government.
\end{lemma}
\begin{proof}
As we drive $\mu$ to $-\infty$, the tax rate approaches a maximum limit, $\bar{\tau}=\frac{\gamma}{1+\gamma}$. In state $s$, the government surplus,
\[
  S(s,\tau) = \theta^\frac\gamma{1+\gamma}(1-\tau)^\frac1\gamma\tau - g(s),
\]  which  is  maximized at $\tau = \frac\gamma{1+\gamma}$ when $(1-\tau)^\frac1\gamma\tau$ is also maximized. This would impose a natural borrowing limit for the government. 

\end{proof}

We begin with some useful lemmas

\smallskip To make this problem convex, let $L\equiv l^{1+\gamma }.$ 

Substitute for $c\left( s\right) $%
\[
V\left( b\_\right) =\max_{L(s),b(s)}\sum_{s\in S}\pi \left(
s\right) \left[ \frac{1}{1+\gamma }L\left( s\right) +\frac{1}{\beta }%
P\left( s\right) b\_-b\left( s\right) +\beta V\left( b\left( s\right)
\right) \right] 
\]%
s.t.%
\begin{eqnarray*}
\frac{1}{\beta }P\left( s\right) b-b\left( s\right) +g\left( s\right) &\leq
&\theta L^{\frac{1}{1+\gamma} }\left( s\right) -L\left( s\right) \\
b\left( s\right) &\leq &\bar{b} \\
L\left( s\right) &\geq &0.
\end{eqnarray*}

\begin{lemma}
\smallskip $V\left( b\right) $ is stictly concave, continuous,
differentiable and $V\left( b\right) <\beta ^{-1}$ for all $%
b<\bar{b}.$ The feasibility constraint binds for all $b\in (-\infty ,\bar{b}%
],\ s\in S$ and $\left( L^{\ast }\left( s\right) \right) ^{1-\frac{1}{1+\gamma} }\geq
\frac{1}{1+\gamma} .\footnote{
This last condition simply means that we do not tax to the right of the peak
of the Laffer curve. The revenue maximizing tax is $1-\bar{\tau}=\frac{1}{%
1+\gamma }.$ At the same time $1-\tau =l^{\gamma}$ so if taxes are always
to the left of the peak, $\frac{1}{1+\gamma }\leq l^{\gamma }=\left(
L^{\frac{1}{1+\gamma} }\right) ^{\gamma}=L^{1-\frac{1}{1+\gamma} }$.}$ 
\end{lemma}

\begin{proof}
\smallskip \textit{Concavity}

$V\left( b\right) $ is concave because we maximize linear objective
function over convex set.

\textit{Binding feasibility}

Suppose that feasibility does not bind for some $b,s.$ Then the optimal $%
L\left( s\right) $ solve $\max_{L\left( s\right) \geq 0}\pi \left( s\right) 
\frac{\gamma}{1+\gamma }L\left( s\right) $ which sets $L\left( s\right)
=\infty .$ This violates feasility for any finite $b,b\left( s\right) .$

\textit{Bounds on }$L$

Let $\phi\left( s\right) >0$ be a Lagrange multiplier on the
feasibility. \ The FOC for $L\left( s\right) $ is 
\[
\frac{1}{1+\gamma }+\phi(s) \left( \frac{1}{1+\gamma }L(s)^{\frac{1}{1+\gamma}
-}-\theta \right) =0. 
\]%
This gives%
\[
\frac{1}{1+\gamma }L^{\frac{1}{1+\gamma} -1}-\theta =-\frac{1}{\lambda }\frac{\gamma}{%
1+\gamma }<0 
\]%
or%
\[
L^{1-\frac{1}{1+\gamma} }\geq \frac{\theta }{1+\gamma }. 
\]

\textit{Continuity}

For any $L$ that satisfy $L^{1-\frac{}{1+\gamma} }\geq \frac{\theta }{1+\gamma} ,$ define function $%
\Psi $ that satisfies $\Psi \left( L^{\frac{1}{1+\gamma} }-\theta L\right) =L.$ Since $%
L^{\frac{1}{1+\gamma} }-L$ is strictly decreasing in $L$ for $L^{1-\frac{1}{1+\gamma} }\geq
\frac{1}{1+\gamma} $, this function is well defined. Note that $\Psi \left(
{}\right) \underbrace{\left( \frac{1}{1+\gamma }L^{\frac{1}{1+\gamma} -1}-\theta \right) }%
_{<0}=1$ (so that $\Psi >0$, i.e. $\Psi $ is strictly decreasing)
and $\Psi ^{\prime \prime }\underbrace{\left( \frac{1}{1+\gamma }L^{\frac{1}{1+\gamma}
-1}-1\right) ^{2}}_{>0}+\underbrace{\Psi }_{<0}\underbrace{\frac{1%
}{1+\gamma }\frac{\gamma }{1+\gamma }L^{\frac{1}{1+\gamma} -2}}_{<0}=0$ (so that $\Psi
^{\prime \prime }\geq 0$, $\Psi ^{\prime \prime }>0,$ i.e. $\Psi $ is
strictly concave on the interior). $\Psi $ is also continuous. When $%
L^{1-\frac{1}{1+\gamma} }=\frac{1}{1+\gamma} ,$ $L=(1+\gamma) ^{-\frac{(1+\gamma)} {\left( \gamma \right)} }.$
Let $D\equiv (1+\gamma) ^{\frac{-1}{ \gamma }-(1+\gamma) ^{-\frac{1+\gamma }{\left(\gamma\right)} }}.$ Then the objective is 
\[
V\left( b\_\right) =\max_{b\left( s\right) }\sum_{s\in S}\pi \left(
s\right) \left[ \Psi \left( \frac{1}{\beta }P\left( s\right) b-b\left(
s\right) +g\left( s\right) \right) +\frac{1}{\beta }P\left( s\right)
b\_-b\left( s\right) +\beta V\left( b\left( s\right) \right) \right] 
\]%
s.t.%
\begin{eqnarray*}
b\left( s\right) &\leq &\bar{b} \\
\frac{1}{\beta }P\left( s\right) b\_-b\left( s\right) +g\left( s\right) &\leq
&D.
\end{eqnarray*}

This function is continuous so $V$ is also continuous.

\textit{Differentiability}

Continuity and convexity implies differentiability everywhere, including the
boundaries.

\textit{Strict concavity}

$\Psi $ is strictly concave, so on the interior $V$ is strictly
concave.

\end{proof}


\begin{lemma}
With $P(s)=1$, the multiplier on the implementability constraint,$\lim_t\mu_t=\mu*$
\end{lemma}
\begin{proof}
 

The envelope theorem together with the FOC with respect to $b(s)$ imply that the multiplier on the implementability constraint is a martingale

\[\mathbb{E}\mu_{t+1}=\mu_t\]

The FOC with respect to labor $l(s)$ can be expressed as 

\[\mu=\frac{\frac{\theta}{l^{\gamma}-1}}{\frac{\theta}{l^{\gamma}-(1+\gamma)}}\]. Note that $(1-\tau)^{-1}=\theta/l^{\gamma}$. Thus, as tax rates go to $-\infty$, $\mu$ approaches $\frac{1}{1+\gamma}$ from below.

Given this bound, the standard martingale convergence theorems imply that $\mu\to \mu^*$ almost surely. 
\end{proof}



We first show that  if $\mu^*=\frac{1}{1+\gamma}$, $b_t\to-\infty$ and then argue that $\mu^*$ cannot converge to any other value. 



Strict concavity of $V$ implies $\mu$ converges, either $b_t$ converges to a constant or diverges to $-\infty$. Suppose $\lim_tb_t$ is finite. The government's budget constraint would imply

\begin{equation}
\label{gbc}
\lim_t \tau_t\theta l_t-\lim_tg_t=(\beta^{-1}-1)\lim_t b_t
\end{equation}



However the left hand side of the previous expression diverges to $-\infty$. However under the hypothesis the right hand side is finite. This gives us a contradiction and hence $\lim_tb_t=-	\infty$. 

Now suppose $\mu^*<\frac{1}{1+\gamma}$. Again strict concavity of $V$ implies $\lim_t b^*>-\infty$.  However, labor supply and taxes will be constant and finite in the limit. Thus, the left hand side of \eqref{gbc} is stochastic and the right hand side approaches a constant. This yields a contradiction.


Thus, combining the above, $\mu^*=\frac{1}{1+\gamma}$ and $\lim_tb_t=-\infty$.







The second part of the theorem is to prove convergence to $B^*$ for a the class of payoffs that are perfectly aligned with the expenditure shocks. We first begin with some characterization of the policy rules
\smallskip Next we characterize policy functions

\begin{lemma}
\label{lem increasing b}
$b\left( b\_,s\right) $ is an increasing function of $b$ for all $s$ for all $%
\left( b\_,s\right) $ where $b\left( s\right) $ is interior.
\end{lemma}

\begin{proof}
Take the FOCs for $b\left( s\right) $ from the condition in the previous
problem. If $b\left( s\right) $ is interior%
\[
\Psi \left( \frac{1}{\beta }P\left( s\right) b\_-b\left( s\right)
+g\left( s\right) \right) =\beta V\left( b\left( s\right)
\right) . 
\]

Suppose $b_{1}<b_{2}$ but $b_{2}\left( s\right) <b_{1}\left( s\right) .$
Then from stict concavity%
\begin{eqnarray*}
V\left( b_{2}\left( s\right) \right) &<&V^{\prime
}\left( b_{1}\left( s\right) \right) \\
\Psi \left( \frac{1}{\beta }P\left( s\right) b_{2}-b_{2}\left(
s\right) +g\left( s\right) \right) &>&\Psi \left( \frac{1}{\beta }%
P\left( s\right) b_{1}-b_{1}\left( s\right) +g\left( s\right) \right) .
\end{eqnarray*}
\end{proof}



\begin{lemma}  Let $\mu(b,s)$ be the optimal policy function for the Lagrange multiplier $\mu(s)$.  If $P(s') > P(s'')$ then there exists a $b^*_{s',s''}$ such that for all $b < (>) \; b_{1,s',s''}$ we have $\mu(b,s') > (<) \;\mu(b,s'')$.  If $\underline b < b^*_{s',s''} < \overline b$ then $\mu(b^*_{s',s''},s') = \mu(b^*_{s',s''},s'')$.
\label{lem.order}
\end{lemma}
\begin{proof} 
Suppose that $\mu(b,s')\leq \mu(b,s'')$.  Subtracting the implementability for $s''$ from the implementability constraint for $s'$ we have 
\begin{align*}
	\frac{P(s')-P(s'')}{\beta}b &= S_{s'}(\mu(b,s'))-S_{s''}(\mu(b,s'')) + b'(b,s')-b'(b,s'')\\
						&\geq S_{s'}(\mu(b,s')) -S_{s''}(\mu(b,s')) + b'(b,s')-b'(b,s'')\\
						&\geq  S_{s'}(\mu(b,s')) -S_{s''}(\mu(b,s')) = g(s'')-g(s')
\end{align*}  We get the first inequality from noting that $S_s(\mu')\geq S_s(\mu'')$ if $\mu' \leq \mu''$.  We obtain the second inequality by noting that $\mu(b,s')\leq \mu(b,s'')$ implies $b'(b,s')\geq b'(b,s'')$ (which comes directly from the concavity of $V$).
Thus, $\mu(b,s')\leq \mu(b,s'')$ implies that 
\[
	b \geq \frac{\beta(g(s'')-g(s'))}{P(s')-P(s'')} = b^*_{s',s''}
\]The converse of this statement is that if $b<b^*_{s',s''}$ then $\mu(b,s') > \mu(b,s'')$.  The reverse statement that $\mu(b,s') \geq \mu(b,s'')$ implies $b \leq b^*_{s,s'}$ follows by symmetry.   Again, the converse implies that if $b > b^*_{s',s''}$ then $\mu(b,s') < \mu(b,s'')$.    Finally, if $\underline b < b^*_{s',s''} <\overline b$ then continuity of the policy functions implies that there must exist a root of $\mu(b,s')-\mu(b,s'')$ and that root can only be at $b^*_{s',s''}$.
\end{proof}

The FOC and the envelope theorem now modify the martingale equation for $\mu_t$ as follows,
\begin{equation}
\label{eq.mart}
\sum_{s}\pi(s)P(s)\mu(s)=\mu\_
\end{equation}

With Lemma \ref{lem.order} we can order the policy functions $\mu(b,\cdot)$ for particular regions of the state space.  Take $b_1$ to be
\[
	b_1 = \min\left\{b^*_{s',s''}\right\}
\] and WLOG choose $\underline b < b_1$.  For all $b < b_1$ we have shown that $P_s > P(s')$ implies that $\mu(b,s) > \mu(b,s')$.  By decomposing $\EE \mu_{t+1}P_{t+1}$ in equation \eqref{eq.mart}, we obtain (using $\EE_t P_{t+1} = 1$)
\begin{equation}
	\mu_t = \EE\mu_{t+1} +\cov_t(\mu_{t+1},P_{t+1})
\end{equation}Our analysis has just shown that for $b_t < b_1$ we have $\cov_t (\mu_{t+1},P_{t+1})  >0$ so 
\[
	\mu_t > \EE_t\mu_{t+1}.
\]  If $p$ is sufficiently volatile:
\[
	P(s') - P(s'') > \frac{\beta(g(s'')-g(s'))}{\overline b}
\] then 
\[
	b_2 = \max\left\{b^*_{s',s''}\right\} <\overline b
\] and through a similar argument  we can conclude that $\cov_t(\mu_{t+1},P_{t+1}) < 0$ 
\[
	\mu_t < \EE_t \mu_{t+1}
\] for $b_t > b_2$


\begin{lemma}
For payoff structures of the form
if \[
  P(s) = 1 + \alpha(g(s)- \EE g)\]
 then  there exists a unique $b^*$ such that $b(b^*,s)=b^*$ for all $s$  
\end{lemma}
\begin{proof}
This follows from taking differences of the \eqref{rep agent implementability constraint } for $s'$,$s''$. We have 
\[[P(s)-P(s'')]\frac{b^*}{\beta}=g(s)-g(s'')\]
We have used the fact that if $b(b^*,s)$ is invariant across states, $\mu(s)=\mu^*$ and this implies that tax rates are constant. Thus, the fluctuations in the surplus of the government only come from $g(s)$.

\end{proof}

Since are payoffs $P(s)$ in this second part are linear in $g(s)$, we can immediately see that by setting $\alpha=\frac{\beta}{b^*}$
 


\begin{lemma}
\label{sub super martingale}
$\mu_t$  is a sub-martingale bounded above in the region $(-\infty,\mu^*)$ and super-martingale bounded below in the region $(\mu^*,\frac{1}{1+\gamma})$


\end{lemma}
\begin{proof}
Let $\mu^*$ be the associated multiplier, i.e $V_b(b^*)=\mu^*$.  Using the results of the previous section, we have that $b_1 = b_2 = b^*$, implying that $\mu_t < (>) \EE_t\mu_{t+1}$ for $b_t  < (>) b^*$. 
\end{proof}

Lastly we show that $\lim_t \mu_t=\mu^*$. Suppose $b_{t}<b^*$, we know that $\mu_t>\mu^*$. The previous lemma implies that in this region, $\mu_t$ is a super martingale. The lemma $\ref{lem increasing b}$ shows that $b(b\_,s)$ is continuous and increasing. This translates into $\mu(\mu(b\_),s)$ to be continuous and increasing as well.
 Thus 
 \[\mu_{t}>\mu^* \implies \mu(\mu_{t},s_{t+1})>\mu(\mu^*,s_{t+1}) \]
or
 \[\mu_{t+1}>\mu^*\]
Thus $\mu*$  provides a lower bound to this super martingale. Using standard martingale convergence theorem converges. The uniqueness of steady state implies that it can only converge to $\mu^*$. For $\mu<\mu^*$, the argument is symmetric.



\end{proof}

{\textbf{Corollary 1}}


\begin{proof}
Here we show that for $P(s)=1$ we show that with probability one, $b_t$ gets arbitrary close to both the boundaries. Adding a lower bound to debt imposes a constraint
\begin{equation}
\label{bound on b}
b(s)\geq\underline{b}
\end{equation}
The foc with respect to $b(s)$ will now be modified to

\[\beta \pi(s) V'(b(s))+\pi(s)\kappa(s)=\mu(s)\]
where $\kappa(s)$ is the Lagrange multiplier on the \eqref{bound on b}.  We first show a lemma that allows us to strictly order $b(b\_,s)$ across $s$

\begin{lemma}
\label{prop: b(s) relative to b}For any $b\_\in (-\underline{b} ,\bar{b}),$ there are $s,s^{\prime \prime }$ s.t. $b\left( s\right) \geq b\_\geq b\left( s^{\prime \prime }\right) .$ Moreover, if there are any states $s^{\prime \prime },s^{\prime \prime \prime }$ s.t. $b\left( s^{\prime \prime
}\right) \neq b\left( s^{\prime \prime \prime }\right) ,$ those inequalities
are strict.
\end{lemma}


\begin{proof}
The FOCs together with the envelope theorem imply that $\mathbb{E}P(s)V'(b(s))=V'(b\_)+\kappa(s)$
We can rewrite this as $\mathbb{\tilde{E}}V'(b(s))=b+\kappa(s)$ with $\tilde{\pi}(s)=P(s)\pi(s)$

Now if there is at least one $b\left( s\right) $ s.t. $b\left(
s\right) >b\_,$ by strict concavity of $V$ there must be some $%
s^{\prime \prime }$ s.t. $b\left( s^{\prime \prime }\right) <b.$

If there is at least one $b\left( s\right) $ s.t. $b\left(
s\right) <b\_,$ the inequality above is strictly only if $b\left(
s^{\prime \prime \prime }\right) =\bar{b}$ for some $s^{\prime \prime \prime
}.$ But $V\left( \bar{b}\right) <V\left(
b\right) $ so there must be some $s^{\prime \prime }$ s.t. $b\left(
s^{\prime \prime }\right) >b.$ Equality is possible only if $b\_=b\left(
s\right) $ for all $s.$
\end{proof}

For payoffs $P(s)=1$, it is easy to see that there does not exists any $b^*$ in the interior that has the property $b(b^*,s)=b^*$. Thus, the relevant inequalities in the previous lemma are strict. This allows us to construct a sequences $\{b_t\}_t $ such that $b_t<b_{t+1}$ with the property that $\lim_tb_t=\underline{b}$.
Thus, for any $\epsilon>0$, there exists a finite history of shocks that can take us arbitrarily close to $\underline{b}$. Since the shocks are i.i.d this finite sequence will repeat i.o. With a symmetric argument we can show that $b_t$ will come arbitrarily close to its upper limit i.o too




The second part of the corollary states that $P(s)=1+\frac{\beta}{B^*}(g(s)-\mathbb{E}g)$ then we never approach the boundaries. This essentially follows from lemma \ref{sub super martingale}. We have the covariance $\cov(\mu_{t+1},P(s_{t+1})<0$ for $\mu<\mu^*$ and vice versa. The only change is that the martingale equation for $\mu$ will read
\[\mathbb{E}\mu_{t+1}=\mu_t-\cov(\mu_{t+1},P(s_{t+1})-\kappa_{t+1}\]

Note that for $\mu<\mu^*$ $\kappa_{t+1}$ will be zero and the analysis is same as before. For $\mu>\mu^*$ we have the covariance term positive and the Lagrange multiplier positive too. Thus
\[\mathbb{E}\mu_{t+1}<\mu_t\]
Thus, it remains a super-martingale drifting towards $\mu^*$.
\end{proof}



{\textbf{Theorem 2}}
\begin{proof}
The first order conditions for a planning problem given portfolio $p_s$ are given by
\begin{align*}
	\frac{b p_s}{\beta \EE p}= S(\mu'(s),s) + b'(s)\\
	V'(b) = \frac{\EE p \mu'}{\EE p}\\
	\mu'(s) = V'(b'(s))
\end{align*} where $S(\mu,s)$ is the government surplus in state $s$ given by
\[
	S(\mu,s) = (1-\tau(\mu))^\frac1\gamma \tau(\mu)-g(s) = I(\mu) - g(s)
\]where
\[
	\tau(\mu) =\frac{\gamma\mu}{(1+\gamma)\mu-1}
\]We note that $V'(b)$ is one-to-one, so we can re-characterize these equations as searching for a function $b(\mu)$ such that the following two equations can be solved for all $\mu$.
\begin{align}\label{eq.lin_imp}
	\frac{b(\mu)p_s}{\beta \EE p} = I(\mu'(s)) - g(s) +b(\mu'(s))\\
	\mu = \frac{\EE\mu' p}{\EE p}\label{eq.lin_mart}
\end{align}  This defines, implicitly, a function $b(\mu; p)$ .  For a given $\overline \mu$ define 
\[
	\overline b = \frac{\beta}{1-\beta}\left( I(\overline\mu) - \overline g\right)
\]where $\overline g = \EE g$ and $\overline p$ as 
\[
	\overline p_s = 1+ \frac\beta{\overline b}(g_s - \overline g)
\]  As noted before $b(\overline\mu;\overline p) = \overline b$ solves the the system of equations (\ref{eq.lin_imp}-\ref{eq.lin_mart}) for $\mu'(s) = \overline \mu$.  We can linearize around this steady state with respect to both $\mu$ and $p$.  Differentiating equation \eqref{eq.lin_imp} with respect to $\mu$ around $(\overline \mu,\overline p)$ we obtain
\[
	\frac{\pbar_s}{\beta}\frac{\partial b}{\partial \mu} = \left[I'(\mubar)+\frac{\partial b}{\partial \mu}\right]\frac{\partial \mu'(s)}{\partial \mu}.
\]Differentiating equation \eqref{eq.lin_mart} with respect to $\mu$ we obtain
\[
	1 = \sum_{s'} \Pi_s \overline p_s \frac{\partial \mu'(s)}{\partial \mu}
\]combining these two equations we see that 
\[
	\frac1\beta\left(\sum_s\Pi_s\pbar_s^2\right)\frac{\partial b}{\partial \mu} = I'(\mubar) + \frac{\partial b}{\partial \mu}
\]Noting that $\EE\overline p^2 = 1 + \frac{\beta^2}{\bbar^2}\sigma^2_g$ we obtain
\begin{equation}
	\frac{\partial b}{\partial \mu} = \frac{I'(\mubar)}{\frac{\beta}{\bbar^2}\sigma_g^2 +\frac{1-\beta}{\beta}} < 0
\end{equation}as $I'(\mubar) < 0$.  We then have directly that 
\begin{equation}
	\frac{\partial \mu'(s)}{\partial \mu} = \frac{\overline p_s}{\frac{\beta^2}{\overline b^2}\sigma^2_g +1} = \frac{\pbar_s}{\EE\pbar^2}
\end{equation}  We can perform the same procedure for $p_s$.  Differentiating equation \eqref{eq.lin_imp} with respect to $p_s$ we around $(\mubar,\pbar)$ we obtain
\begin{equation}\label{eq.dimp_dps}
\frac{\pbar_{s'}}{\beta}\frac{\partial b}{\partial p_s} + 1_{s,s'}\frac{\bbar}{\beta} - \frac{\Pi_s\bbar\pbar_{s'}}{\beta} = \left[I'(\mubar) + \frac{\partial b}{\partial \mu}\right]\frac{\partial \mu'(s')}{\partial p_s}
\end{equation} Here $1_{s,s'}$ is $1$ if $s = s'$ and zero otherwise.  Differentiating equation \eqref{eq.lin_mart} with respect to $p_s$ we obtain
\[
	0 = \Pi_s\mubar - \Pi_s\mubar + \sum_{s'} \Pi_{s'}\pbar_s\frac{\partial \mu'(s')}{\partial p_s} =  \sum_{s'} \Pi_{s'}\pbar_s\frac{\partial \mu'(s')}{\partial p_s}
\]  Again we can combine these two equations to give us
\[
	\frac{\EE\pbar^2}{\beta}\frac{\partial b}{\partial p_s} + \frac{\Pi_s\bbar}{\beta}(\pbar_s-\EE\pbar^2) = 0
\] or
\begin{equation}
	\frac{\partial b}{\partial p_s} = \Pi_s\bbar \frac{\EE\pbar^2-\pbar_s}{\EE\pbar^2}
\end{equation}Going back to equation \eqref{eq.dimp_dps} we have
\begin{equation}
	\frac{\partial \mu'(s')}{\partial p_s} = \frac{\bbar}{\beta\left[I '(\mubar) + \frac{\partial b}{\partial \mu}\right]}\left(1_{s,s'}-\frac{\Pi_s\pbar_s\pbar_{s'}}{\EE\pbar^2}\right)
\end{equation}
\subsection{Ergodic Distribution of the Linearized System}
The linearized system for $\mu$ now follows
\[
	\hat \mu_{t+1} = B \hat\mu_t + C
\]  where $B$ and $C$ are both random with means $\barB$ and $\barC$, and variances $\sigma_B^2$ and $\sigma_C^2$ .  Suppose that $\hat\mu$ is distributed according to the ergodic distribution of this linear system with mean $\EE\hat\mu$ and variance $\sigma^2_\mu$.  Since 
\[
	B\hat\mu +C
\]has the same distribution we can compute the mean of this distribution as
\[
\begin{split}
	\EE\hat\mu &= \EE\left[ B\hat\mu+C\right]\\
			  &= \EE\left[\EE_{\hat\mu}\left[B\hat\mu+C\right]\right]\\
			  &= \EE\left[\barB\hat\mu +\barC\right]\\
			  &=\barB\EE\hat\mu+\barC
\end{split}
\]solving for $\EE\hat\mu$ we get
\begin{equation}
	\EE\hat\mu = \frac{\barC}{1-\barB}
\end{equation}For the variance $\sigma^2_{\hat\mu}$ we know that 
\[
	\sigma^2_{\hat\mu} = \var(B\hat\mu+C) = \var(B\hat\mu) + \sigma_C^2 + 2\cov(B\hat\mu,C)
\]Computing the variance of $B\hat \mu$ we have
\[
\begin{split}
	\var(B\hat\mu) &=\EE\left[(B\hat\mu - \barB\EE\hat\mu)^2\right]\\
			       &=\EE\left[(B\hat\mu-\barB\hat\mu +\barB\hat\mu -\barB\EE\hat\mu)^2\right]\\
			      &=\EE\left[\EE_{\hat\mu}\left[(B-\barB)^2\hat\mu^2 +2(B-\barB)(\hat\mu-\EE\hat\mu)\barB\EE\hat\mu + (\hat\mu-\EE\hat\mu)^2\bar B^2\right]\right]\\
			&=\EE\left[\sigma_B^2\hat\mu^2 +(\hat\mu-\EE\hat\mu)^2\barB\right]\\
			& = \sigma_B^2(\sigma_{\hat\mu}^2+(\EE\hat\mu)^2) + \sigma_{\hat\mu}^2\barB^2
\end{split}
\]while for the covariance of $B\hat\mu$ and $C$
\[
	\cov(B\hat\mu,C) = \sigma_{BC}\EE\hat\mu
\]Putting this all together we have
\begin{equation}
	\sigma_{\hat\mu}^2 = \frac{\sigma_B^2(\EE\hat\mu)^2 + \sigma_{BC}\EE\hat\mu + \sigma_C^2}{1-\barB^2-\sigma_B^2}
\end{equation}
\subsection{What is the best $\pbar$}
We want to study the properties of the ergodic distribution of an economy with payoff structure $ p_s$, normalized so that $\EE p = 1$.  The immediate question to ask is where we should linearize around?  The natural answer is to choose $\mubar$ so as to minimize 
\[
\| p-\pbar(\mubar)\|^2 = \sum_{s'}\Pi_{s'}( p_{s'}-\pbar(\mubar)_{s'})^2.
\]That is to choose $\mubar$ so as to minimize the variance of the difference between $ p$ and the set of steady state payoffs.  We shall see how this is the ``best'' choice by another metric.  The first order condition for this linearization gives us 
\[
	2\sum_{s'}\Pi_s'( p_{s'}-\pbar(\mubar)_{s'}) \pbar'(\mubar)_{s'} = 0
\]as noted before 
\[
	\pbar(\mubar)_s =  1 -\frac{\beta}{\bbar(\mubar)}\left(g_s - \EE g\right)
\]thus
\[
	\pbar'(\mubar) \propto \pbar(\mubar)-1
\]Thus, we can see the the optimal choice of $\mubar$ is equivalent to choosing $\mubar$ such that 
\begin{equation}
	\begin{split}
		0 &= \sum_{s'}\Pi_{s'}( p_{s'} - \pbar(\mubar)_{s'})(\pbar(\mubar)_{s'}-1)\\
		&= -\sum_{s'}\Pi_{s'}( p_{s'}-\pbar(\mubar)_{s'}) + \sum_{s'}\Pi_{s'}( p_{s'}-\pbar(\mubar)_{s'})\pbar(\mubar)_{s'}\\
		&= \sum_{s'}\Pi_{s'}( p_{s'}-\pbar(\mubar)_{s'})\pbar(\mubar)_{s'}\\
		&=\EE\left[( p-\pbar(\mubar))\pbar(\mubar)\right]
	\end{split}
\end{equation}  Using this $\pbar$ and $\mubar$ we have that $C$ for our linearized system is
\[
	C_{s'} = \sum_s\left\{\frac{\bbar}{\beta\left[I'(\mubar)+\frac{\partial b}{\partial\mu}\right]}\left(1_{s,s'}-\frac{\Pi_s \pbar_s\pbar_{s'}}{\EE\pbar^2}\right)({p}_s-\pbar_s) \right\}
\]Taking expectations we have that 
\begin{equation}
\begin{split}
	\barC &= \sum_s\left\{\frac{\bbar}{\beta\left[I'(\mubar)+\frac{\partial b}{\partial\mu}\right]}\left(\Pi_s - \frac{\Pi_s\pbar_s}{\EE\pbar^2}\right)( p_s-\pbar_s)\right\}\\
	&=\frac{\bbar}{\beta\left[I'(\mubar)+\frac{\partial b}{\partial\mu}\right]}\left(\EE( p-\pbar) -\frac{\EE\left[( p-\pbar)\pbar\right]}{\EE\pbar^2}\right)\\
	&= 0 
\end{split}
\end{equation}  Thus, the linearized system will have the same mean for $\mu$, $\mubar$, as the closest approximating steady state payoff structure.

We can also compute the variance of the ergodic distribution for $\mu$.  Note 
\[
\begin{split}
	C_{s'} &= \sum_s\left\{\frac{\bbar}{\beta\left[I'(\mubar)+\frac{\partial b}{\partial\mu}\right]}\left(1_{s,s'}-\frac{\Pi_s \pbar_s\pbar_{s'}}{\EE\pbar^2}\right)({p}_s-\pbar_s) \right\}\\
		 &=\frac{\bbar}{\beta\left[I'(\mubar)+\frac{\partial b}{\partial\mu}\right]}\left( p_{s'}-\pbar_{s'} -\pbar_{s'}\frac{\sum_s\Pi_s\pbar_s( p_s-\pbar_s)}{\EE\pbar^2}\right)\\
		&= \frac{\bbar}{\beta\left[I'(\mubar)+\frac{\partial b}{\partial\mu}\right]}( p_{s'}-\pbar_{s})
\end{split}
\]  As noted before
\[
	\sigma_{\mu}^2 = \frac{\bbar^2}{\beta^2\left[I'(\mubar)+\frac{\partial b}{\partial\mu}\right]^2\left(1-\barB^2-\sigma_B^2\right)}\| p-\pbar\|^2
\]  The variance of government debt in the linearized system is 
\[
	\sigma_b^2 = \frac{\bbar^2\left(\frac{\partial b}{\partial\mu}\right)^2}{\beta^2\left[I'(\mubar)+\frac{\partial b}{\partial\mu}\right]^2\left(1-\barB^2-\sigma_B^2\right)}\| p-\pbar\|^2
\]  This can be simplified using the following expressions: 
\[
	I'(\mubar)+\frac{\partial b}{\partial \mu} = \frac{\EE\pbar^2}{\beta}\frac{\partial b}{\partial\mu},
\]
\[
	\barB = \frac{1}{\EE\pbar^2}
\]and
\[
	\sigma_B^2 = \frac{\var(\pbar)}{(\EE\pbar^2)^2}
\] to
\begin{equation}
	\sigma^2_b = \frac{\bbar^2}{\EE\pbar^2\var(\pbar)}\|p-\pbar\|^2
\end{equation}Noting that $\EE\pbar^2 = 1 +\var(\pbar) > 1$, we have immediately that up to first order the relative spread of debt is bounded by
\begin{equation}
	\frac{\sigma_b}{\bbar} \leq \sqrt\frac{\|p-\pbar\|^2}{\var(\pbar)}
\end{equation}  Thus, as $p$ approaches a the steady state payoff vector the ergodic distribution becomes degenerate.
\end{proof}


{\textbf{Theorem 3}}

\begin{proof}

Suppose the unproductive agent has CRRA utility function with risk aversion $\sigma$. Under these assumption we can formulate the Bellman equation that solves for the Ramsey plan as follows:


\begin{equation}
  \label{eq-QLRA obj}
    V(b\_,\rho\_)=\max_{c_1(s),c_2(s),\rho(s), b(s)} \sum_{s}\pi(s)\left\{\omega \left[u(c_1(s),l_1(s))\right]+(1-\omega )\left[\frac{c_2(s)^{1-\sigma}}{1-\sigma}\right]+\beta V(b(s),\rho(s)) \right\}
\end{equation}   
subject to

   \begin{subequations}
   \label{sys- QLRA constraint}
    \begin{equation}
    \label{eq-implementability constraint}
    c_1(s)-c_2(s)+b(s)=l(s)^{1+\gamma}+\beta^{-1} b\_
    \end{equation}
 
\begin{equation}
   \label{eq-ee}
\mathbb{E}c_2^{-\sigma}(s)=\rho\_
\end{equation}   

\begin{equation}
   \label{eq-definition rho}
   \rho(s)=c_2^{-\sigma}(s)
\end{equation}   

 
\begin{equation}
  \label{eq-resoruces}
    n c_1(s)+(1-n)c_2(s)+g(s)\leq\theta_2 n l(s)
\end{equation}   


   \end{subequations}

Let $\mu(s)\pi(s),\lambda\_,\kappa(s)\pi(s),\phi(s)\pi(s)$ be the Lagrange multipliers on the respective constraints. The FOC and the envelope conditions are summarized below


\begin{subequations}
   \label{sys-FOC QLRA}
   \begin{equation}
   \label{eq- foc c1 QLRA}
    \phi(s)=\omega-\frac{\mu(s)}{n}
   \end{equation}

   \begin{equation}
   \label{eq-foc c_2 QLRA}
   c_2(s)^{1-\sigma}=\frac{\phi(s)-\omega }{1-\omega}-\sigma c_2(s)^{-\sigma-1}\left(\lambda\_ \pi(s)+\beta\lambda[s]\right)
   \end{equation}

    \begin{equation}
\label{eq-foc l_1 QLRA}
   \mu(s)= \frac{\omega\left[\frac{\theta}{l^{\gamma}(s)}-1\right]  }{\left[\frac{\theta}{l^{\gamma}(s)}-1-\gamma\right]}
    \end{equation}

    \begin{equation}
\label{eq-foc b(s) QLRA}
   \mu\_=\mathbb{E}\mu(s)
    \end{equation}    
 \end{subequations}

We begin with some useful lemmas
\begin{lemma}
\label{lem-convergence mu}
The multiplier on the implementability constraint \eqref{eq-implementability constraint} $\mu_t\to\mu^*$ a.s
\end{lemma}

\begin{proof}
Notice that the labor choice of the productive household implies $\frac{1}{1-\tau}=\frac{\theta_2}{l^{\gamma}(s)}$. 

As taxes go to $-\infty$ \eqref{eq-foc l_1 QLRA} implies that $\mu(s)$ approaches $\frac{\omega}{1+\gamma}$ from below. This provides us an upper bound .
\[\mu(s)\leq  \frac{\omega }{(1+\gamma)}\].
Applying the standard martingale convergence theorems we get the result
\end{proof}

\begin{lemma}
There exists  a $\overline{b}$ such that $b_t\leq\overline{b}_n$. This is the natural debt limit for the government.
\end{lemma}
\begin{proof}
As we drive $\mu$ to $-\infty$, the tax rate approaches a maximum limit, $\bar{\tau}=\frac{\omega \gamma}{1+\gamma}$. Further transfers are bounded below by zero. Applying the same steps in Lemma \ref{lem-natural debt limit} we get the natural debt limit for the government.
\end{proof}



\begin{lemma}
\label{lem-slack EE}
\[\lim_t \lambda_t=0\]
\end{lemma}
\begin{proof}
Equations \eqref{eq-ee} and \eqref{eq-definition rho} imply that $E\rho_{t+1}=\rho_t$ and the Inada conditions provide us a lower bound on $\rho_t$. Thus, again applying the Martingale convergence theorem we have $\rho_t\to \rho^*$ almost surely. This also implies that $E (\rho_{+1}-\rho_t)\to 0$. Thus asymptotically \eqref{eq-ee} is slack and the multiplier $\lim_t\lambda_t=0$.
 \end{proof}

 \noindent Now we can prove the main theorem,

 
Under the normalization in the theorem we have $c_t=T_t$ and $-B_t=b_t$ is the government debt. 
 
Lemma \ref{lem-slack EE} shows $\lim_t\lambda_t=0$. Now we verify the first order constraints along with feasibility of the resulting allocation. By Lemma \ref{lem-convergence mu}, $\mu_t$ and $\phi_t$ converge to $\mu^*$ and $\phi^*$ respectively. Note that \eqref{eq-foc c_2 QLRA} implies that $\phi^*\geq \omega$. This provides us with another bound on $\mu^*$. 
\[\mu^*\leq  \min \left \{\frac{\omega }{(1+\gamma)},(1-n)\omega \right\}\].
We first show that $\mu^*= \min\{(1-n)\omega,\frac{\omega}{1+\gamma}\}$. Suppose $\mu^*$ is strictly lower than the bound. Then we have that  $c_{2,t}$ converges to a finite positive number $c_2^*$ and $\lim_tc_{1,t}=\frac{n\theta l^*-g(s_t)-(1-n)c_2^*}{n}$.The implementability constraint \eqref{eq-implementability constraint} further implies that $\lim_tb_t$ has to diverge. Suppose $\lim_tb_t=\infty$, this yields a contradiction as the maximum revenue the government can collect is finite in this economy. On the other hand if $\lim_tb_t=-\infty$, we will violate the productive agent's natural debt limit. This is because $\tau^*>-\infty$ and $T^*<\infty$ and hence his post tax income is bounded. \footnote{Since we do not impose  non-negativity constraint on his consumption, in prnciple he could sustain arbitrary debt with arbitrary negative consumption, but under the current guess $\min{c_t}>-\infty$.} 

Now we have two cases to consider for $\mu^*$. Suppose $\mu^*=(1-n)\omega$. This implies $\phi^*=\omega$ and $c_2^*=\infty$. This makes limiting transfers,$T^*$ diverge to infinity and we can construct a solution where $\lim_tb_t=-\infty$. However tax rates will be bounded and given by 
\[\tau^*=\frac{-\gamma (1-n)}{1-(1+\gamma)(1+n)}\]
\footnote{Under this solution $c_{1,t}$ diverges to $-\infty$ and we do not violate any natural debt limits for the productive agent.}

Lastly if $\mu^*=\frac{\omega}{1+\gamma}$, $\tau^*\to -\infty$. In this case, we have $c_2^*$ finite and again the government debt $b_t$ diverges to $-\infty$. The transfers are given by the following expression,
\[T^*=\left(\frac{\omega(1-n)-\frac{\omega}{1+\gamma}}{n(1-\omega)}\right)^{-\frac{1}{\sigma}}\]

Which of this case characterizes the solution depends on $n$ relative to $\frac{\gamma}{1+\gamma}$ and is independent of $\omega$. As $n\to 0$ we have $\frac{\omega}{1+\gamma}<\omega$ and conversely when $n \to 1$, we have $\omega(1-n)<\frac{\omega}{1+\gamma}$

\end{proof}

\textbf{Corollary 1}
\begin{proof}
\end{proof}


\textbf{Theorem 4}
\begin{proof}
\end{proof}

\textbf{Corollary 2}
\begin{proof}
\end{proof}


\end{document}



 Suppose $P(s)=1\end{lemma}




\begin{theorem}
 Suppose $P(s)=1$ and we have an lower bound on debt $\underline{b}>-\infty$. For $n<1$ there is an invariant distribution $\psi .$ Morever, for any $\hat{b}\in \left( \underline{b},\bar{b}_n\right) ,$ $\psi \left( \left[ \underline{b},\hat{b}%
\right] \right) >0$ and $\psi \left( \left[ \hat{b},\bar{b}_n\right] \right)$. 

If $n>\frac{\gamma}{1+\gamma}$, there exists a $\underline{\tau}$ independent of $\underline{b}$ such that
 \begin{itemize}
  \item $\tau_t\geq\underline{\tau}>-\infty$, and
  \item As $\underline{b} =-\infty$, we have $\tau_t\to \underline{ \tau}$ a.s
 \end{itemize}

 \end{theorem}

 
 
\begin{proof}
Consider the case when $\underline{b}=-\infty$.   Lemma \ref{lem-bounds on multipliers} gives us bounds on $\mu_t$. The super-martingale converge theorem implies $\lim_t\mu_t=\mu^*$. First we show that $b_t\to-\infty$

\[\mu^* \leq \min \left \{\frac{\omega (1-n)}{n},\frac{\omega }{(1+\gamma)} \right\}\]

Suppose not. As $\mu_t\to \mu^*$. Thus taxes, labor supply and output converges to a constant.  

If $n<\frac{\gamma}{1+\gamma}$, $\mu_t\to\frac{\omega}{1+\gamma}$. In this case taxes diverge to $-\infty$. Also$\phi_t\to \phi^*=\frac{\omega \gamma}{n(1+\gamma)} >\omega$. Thus  $T_t \to T^*<\infty$ and The fluctuations $g(s)$ are absorbed by $c_1(s)$. With sufficient stochasticity of $g(s)$, the implementability constraint implies that $b_t$ will violate any bounds. 

Now if $n>\frac{\gamma}{1+\gamma}$, the multiplier $\mu_t$ converges to $\frac{\omega (1-n)}{n}$. The limiting taxes $\underline{\tau} $ can be obtained from the FOC \eqref{eq-foc l_1 QLRA}. In this case $


T_t\to \infty$ and $c_{1,t}\to-\infty$.  

\end{proof}


\begin{remark}
 The results for the perfectly aligned payoffs and approximation results are analogously extended
\end{remark}
