Email number 1 from Annmol, Dec 21

Suppose we dont impose c_2>0, the planner's problem is concave only for omega < n(1+gamma)/gamma.

The solution here entails  a finite tax that is decreasing in omega, constant debts and transfers do all the job.  The tax rate  approaches -infty as omega approaches the threshold. At this point transfers are negative infinity, however the since the objective function is decreasing in taxes it reaches + infinity.

Now for omega > n(1+gamma)/gamma the argmax is not defined as the problem is no longer concave and hence there is no solution. however we can make a limiting argument.

Suppose we impose an arbirtary lower bound on taxes tau_bar. Then for omega higher than some threshold omega_bar the optimal policy will keep tax rates at this lower bond for ever.

The following statement links this limiting argument to the previous discussion: As this lower bound on tax rates tau_bar approaches -infty the threshold approaches n(1+gamma)/gamma

whats happening here an outcome of two things
1] welfare is decreasing in taxes
2] taxes can reach -infinity for high enough but interior pareto weight of the productive agent.


%%%%% additional response to my e-mail %%%%%%%%%%%%%%%%



I have a Ramsey problem written down for this case in my notes. I can type it out and we can booth have a look. But as far as your first comment goes .. Period by period the billowing happens : the government comes in with some initial assets, imposes a constant tax (which pins down labor supply, output) and rolls over the exact same level of assets. Given the expenditure shock this pins down transfers andin turn consumption for the unproductive agent. The consumption for e productive agent is obtained by calculating his income after taxes, transfers and interest payments on the constant level of debt.
The Pareto weights directly map into the tax rates and the net of interest deficit that is balanced by transfers.

I agree with you that making the connections with AMSS can be helpful for a lot of readers. There were other results that got neglected where we did he case with a risk averse agent unproductive agent. There we don't have as many results as the quasilinear case but still can say something about the limiting taxes.


%%%%%%%%%%%%%%  further response after we worked out by hand the quasi-linear two agent with no inequality constraint  %%%%%%%%%%%%%%%%%%%%

Dear Sir,
I will try to construct these modules. Can we talk about it a bit more- I started scribbling something and found myself more or less repeating what we already have in the paper.

The takeaway for me from the version of our model with no restriction on c_2t was that it effectively makes the market structure irrelevant and we have no dynamics in debt; similiar to a version with complete markets. The only difference worth pointing is that the tax rate is independent of the initial distribution of debt in the case with unrestricted transfers but not in the complete market economy.

This is exactly the same insight that would come from the following two versions of  a representative agent model:

1] A representative agent economy with no lump sum transfers and complete markets

2] A representative agent economy with arbitrary market structure but no restrictions on lump sum transfers

Either complete markets or unrestricted transfers make the dynamics of asset holdings trivial. With QL preferences asset holdings are constant at their initial level.  Again the difference being in the dependence of tax rates on initial conditions.


Now what about the intermediate cases? what if markets are less than complete and transfers are restricted? The answer has to depend on "extent of incompleteness" and "welfare costs of transfers". Our paper  with the two modelling ingredients : Payoff shocks and  agents heterogeneous in productivity make these ideas precise.




We can talk more if things are still unclear. I will write you the second email shortly
==========================
Email number 2 from Anmol, Dec 21

In this email I will describe how the dynamics of taxes/debt depend on absence of transfers relative to non-negative transfers as in AMSS. So we look at the case with risk free bond.

Consider a case where we have debt limits and the upper bound on goverment assets is strictly larger than the first best (associated with zero taxes and highest level of expenditure)

In both cases the Ramsey plan is characterized by a martingale equation for the multiplier on the implementability constraint.

1] In AMSS the presence of non-negativity constraint imposes an upperbound of zero on the value of the multiplier; inturn allowing us to make a sharp characterization that  all paths of the multiplier (and hence tax rates) will converge to zero.

2] In our case with absence of transfers there is no absorbing point for the multiplier (and consequently debt and tax rates) so we have an ergodic distribution with a bounded support that is pinned down by the asset limits.Further we show that any small interval containing either the upper bound or the lower bound on assets will be will be visited recurrently.

A few remarks
 1] As the upper bound on government assets goes to infinity, all asset sequences in our case will diverge to + infinity and tax rates to - infinity.

2] Adding payoffs from the set P^* will imply that the limiting debt is degenerate in our case with tax rates that can be either positive and negative in the limit. Now adding these payoffs in AMSS is interesting - The ergodic distribution will be either of the  two points the first best or the steady state associated with P^*; which one depends on initial conditions.

Let me know if this is clear 

%%%%%%%%%%%%%%%%%%%%   Anmol's Dec 23 e-mail %%%%%%%%%%%%%%%%%%%%%%%%%%%%%



Thanks for this. I will write down the equations and results from the settings a) with and without non negativity constraints for the b) two agents and the single agent case

So we have everything about the 4 cases in a single place.

I am free to talk about the details whenever you have time.

And for the answer to the question about debt dynamics: For quasilinear preferences, assets of all agents for all dates are equal to their initial values with either complete markets or unrestricted transfers.

As a side remark:
1] Complete Markets: With risk aversion and complete markets we will have marginal utility adjusted assets being constant and so assets will be an exact function of the shock from date 1. Between date 0 and date 1 the time inconsistency problem will induce a mapping from initial assets to the "stationary" asset distribution that the optimal plan adheres to. Ofcourse the mean of this distrivution will depend on the initial conditions.

2] Unrestricted Transfers: For a representative agent, I think having unrestricted transfers will make the asset holdings to be equal to their initial level for all dates irrespective of either curvature or market structure. For multi agents and risk aversion having unrestricted transfers we come back to our problem (or werning depeding on the market structure). In our world dynamics of asset holdings are non trivial and depend on the two features that we talked about " how far are we away from complete markets" and "how costly is it use transfers". The general findings being that magnitude of asset depends on the strength of a) correlation between returns and net of interest deficit and b) Pareto weights that proxy for costs of transfers

The QL with c_2>=0 example gives us predictions for how these two things work. It induces costs of transfers that depend on omega which allow us to separate this two things.

We verify the implications numerically, first by looking at special cases with risk aversion where we know the ergodic distribution is "easy" (this is looking at the binary iid shocks case) and then staring simulations from very general models.
