%2multibyte Version: 5.50.0.2953 CodePage: 1251
%%              Scientific Word   Wrap/Unwrap  Version 2.5             %
%%              Scientific Word   Wrap/Unwrap  Version 3.0             %
%% If you are separating the files in this message by hand, you will   %
%% need to identify the file type and place it in the appropriate      %
%% directory.  The possible types are: Document, DocAssoc, Other,      %
%% Macro, Style, Graphic, PastedPict, and PlotPict. Extract files      %
%% tagged as Document, DocAssoc, or Other into your TeX source file    %
%% directory.  Macro files go into your TeX macros directory. Style    %
%% files are used by Scientific Word and do not need to be extracted.  %
%% Graphic, PastedPict, and PlotPict files should be placed in a       %
%% graphics directory.                                                 %
%% Graphic files need to be converted from the text format (this is    %
%% done for e-mail compatability) to the original 8-bit binary format. %
%% Files included:                                                     %
%% "/document/NonbalancedJune18.tex", Document, 124302, 6/18/2004, 0:45:10, ""%
%% "/document/HZH9JA00.wmf", PastePict, 30602, 6/17/2004, 21:53:40, "" %
%% "/document/HZH9JA01.wmf", PastePict, 31310, 6/17/2004, 21:55:05, "" %
%% "/document/HZH9JA02.wmf", PastePict, 31498, 6/17/2004, 21:56:09, "" %
%% "/document/HXIWF402.wmf", ImportPict, 31112, 6/11/2004, 2:52:58, "" %
%% "/document/figure1new.wmf", ImportPict, 8034, 6/11/2004, 2:52:58, ""%
%% "/document/HZH9JA03.wmf", PastePict, 49212, 6/11/2004, 2:52:58, ""  %
%%%%%%%%%%%%%%%%% Start /document/NonbalancedJune18.tex %%%%%%%%%%%%%%%%
%%TCIDATA{OutputFilter=LATEX.DLL}
%%TCIDATA{Version=5.50.0.2890}
%%TCIDATA{Codepage=1251}
%%TCIDATA{<META NAME="SaveForMode" CONTENT="1">}
%%TCIDATA{BibliographyScheme=Manual}
%%TCIDATA{Created=Sat Jul 31 12:49:26 1999}
%%TCIDATA{LastRevised=Monday, February 20, 2012 12:05:46}
%%TCIDATA{<META NAME="GraphicsSave" CONTENT="32">}
%%TCIDATA{Language=American English}
%%TCIDATA{CSTFile=LaTeX article (bright).cst}
%=====================================================  tom front end %==================================================
%\documentclass[12pt]{article}
%\usepackage{amsmath,amssymb,amsthm,enumerate,graphicx}
%\usepackage{ifthen,latexsym,syntonly}
%\usepackage{setspace}
%\usepackage[showrefs]{refcheck}  %use this to show equation and section labels
%\usepackage{color}
%\usepackage[round,comma,authoryear]{natbib}   % for natbib
%\usepackage{subfigure}
%\usepackage{float}
%\bibliographystyle{mynat}
%\onehalfspacing
%   % for natbib
%%\bibpunct{(}{)}{,}{a}{}{,}  % for natbib
%%                            % need to have mynat.bst in an accesssible directory
%\bibpunct{(}{)}{,}{a}{}{,}  % for natbib
%                            % need to have mynat.bst in an accesssible directory
%\newtheorem{theorem}{Theorem}[section]
%\newtheorem{remark}[theorem]{Remark}
%\newtheorem{assumption}[theorem]{Assumption}
%\newtheorem{case}[theorem]{Case}
%\newtheorem{claim}[theorem]{Claim}
%%\newtheorem{conclusion}[theorem]{Conclusion}
%\newtheorem{corollary}[theorem]{Corollary}
%\newtheorem{condition}[theorem]{Condition}
%\newtheorem{criterion}[theorem]{Criterion}
%\newtheorem{definition}[theorem]{Definition}
%\newtheorem{example}[theorem]{Example}
%\newtheorem{lemma}[theorem]{Lemma}
%\newtheorem{problem}[theorem]{Problem}
%\newtheorem{proposition}[theorem]{Proposition}
%%\newtheorem{solution}[theorem]{Solution}
%%\newtheorem{summary}[theorem]{Summary}
%\newtheorem{thm}[theorem]{Theorem}
%\setlength{\oddsidemargin}{.05in} \setlength{\topmargin}{-.45in}
%\setlength{\textwidth}{6.4in} \setlength{\textheight}{8.5in}
%%\pagestyle{empty}
%%=================================================== end of tom front end %=================================================
%\usepackage[showrefs]{refcheck}
%\usepackage{showlabels}
%% Macros for Scientific Word 3.5 documents saved with the LaTeX filter.
% Copyright (C) 2000 Mackichan Software, Inc.

\typeout{TCILATEX Macros for Scientific Word 3.5 <19 July 2000>.}
\typeout{NOTICE:  This macro file is NOT proprietary and may be 
freely copied and distributed.}
%
\makeatletter

%%%%%%%%%%%%%%%%%%%%%
% FMTeXButton
% This is used for putting TeXButtons in the 
% frontmatter of a document. Add a line like
% \QTagDef{FMTeXButton}{101}{} to the filter 
% section of the cst being used. Also add a
% new section containing:
%     [f_101]
%     ALIAS=FMTexButton
%     TAG_TYPE=FIELD
%     TAG_LEADIN=TeX Button:
%
% It also works to put \defs in the preamble after 
% the \input tcilatex
\def\FMTeXButton#1{#1}
%
%%%%%%%%%%%%%%%%%%%%%%
% macros for time
\newcount\@hour\newcount\@minute\chardef\@x10\chardef\@xv60
\def\tcitime{
\def\@time{%
  \@minute\time\@hour\@minute\divide\@hour\@xv
  \ifnum\@hour<\@x 0\fi\the\@hour:%
  \multiply\@hour\@xv\advance\@minute-\@hour
  \ifnum\@minute<\@x 0\fi\the\@minute
  }}%

%%%%%%%%%%%%%%%%%%%%%%
% macro for hyperref
%%% \@ifundefined{hyperref}{\def\hyperref#1#2#3#4{#2\ref{#4}#3}}{}

\def\x@hyperref#1#2#3{%
   % Trun off various catcodes before reading parameter 4
   \catcode`\~ = 12
   \catcode`\$ = 12
   \catcode`\_ = 12
   \catcode`\# = 12
   \catcode`\& = 12
   \y@hyperref{#1}{#2}{#3}%
}

\def\y@hyperref#1#2#3#4{%
   #2\ref{#4}#3
   \catcode`\~ = 13
   \catcode`\$ = 3
   \catcode`\_ = 8
   \catcode`\# = 6
   \catcode`\& = 4
}

\@ifundefined{hyperref}{\let\hyperref\x@hyperref}{}


% macro for external program call
\@ifundefined{qExtProgCall}{\def\qExtProgCall#1#2#3#4#5#6{\relax}}{}
%%%%%%%%%%%%%%%%%%%%%%
%
% macros for graphics
%
\def\FILENAME#1{#1}%
%
\def\QCTOpt[#1]#2{%
  \def\QCTOptB{#1}
  \def\QCTOptA{#2}
}
\def\QCTNOpt#1{%
  \def\QCTOptA{#1}
  \let\QCTOptB\empty
}
\def\Qct{%
  \@ifnextchar[{%
    \QCTOpt}{\QCTNOpt}
}
\def\QCBOpt[#1]#2{%
  \def\QCBOptB{#1}%
  \def\QCBOptA{#2}%
}
\def\QCBNOpt#1{%
  \def\QCBOptA{#1}%
  \let\QCBOptB\empty
}
\def\Qcb{%
  \@ifnextchar[{%
    \QCBOpt}{\QCBNOpt}%
}
\def\PrepCapArgs{%
  \ifx\QCBOptA\empty
    \ifx\QCTOptA\empty
      {}%
    \else
      \ifx\QCTOptB\empty
        {\QCTOptA}%
      \else
        [\QCTOptB]{\QCTOptA}%
      \fi
    \fi
  \else
    \ifx\QCBOptA\empty
      {}%
    \else
      \ifx\QCBOptB\empty
        {\QCBOptA}%
      \else
        [\QCBOptB]{\QCBOptA}%
      \fi
    \fi
  \fi
}
\newcount\GRAPHICSTYPE
%\GRAPHICSTYPE 0 is for TurboTeX
%\GRAPHICSTYPE 1 is for DVIWindo (PostScript)
%%%(removed)%\GRAPHICSTYPE 2 is for psfig (PostScript)
\GRAPHICSTYPE=\z@
\def\GRAPHICSPS#1{%
 \ifcase\GRAPHICSTYPE%\GRAPHICSTYPE=0
   \special{ps: #1}%
 \or%\GRAPHICSTYPE=1
   \special{language "PS", include "#1"}%
%%%\or%\GRAPHICSTYPE=2
%%%  #1%
 \fi
}%
%
\def\GRAPHICSHP#1{\special{include #1}}%
%
% \graffile{ body }                                  %#1
%          { contentswidth (scalar)  }               %#2
%          { contentsheight (scalar) }               %#3
%          { vertical shift when in-line (scalar) }  %#4

\def\graffile#1#2#3#4{%
%%% \ifnum\GRAPHICSTYPE=\tw@
%%%  %Following if using psfig
%%%  \@ifundefined{psfig}{\input psfig.tex}{}%
%%%  \psfig{file=#1, height=#3, width=#2}%
%%% \else
  %Following for all others
  % JCS - added BOXTHEFRAME, see below
    \bgroup
	   \@inlabelfalse
       \leavevmode
       \@ifundefined{bbl@deactivate}{\def~{\string~}}{\activesoff}%
        \raise -#4 \BOXTHEFRAME{%
           \hbox to #2{\raise #3\hbox to #2{\null #1\hfil}}}%
    \egroup
}%
%
% A box for drafts
\def\draftbox#1#2#3#4{%
 \leavevmode\raise -#4 \hbox{%
  \frame{\rlap{\protect\tiny #1}\hbox to #2%
   {\vrule height#3 width\z@ depth\z@\hfil}%
  }%
 }%
}%
%
\newcount\draft
\draft=\z@
\let\nographics=\draft
\newif\ifwasdraft
\wasdraftfalse

%  \GRAPHIC{ body }                                  %#1
%          { draft name }                            %#2
%          { contentswidth (scalar)  }               %#3
%          { contentsheight (scalar) }               %#4
%          { vertical shift when in-line (scalar) }  %#5
\def\GRAPHIC#1#2#3#4#5{%
   \ifnum\draft=\@ne\draftbox{#2}{#3}{#4}{#5}%
   \else\graffile{#1}{#3}{#4}{#5}%
   \fi
}
%
\def\addtoLaTeXparams#1{%
    \edef\LaTeXparams{\LaTeXparams #1}}%
%
% JCS -  added a switch BoxFrame that can 
% be set by including X in the frame params.
% If set a box is drawn around the frame.

\newif\ifBoxFrame \BoxFramefalse
\newif\ifOverFrame \OverFramefalse
\newif\ifUnderFrame \UnderFramefalse

\def\BOXTHEFRAME#1{%
   \hbox{%
      \ifBoxFrame
         \frame{#1}%
      \else
         {#1}%
      \fi
   }%
}


\def\doFRAMEparams#1{\BoxFramefalse\OverFramefalse\UnderFramefalse\readFRAMEparams#1\end}%
\def\readFRAMEparams#1{%
 \ifx#1\end%
  \let\next=\relax
  \else
  \ifx#1i\dispkind=\z@\fi
  \ifx#1d\dispkind=\@ne\fi
  \ifx#1f\dispkind=\tw@\fi
  \ifx#1t\addtoLaTeXparams{t}\fi
  \ifx#1b\addtoLaTeXparams{b}\fi
  \ifx#1p\addtoLaTeXparams{p}\fi
  \ifx#1h\addtoLaTeXparams{h}\fi
  \ifx#1X\BoxFrametrue\fi
  \ifx#1O\OverFrametrue\fi
  \ifx#1U\UnderFrametrue\fi
  \ifx#1w
    \ifnum\draft=1\wasdrafttrue\else\wasdraftfalse\fi
    \draft=\@ne
  \fi
  \let\next=\readFRAMEparams
  \fi
 \next
 }%
%
%Macro for In-line graphics object
%   \IFRAME{ contentswidth (scalar)  }               %#1
%          { contentsheight (scalar) }               %#2
%          { vertical shift when in-line (scalar) }  %#3
%          { draft name }                            %#4
%          { body }                                  %#5
%          { caption}                                %#6


\def\IFRAME#1#2#3#4#5#6{%
      \bgroup
      \let\QCTOptA\empty
      \let\QCTOptB\empty
      \let\QCBOptA\empty
      \let\QCBOptB\empty
      #6%
      \parindent=0pt
      \leftskip=0pt
      \rightskip=0pt
      \setbox0=\hbox{\QCBOptA}%
      \@tempdima=#1\relax
      \ifOverFrame
          % Do this later
          \typeout{This is not implemented yet}%
          \show\HELP
      \else
         \ifdim\wd0>\@tempdima
            \advance\@tempdima by \@tempdima
            \ifdim\wd0 >\@tempdima
               \setbox1 =\vbox{%
                  \unskip\hbox to \@tempdima{\hfill\GRAPHIC{#5}{#4}{#1}{#2}{#3}\hfill}%
                  \unskip\hbox to \@tempdima{\parbox[b]{\@tempdima}{\QCBOptA}}%
               }%
               \wd1=\@tempdima
            \else
               \textwidth=\wd0
               \setbox1 =\vbox{%
                 \noindent\hbox to \wd0{\hfill\GRAPHIC{#5}{#4}{#1}{#2}{#3}\hfill}\\%
                 \noindent\hbox{\QCBOptA}%
               }%
               \wd1=\wd0
            \fi
         \else
            \ifdim\wd0>0pt
              \hsize=\@tempdima
              \setbox1=\vbox{%
                \unskip\GRAPHIC{#5}{#4}{#1}{#2}{0pt}%
                \break
                \unskip\hbox to \@tempdima{\hfill \QCBOptA\hfill}%
              }%
              \wd1=\@tempdima
           \else
              \hsize=\@tempdima
              \setbox1=\vbox{%
                \unskip\GRAPHIC{#5}{#4}{#1}{#2}{0pt}%
              }%
              \wd1=\@tempdima
           \fi
         \fi
         \@tempdimb=\ht1
         %\advance\@tempdimb by \dp1
         \advance\@tempdimb by -#2
         \advance\@tempdimb by #3
         \leavevmode
         \raise -\@tempdimb \hbox{\box1}%
      \fi
      \egroup%
}%
%
%Macro for Display graphics object
%   \DFRAME{ contentswidth (scalar)  }               %#1
%          { contentsheight (scalar) }               %#2
%          { draft label }                           %#3
%          { name }                                  %#4
%          { caption}                                %#5
\def\DFRAME#1#2#3#4#5{%
 \begin{center}
     \let\QCTOptA\empty
     \let\QCTOptB\empty
     \let\QCBOptA\empty
     \let\QCBOptB\empty
	 \vbox\bgroup
        \ifOverFrame 
           #5\QCTOptA\par
        \fi
        \GRAPHIC{#4}{#3}{#1}{#2}{\z@}
        \ifUnderFrame 
           \par#5\QCBOptA
        \fi
	 \egroup
 \end{center}%
 }%
%
%Macro for Floating graphic object
%   \FFRAME{ framedata f|i tbph x F|T }              %#1
%          { contentswidth (scalar)  }               %#2
%          { contentsheight (scalar) }               %#3
%          { caption }                               %#4
%          { label }                                 %#5
%          { draft name }                            %#6
%          { body }                                  %#7
\def\FFRAME#1#2#3#4#5#6#7{%
 %If float.sty loaded and float option is 'h', change to 'H'  (gp) 1998/09/05
  \@ifundefined{floatstyle}
    {%floatstyle undefined (and float.sty not present), no change
     \begin{figure}[#1]%
    }
    {%floatstyle DEFINED
	 \ifx#1h%Only the h parameter, change to H
      \begin{figure}[H]%
	 \else
      \begin{figure}[#1]%
	 \fi
	}
  \let\QCTOptA\empty
  \let\QCTOptB\empty
  \let\QCBOptA\empty
  \let\QCBOptB\empty
  \ifOverFrame
    #4
    \ifx\QCTOptA\empty
    \else
      \ifx\QCTOptB\empty
        \caption{\QCTOptA}%
      \else
        \caption[\QCTOptB]{\QCTOptA}%
      \fi
    \fi
    \ifUnderFrame\else
      \label{#5}%
    \fi
  \else
    \UnderFrametrue%
  \fi
  \begin{center}\GRAPHIC{#7}{#6}{#2}{#3}{\z@}\end{center}%
  \ifUnderFrame
    #4
    \ifx\QCBOptA\empty
      \caption{}%
    \else
      \ifx\QCBOptB\empty
        \caption{\QCBOptA}%
      \else
        \caption[\QCBOptB]{\QCBOptA}%
      \fi
    \fi
    \label{#5}%
  \fi
  \end{figure}%
 }%
%
%
%    \FRAME{ framedata f|i tbph x F|T }              %#1
%          { contentswidth (scalar)  }               %#2
%          { contentsheight (scalar) }               %#3
%          { vertical shift when in-line (scalar) }  %#4
%          { caption }                               %#5
%          { label }                                 %#6
%          { name }                                  %#7
%          { body }                                  %#8
%
%    framedata is a string which can contain the following
%    characters: idftbphxFT
%    Their meaning is as follows:
%             i, d or f : in-line, display, or floating
%             t,b,p,h   : LaTeX floating placement options
%             x         : fit contents box to contents
%             F or T    : Figure or Table. 
%                         Later this can expand
%                         to a more general float class.
%
%
\newcount\dispkind%

\def\makeactives{
  \catcode`\"=\active
  \catcode`\;=\active
  \catcode`\:=\active
  \catcode`\'=\active
  \catcode`\~=\active
}
\bgroup
   \makeactives
   \gdef\activesoff{%
      \def"{\string"}
      \def;{\string;}
      \def:{\string:}
      \def'{\string'}
      \def~{\string~}
      %\bbl@deactivate{"}%
      %\bbl@deactivate{;}%
      %\bbl@deactivate{:}%
      %\bbl@deactivate{'}%
    }
\egroup

\def\FRAME#1#2#3#4#5#6#7#8{%
 \bgroup
 \ifnum\draft=\@ne
   \wasdrafttrue
 \else
   \wasdraftfalse%
 \fi
 \def\LaTeXparams{}%
 \dispkind=\z@
 \def\LaTeXparams{}%
 \doFRAMEparams{#1}%
 \ifnum\dispkind=\z@\IFRAME{#2}{#3}{#4}{#7}{#8}{#5}\else
  \ifnum\dispkind=\@ne\DFRAME{#2}{#3}{#7}{#8}{#5}\else
   \ifnum\dispkind=\tw@
    \edef\@tempa{\noexpand\FFRAME{\LaTeXparams}}%
    \@tempa{#2}{#3}{#5}{#6}{#7}{#8}%
    \fi
   \fi
  \fi
  \ifwasdraft\draft=1\else\draft=0\fi{}%
  \egroup
 }%
%
% This macro added to let SW gobble a parameter that
% should not be passed on and expanded. 

\def\TEXUX#1{"texux"}

%
% Macros for text attributes:
%
\def\BF#1{{\bf {#1}}}%
\def\NEG#1{\leavevmode\hbox{\rlap{\thinspace/}{$#1$}}}%
%
%%%%%%%%%%%%%%%%%%%%%%%%%%%%%%%%%%%%%%%%%%%%%%%%%%%%%%%%%%%%%%%%%%%%%%%%
%
%
% macros for user - defined functions
\def\limfunc#1{\mathop{\rm #1}}%
\def\func#1{\mathop{\rm #1}\nolimits}%
% macro for unit names
\def\unit#1{\mathop{\rm #1}\nolimits}%

%
% miscellaneous 
\long\def\QQQ#1#2{%
     \long\expandafter\def\csname#1\endcsname{#2}}%
\@ifundefined{QTP}{\def\QTP#1{}}{}
\@ifundefined{QEXCLUDE}{\def\QEXCLUDE#1{}}{}
\@ifundefined{Qlb}{\def\Qlb#1{#1}}{}
\@ifundefined{Qlt}{\def\Qlt#1{#1}}{}
\def\QWE{}%
\long\def\QQA#1#2{}%
\def\QTR#1#2{{\csname#1\endcsname #2}}%(gp) Is this the best?
\long\def\TeXButton#1#2{#2}%
\long\def\QSubDoc#1#2{#2}%
\def\EXPAND#1[#2]#3{}%
\def\NOEXPAND#1[#2]#3{}%
\def\PROTECTED{}%
\def\LaTeXparent#1{}%
\def\ChildStyles#1{}%
\def\ChildDefaults#1{}%
\def\QTagDef#1#2#3{}%

% Constructs added with Scientific Notebook
\@ifundefined{correctchoice}{\def\correctchoice{\relax}}{}
\@ifundefined{HTML}{\def\HTML#1{\relax}}{}
\@ifundefined{TCIIcon}{\def\TCIIcon#1#2#3#4{\relax}}{}
\if@compatibility
  \typeout{Not defining UNICODE  U or CustomNote commands for LaTeX 2.09.}
\else
  \providecommand{\UNICODE}[2][]{\protect\rule{.1in}{.1in}}
  \providecommand{\U}[1]{\protect\rule{.1in}{.1in}}
  \providecommand{\CustomNote}[3][]{\marginpar{#3}}
\fi

%
% Macros for style editor docs
\@ifundefined{StyleEditBeginDoc}{\def\StyleEditBeginDoc{\relax}}{}
%
% Macros for footnotes
\def\QQfnmark#1{\footnotemark}
\def\QQfntext#1#2{\addtocounter{footnote}{#1}\footnotetext{#2}}
%
% Macros for indexing.
%
\@ifundefined{TCIMAKEINDEX}{}{\makeindex}%
%
% Attempts to avoid problems with other styles
\@ifundefined{abstract}{%
 \def\abstract{%
  \if@twocolumn
   \section*{Abstract (Not appropriate in this style!)}%
   \else \small 
   \begin{center}{\bf Abstract\vspace{-.5em}\vspace{\z@}}\end{center}%
   \quotation 
   \fi
  }%
 }{%
 }%
\@ifundefined{endabstract}{\def\endabstract
  {\if@twocolumn\else\endquotation\fi}}{}%
\@ifundefined{maketitle}{\def\maketitle#1{}}{}%
\@ifundefined{affiliation}{\def\affiliation#1{}}{}%
\@ifundefined{proof}{\def\proof{\noindent{\bfseries Proof. }}}{}%
\@ifundefined{endproof}{\def\endproof{\mbox{\ \rule{.1in}{.1in}}}}{}%
\@ifundefined{newfield}{\def\newfield#1#2{}}{}%
\@ifundefined{chapter}{\def\chapter#1{\par(Chapter head:)#1\par }%
 \newcount\c@chapter}{}%
\@ifundefined{part}{\def\part#1{\par(Part head:)#1\par }}{}%
\@ifundefined{section}{\def\section#1{\par(Section head:)#1\par }}{}%
\@ifundefined{subsection}{\def\subsection#1%
 {\par(Subsection head:)#1\par }}{}%
\@ifundefined{subsubsection}{\def\subsubsection#1%
 {\par(Subsubsection head:)#1\par }}{}%
\@ifundefined{paragraph}{\def\paragraph#1%
 {\par(Subsubsubsection head:)#1\par }}{}%
\@ifundefined{subparagraph}{\def\subparagraph#1%
 {\par(Subsubsubsubsection head:)#1\par }}{}%
%%%%%%%%%%%%%%%%%%%%%%%%%%%%%%%%%%%%%%%%%%%%%%%%%%%%%%%%%%%%%%%%%%%%%%%%
% These symbols are not recognized by LaTeX
\@ifundefined{therefore}{\def\therefore{}}{}%
\@ifundefined{backepsilon}{\def\backepsilon{}}{}%
\@ifundefined{yen}{\def\yen{\hbox{\rm\rlap=Y}}}{}%
\@ifundefined{registered}{%
   \def\registered{\relax\ifmmode{}\r@gistered
                    \else$\m@th\r@gistered$\fi}%
 \def\r@gistered{^{\ooalign
  {\hfil\raise.07ex\hbox{$\scriptstyle\rm\text{R}$}\hfil\crcr
  \mathhexbox20D}}}}{}%
\@ifundefined{Eth}{\def\Eth{}}{}%
\@ifundefined{eth}{\def\eth{}}{}%
\@ifundefined{Thorn}{\def\Thorn{}}{}%
\@ifundefined{thorn}{\def\thorn{}}{}%
% A macro to allow any symbol that requires math to appear in text
\def\TEXTsymbol#1{\mbox{$#1$}}%
\@ifundefined{degree}{\def\degree{{}^{\circ}}}{}%
%
% macros for T3TeX files
\newdimen\theight
\@ifundefined{Column}{\def\Column{%
 \vadjust{\setbox\z@=\hbox{\scriptsize\quad\quad tcol}%
  \theight=\ht\z@\advance\theight by \dp\z@\advance\theight by \lineskip
  \kern -\theight \vbox to \theight{%
   \rightline{\rlap{\box\z@}}%
   \vss
   }%
  }%
 }}{}%
%
\@ifundefined{qed}{\def\qed{%
 \ifhmode\unskip\nobreak\fi\ifmmode\ifinner\else\hskip5\p@\fi\fi
 \hbox{\hskip5\p@\vrule width4\p@ height6\p@ depth1.5\p@\hskip\p@}%
 }}{}%
%
\@ifundefined{cents}{\def\cents{\hbox{\rm\rlap/c}}}{}%
\@ifundefined{miss}{\def\miss{\hbox{\vrule height2\p@ width 2\p@ depth\z@}}}{}%
%
\@ifundefined{vvert}{\def\vvert{\Vert}}{}%  %always translated to \left| or \right|
%
\@ifundefined{tcol}{\def\tcol#1{{\baselineskip=6\p@ \vcenter{#1}} \Column}}{}%
%
\@ifundefined{dB}{\def\dB{\hbox{{}}}}{}%        %dummy entry in column 
\@ifundefined{mB}{\def\mB#1{\hbox{$#1$}}}{}%   %column entry
\@ifundefined{nB}{\def\nB#1{\hbox{#1}}}{}%     %column entry (not math)
%
\@ifundefined{note}{\def\note{$^{\dag}}}{}%
%
\def\newfmtname{LaTeX2e}
% No longer load latexsym.  This is now handled by SWP, which uses amsfonts if necessary
%
\ifx\fmtname\newfmtname
  \DeclareOldFontCommand{\rm}{\normalfont\rmfamily}{\mathrm}
  \DeclareOldFontCommand{\sf}{\normalfont\sffamily}{\mathsf}
  \DeclareOldFontCommand{\tt}{\normalfont\ttfamily}{\mathtt}
  \DeclareOldFontCommand{\bf}{\normalfont\bfseries}{\mathbf}
  \DeclareOldFontCommand{\it}{\normalfont\itshape}{\mathit}
  \DeclareOldFontCommand{\sl}{\normalfont\slshape}{\@nomath\sl}
  \DeclareOldFontCommand{\sc}{\normalfont\scshape}{\@nomath\sc}
\fi

%
% Greek bold macros
% Redefine all of the math symbols 
% which might be bolded	 - there are 
% probably others to add to this list

\def\alpha{{\Greekmath 010B}}%
\def\beta{{\Greekmath 010C}}%
\def\gamma{{\Greekmath 010D}}%
\def\delta{{\Greekmath 010E}}%
\def\epsilon{{\Greekmath 010F}}%
\def\zeta{{\Greekmath 0110}}%
\def\eta{{\Greekmath 0111}}%
\def\theta{{\Greekmath 0112}}%
\def\iota{{\Greekmath 0113}}%
\def\kappa{{\Greekmath 0114}}%
\def\lambda{{\Greekmath 0115}}%
\def\mu{{\Greekmath 0116}}%
\def\nu{{\Greekmath 0117}}%
\def\xi{{\Greekmath 0118}}%
\def\pi{{\Greekmath 0119}}%
\def\rho{{\Greekmath 011A}}%
\def\sigma{{\Greekmath 011B}}%
\def\tau{{\Greekmath 011C}}%
\def\upsilon{{\Greekmath 011D}}%
\def\phi{{\Greekmath 011E}}%
\def\chi{{\Greekmath 011F}}%
\def\psi{{\Greekmath 0120}}%
\def\omega{{\Greekmath 0121}}%
\def\varepsilon{{\Greekmath 0122}}%
\def\vartheta{{\Greekmath 0123}}%
\def\varpi{{\Greekmath 0124}}%
\def\varrho{{\Greekmath 0125}}%
\def\varsigma{{\Greekmath 0126}}%
\def\varphi{{\Greekmath 0127}}%

\def\nabla{{\Greekmath 0272}}
\def\FindBoldGroup{%
   {\setbox0=\hbox{$\mathbf{x\global\edef\theboldgroup{\the\mathgroup}}$}}%
}

\def\Greekmath#1#2#3#4{%
    \if@compatibility
        \ifnum\mathgroup=\symbold
           \mathchoice{\mbox{\boldmath$\displaystyle\mathchar"#1#2#3#4$}}%
                      {\mbox{\boldmath$\textstyle\mathchar"#1#2#3#4$}}%
                      {\mbox{\boldmath$\scriptstyle\mathchar"#1#2#3#4$}}%
                      {\mbox{\boldmath$\scriptscriptstyle\mathchar"#1#2#3#4$}}%
        \else
           \mathchar"#1#2#3#4% 
        \fi 
    \else 
        \FindBoldGroup
        \ifnum\mathgroup=\theboldgroup % For 2e
           \mathchoice{\mbox{\boldmath$\displaystyle\mathchar"#1#2#3#4$}}%
                      {\mbox{\boldmath$\textstyle\mathchar"#1#2#3#4$}}%
                      {\mbox{\boldmath$\scriptstyle\mathchar"#1#2#3#4$}}%
                      {\mbox{\boldmath$\scriptscriptstyle\mathchar"#1#2#3#4$}}%
        \else
           \mathchar"#1#2#3#4% 
        \fi     	    
	  \fi}

\newif\ifGreekBold  \GreekBoldfalse
\let\SAVEPBF=\pbf
\def\pbf{\GreekBoldtrue\SAVEPBF}%
%

\@ifundefined{theorem}{\newtheorem{theorem}{Theorem}}{}
\@ifundefined{lemma}{\newtheorem{lemma}[theorem]{Lemma}}{}
\@ifundefined{corollary}{\newtheorem{corollary}[theorem]{Corollary}}{}
\@ifundefined{conjecture}{\newtheorem{conjecture}[theorem]{Conjecture}}{}
\@ifundefined{proposition}{\newtheorem{proposition}[theorem]{Proposition}}{}
\@ifundefined{axiom}{\newtheorem{axiom}{Axiom}}{}
\@ifundefined{remark}{\newtheorem{remark}{Remark}}{}
\@ifundefined{example}{\newtheorem{example}{Example}}{}
\@ifundefined{exercise}{\newtheorem{exercise}{Exercise}}{}
\@ifundefined{definition}{\newtheorem{definition}{Definition}}{}


\@ifundefined{mathletters}{%
  %\def\theequation{\arabic{equation}}
  \newcounter{equationnumber}  
  \def\mathletters{%
     \addtocounter{equation}{1}
     \edef\@currentlabel{\theequation}%
     \setcounter{equationnumber}{\c@equation}
     \setcounter{equation}{0}%
     \edef\theequation{\@currentlabel\noexpand\alph{equation}}%
  }
  \def\endmathletters{%
     \setcounter{equation}{\value{equationnumber}}%
  }
}{}

%Logos
\@ifundefined{BibTeX}{%
    \def\BibTeX{{\rm B\kern-.05em{\sc i\kern-.025em b}\kern-.08em
                 T\kern-.1667em\lower.7ex\hbox{E}\kern-.125emX}}}{}%
\@ifundefined{AmS}%
    {\def\AmS{{\protect\usefont{OMS}{cmsy}{m}{n}%
                A\kern-.1667em\lower.5ex\hbox{M}\kern-.125emS}}}{}%
\@ifundefined{AmSTeX}{\def\AmSTeX{\protect\AmS-\protect\TeX\@}}{}%
%

% This macro is a fix to eqnarray
\def\@@eqncr{\let\@tempa\relax
    \ifcase\@eqcnt \def\@tempa{& & &}\or \def\@tempa{& &}%
      \else \def\@tempa{&}\fi
     \@tempa
     \if@eqnsw
        \iftag@
           \@taggnum
        \else
           \@eqnnum\stepcounter{equation}%
        \fi
     \fi
     \global\tag@false
     \global\@eqnswtrue
     \global\@eqcnt\z@\cr}


\def\TCItag{\@ifnextchar*{\@TCItagstar}{\@TCItag}}
\def\@TCItag#1{%
    \global\tag@true
    \global\def\@taggnum{(#1)}}
\def\@TCItagstar*#1{%
    \global\tag@true
    \global\def\@taggnum{#1}}
%
%%%%%%%%%%%%%%%%%%%%%%%%%%%%%%%%%%%%%%%%%%%%%%%%%%%%%%%%%%%%%%%%%%%%%
%
\def\tfrac#1#2{{\textstyle {#1 \over #2}}}%
\def\dfrac#1#2{{\displaystyle {#1 \over #2}}}%
\def\binom#1#2{{#1 \choose #2}}%
\def\tbinom#1#2{{\textstyle {#1 \choose #2}}}%
\def\dbinom#1#2{{\displaystyle {#1 \choose #2}}}%
\def\QATOP#1#2{{#1 \atop #2}}%
\def\QTATOP#1#2{{\textstyle {#1 \atop #2}}}%
\def\QDATOP#1#2{{\displaystyle {#1 \atop #2}}}%
\def\QABOVE#1#2#3{{#2 \above#1 #3}}%
\def\QTABOVE#1#2#3{{\textstyle {#2 \above#1 #3}}}%
\def\QDABOVE#1#2#3{{\displaystyle {#2 \above#1 #3}}}%
\def\QOVERD#1#2#3#4{{#3 \overwithdelims#1#2 #4}}%
\def\QTOVERD#1#2#3#4{{\textstyle {#3 \overwithdelims#1#2 #4}}}%
\def\QDOVERD#1#2#3#4{{\displaystyle {#3 \overwithdelims#1#2 #4}}}%
\def\QATOPD#1#2#3#4{{#3 \atopwithdelims#1#2 #4}}%
\def\QTATOPD#1#2#3#4{{\textstyle {#3 \atopwithdelims#1#2 #4}}}%
\def\QDATOPD#1#2#3#4{{\displaystyle {#3 \atopwithdelims#1#2 #4}}}%
\def\QABOVED#1#2#3#4#5{{#4 \abovewithdelims#1#2#3 #5}}%
\def\QTABOVED#1#2#3#4#5{{\textstyle 
   {#4 \abovewithdelims#1#2#3 #5}}}%
\def\QDABOVED#1#2#3#4#5{{\displaystyle 
   {#4 \abovewithdelims#1#2#3 #5}}}%
%
% Macros for text size operators:
%
\def\tint{\mathop{\textstyle \int}}%
\def\tiint{\mathop{\textstyle \iint }}%
\def\tiiint{\mathop{\textstyle \iiint }}%
\def\tiiiint{\mathop{\textstyle \iiiint }}%
\def\tidotsint{\mathop{\textstyle \idotsint }}%
\def\toint{\mathop{\textstyle \oint}}%
\def\tsum{\mathop{\textstyle \sum }}%
\def\tprod{\mathop{\textstyle \prod }}%
\def\tbigcap{\mathop{\textstyle \bigcap }}%
\def\tbigwedge{\mathop{\textstyle \bigwedge }}%
\def\tbigoplus{\mathop{\textstyle \bigoplus }}%
\def\tbigodot{\mathop{\textstyle \bigodot }}%
\def\tbigsqcup{\mathop{\textstyle \bigsqcup }}%
\def\tcoprod{\mathop{\textstyle \coprod }}%
\def\tbigcup{\mathop{\textstyle \bigcup }}%
\def\tbigvee{\mathop{\textstyle \bigvee }}%
\def\tbigotimes{\mathop{\textstyle \bigotimes }}%
\def\tbiguplus{\mathop{\textstyle \biguplus }}%
%
%
%Macros for display size operators:
%
\def\dint{\displaystyle \int}%
\def\diint{\displaystyle \iint}%
\def\diiint{\displaystyle \iiint}%
\def\diiiint{\mathop{\displaystyle \iiiint }}%
\def\didotsint{\mathop{\displaystyle \idotsint }}%
\def\doint{\mathop{\displaystyle \oint}}%
\def\dsum{\mathop{\displaystyle \sum }}%
\def\dprod{\mathop{\displaystyle \prod }}%
\def\dbigcap{\mathop{\displaystyle \bigcap }}%
\def\dbigwedge{\mathop{\displaystyle \bigwedge }}%
\def\dbigoplus{\mathop{\displaystyle \bigoplus }}%
\def\dbigodot{\mathop{\displaystyle \bigodot }}%
\def\dbigsqcup{\mathop{\displaystyle \bigsqcup }}%
\def\dcoprod{\mathop{\displaystyle \coprod }}%
\def\dbigcup{\mathop{\displaystyle \bigcup }}%
\def\dbigvee{\mathop{\displaystyle \bigvee }}%
\def\dbigotimes{\mathop{\displaystyle \bigotimes }}%
\def\dbiguplus{\mathop{\displaystyle \biguplus }}%

%%%%%%%%%%%%%%%%%%%%%%%%%%%%%%%%%%%%%%%%%%%%%%%%%%%%%%%%%%%%%%%%%%%%%%%
% NOTE: The rest of this file is read only if amstex has not been
% loaded.  This section is used to define amstex constructs in the
% event they have not been defined.
%
%


\def\ExitTCILatex{\makeatother\endinput}

\bgroup
\ifx\ds@amstex\relax
   \message{amstex already loaded}\aftergroup\ExitTCILatex
\else
   \@ifpackageloaded{amsmath}%
      {\message{amsmath already loaded}\aftergroup\ExitTCILatex}
      {}
   \@ifpackageloaded{amstex}%
      {\message{amstex already loaded}\aftergroup\ExitTCILatex}
      {}
   \@ifpackageloaded{amsgen}%
      {\message{amsgen already loaded}\aftergroup\ExitTCILatex}
      {}
\fi
\egroup


%%%%%%%%%%%%%%%%%%%%%%%%%%%%%%%%%%%%%%%%%%%%%%%%%%%%%%%%%%%%%%%%%%%%%%%%
%%
%
%
%  Macros to define some AMS LaTeX constructs when 
%  AMS LaTeX has not been loaded
% 
% These macros are copied from the AMS-TeX package for doing
% multiple integrals.
%
\typeout{TCILATEX defining AMS-like constructs}
\let\DOTSI\relax
\def\RIfM@{\relax\ifmmode}%
\def\FN@{\futurelet\next}%
\newcount\intno@
\def\iint{\DOTSI\intno@\tw@\FN@\ints@}%
\def\iiint{\DOTSI\intno@\thr@@\FN@\ints@}%
\def\iiiint{\DOTSI\intno@4 \FN@\ints@}%
\def\idotsint{\DOTSI\intno@\z@\FN@\ints@}%
\def\ints@{\findlimits@\ints@@}%
\newif\iflimtoken@
\newif\iflimits@
\def\findlimits@{\limtoken@true\ifx\next\limits\limits@true
 \else\ifx\next\nolimits\limits@false\else
 \limtoken@false\ifx\ilimits@\nolimits\limits@false\else
 \ifinner\limits@false\else\limits@true\fi\fi\fi\fi}%
\def\multint@{\int\ifnum\intno@=\z@\intdots@                          %1
 \else\intkern@\fi                                                    %2
 \ifnum\intno@>\tw@\int\intkern@\fi                                   %3
 \ifnum\intno@>\thr@@\int\intkern@\fi                                 %4
 \int}%                                                               %5
\def\multintlimits@{\intop\ifnum\intno@=\z@\intdots@\else\intkern@\fi
 \ifnum\intno@>\tw@\intop\intkern@\fi
 \ifnum\intno@>\thr@@\intop\intkern@\fi\intop}%
\def\intic@{%
    \mathchoice{\hskip.5em}{\hskip.4em}{\hskip.4em}{\hskip.4em}}%
\def\negintic@{\mathchoice
 {\hskip-.5em}{\hskip-.4em}{\hskip-.4em}{\hskip-.4em}}%
\def\ints@@{\iflimtoken@                                              %1
 \def\ints@@@{\iflimits@\negintic@
   \mathop{\intic@\multintlimits@}\limits                             %2
  \else\multint@\nolimits\fi                                          %3
  \eat@}%                                                             %4
 \else                                                                %5
 \def\ints@@@{\iflimits@\negintic@
  \mathop{\intic@\multintlimits@}\limits\else
  \multint@\nolimits\fi}\fi\ints@@@}%
\def\intkern@{\mathchoice{\!\!\!}{\!\!}{\!\!}{\!\!}}%
\def\plaincdots@{\mathinner{\cdotp\cdotp\cdotp}}%
\def\intdots@{\mathchoice{\plaincdots@}%
 {{\cdotp}\mkern1.5mu{\cdotp}\mkern1.5mu{\cdotp}}%
 {{\cdotp}\mkern1mu{\cdotp}\mkern1mu{\cdotp}}%
 {{\cdotp}\mkern1mu{\cdotp}\mkern1mu{\cdotp}}}%
%
%
%  These macros are for doing the AMS \text{} construct
%
\def\RIfM@{\relax\protect\ifmmode}
\def\text{\RIfM@\expandafter\text@\else\expandafter\mbox\fi}
\let\nfss@text\text
\def\text@#1{\mathchoice
   {\textdef@\displaystyle\f@size{#1}}%
   {\textdef@\textstyle\tf@size{\firstchoice@false #1}}%
   {\textdef@\textstyle\sf@size{\firstchoice@false #1}}%
   {\textdef@\textstyle \ssf@size{\firstchoice@false #1}}%
   \glb@settings}

\def\textdef@#1#2#3{\hbox{{%
                    \everymath{#1}%
                    \let\f@size#2\selectfont
                    #3}}}
\newif\iffirstchoice@
\firstchoice@true
%
%These are the AMS constructs for multiline limits.
%
\def\Let@{\relax\iffalse{\fi\let\\=\cr\iffalse}\fi}%
\def\vspace@{\def\vspace##1{\crcr\noalign{\vskip##1\relax}}}%
\def\multilimits@{\bgroup\vspace@\Let@
 \baselineskip\fontdimen10 \scriptfont\tw@
 \advance\baselineskip\fontdimen12 \scriptfont\tw@
 \lineskip\thr@@\fontdimen8 \scriptfont\thr@@
 \lineskiplimit\lineskip
 \vbox\bgroup\ialign\bgroup\hfil$\m@th\scriptstyle{##}$\hfil\crcr}%
\def\Sb{_\multilimits@}%
\def\endSb{\crcr\egroup\egroup\egroup}%
\def\Sp{^\multilimits@}%
\let\endSp\endSb
%
%
%These are AMS constructs for horizontal arrows
%
\newdimen\ex@
\ex@.2326ex
\def\rightarrowfill@#1{$#1\m@th\mathord-\mkern-6mu\cleaders
 \hbox{$#1\mkern-2mu\mathord-\mkern-2mu$}\hfill
 \mkern-6mu\mathord\rightarrow$}%
\def\leftarrowfill@#1{$#1\m@th\mathord\leftarrow\mkern-6mu\cleaders
 \hbox{$#1\mkern-2mu\mathord-\mkern-2mu$}\hfill\mkern-6mu\mathord-$}%
\def\leftrightarrowfill@#1{$#1\m@th\mathord\leftarrow
\mkern-6mu\cleaders
 \hbox{$#1\mkern-2mu\mathord-\mkern-2mu$}\hfill
 \mkern-6mu\mathord\rightarrow$}%
\def\overrightarrow{\mathpalette\overrightarrow@}%
\def\overrightarrow@#1#2{\vbox{\ialign{##\crcr\rightarrowfill@#1\crcr
 \noalign{\kern-\ex@\nointerlineskip}$\m@th\hfil#1#2\hfil$\crcr}}}%
\let\overarrow\overrightarrow
\def\overleftarrow{\mathpalette\overleftarrow@}%
\def\overleftarrow@#1#2{\vbox{\ialign{##\crcr\leftarrowfill@#1\crcr
 \noalign{\kern-\ex@\nointerlineskip}$\m@th\hfil#1#2\hfil$\crcr}}}%
\def\overleftrightarrow{\mathpalette\overleftrightarrow@}%
\def\overleftrightarrow@#1#2{\vbox{\ialign{##\crcr
   \leftrightarrowfill@#1\crcr
 \noalign{\kern-\ex@\nointerlineskip}$\m@th\hfil#1#2\hfil$\crcr}}}%
\def\underrightarrow{\mathpalette\underrightarrow@}%
\def\underrightarrow@#1#2{\vtop{\ialign{##\crcr$\m@th\hfil#1#2\hfil
  $\crcr\noalign{\nointerlineskip}\rightarrowfill@#1\crcr}}}%
\let\underarrow\underrightarrow
\def\underleftarrow{\mathpalette\underleftarrow@}%
\def\underleftarrow@#1#2{\vtop{\ialign{##\crcr$\m@th\hfil#1#2\hfil
  $\crcr\noalign{\nointerlineskip}\leftarrowfill@#1\crcr}}}%
\def\underleftrightarrow{\mathpalette\underleftrightarrow@}%
\def\underleftrightarrow@#1#2{\vtop{\ialign{##\crcr$\m@th
  \hfil#1#2\hfil$\crcr
 \noalign{\nointerlineskip}\leftrightarrowfill@#1\crcr}}}%
%%%%%%%%%%%%%%%%%%%%%

\def\qopnamewl@#1{\mathop{\operator@font#1}\nlimits@}
\let\nlimits@\displaylimits
\def\setboxz@h{\setbox\z@\hbox}


\def\varlim@#1#2{\mathop{\vtop{\ialign{##\crcr
 \hfil$#1\m@th\operator@font lim$\hfil\crcr
 \noalign{\nointerlineskip}#2#1\crcr
 \noalign{\nointerlineskip\kern-\ex@}\crcr}}}}

 \def\rightarrowfill@#1{\m@th\setboxz@h{$#1-$}\ht\z@\z@
  $#1\copy\z@\mkern-6mu\cleaders
  \hbox{$#1\mkern-2mu\box\z@\mkern-2mu$}\hfill
  \mkern-6mu\mathord\rightarrow$}
\def\leftarrowfill@#1{\m@th\setboxz@h{$#1-$}\ht\z@\z@
  $#1\mathord\leftarrow\mkern-6mu\cleaders
  \hbox{$#1\mkern-2mu\copy\z@\mkern-2mu$}\hfill
  \mkern-6mu\box\z@$}


\def\projlim{\qopnamewl@{proj\,lim}}
\def\injlim{\qopnamewl@{inj\,lim}}
\def\varinjlim{\mathpalette\varlim@\rightarrowfill@}
\def\varprojlim{\mathpalette\varlim@\leftarrowfill@}
\def\varliminf{\mathpalette\varliminf@{}}
\def\varliminf@#1{\mathop{\underline{\vrule\@depth.2\ex@\@width\z@
   \hbox{$#1\m@th\operator@font lim$}}}}
\def\varlimsup{\mathpalette\varlimsup@{}}
\def\varlimsup@#1{\mathop{\overline
  {\hbox{$#1\m@th\operator@font lim$}}}}

%
%Companion to stackrel
\def\stackunder#1#2{\mathrel{\mathop{#2}\limits_{#1}}}%
%
%
% These are AMS environments that will be defined to
% be verbatims if amstex has not actually been 
% loaded
%
%
\begingroup \catcode `|=0 \catcode `[= 1
\catcode`]=2 \catcode `\{=12 \catcode `\}=12
\catcode`\\=12 
|gdef|@alignverbatim#1\end{align}[#1|end[align]]
|gdef|@salignverbatim#1\end{align*}[#1|end[align*]]

|gdef|@alignatverbatim#1\end{alignat}[#1|end[alignat]]
|gdef|@salignatverbatim#1\end{alignat*}[#1|end[alignat*]]

|gdef|@xalignatverbatim#1\end{xalignat}[#1|end[xalignat]]
|gdef|@sxalignatverbatim#1\end{xalignat*}[#1|end[xalignat*]]

|gdef|@gatherverbatim#1\end{gather}[#1|end[gather]]
|gdef|@sgatherverbatim#1\end{gather*}[#1|end[gather*]]

|gdef|@gatherverbatim#1\end{gather}[#1|end[gather]]
|gdef|@sgatherverbatim#1\end{gather*}[#1|end[gather*]]


|gdef|@multilineverbatim#1\end{multiline}[#1|end[multiline]]
|gdef|@smultilineverbatim#1\end{multiline*}[#1|end[multiline*]]

|gdef|@arraxverbatim#1\end{arrax}[#1|end[arrax]]
|gdef|@sarraxverbatim#1\end{arrax*}[#1|end[arrax*]]

|gdef|@tabulaxverbatim#1\end{tabulax}[#1|end[tabulax]]
|gdef|@stabulaxverbatim#1\end{tabulax*}[#1|end[tabulax*]]


|endgroup
  

  
\def\align{\@verbatim \frenchspacing\@vobeyspaces \@alignverbatim
You are using the "align" environment in a style in which it is not defined.}
\let\endalign=\endtrivlist
 
\@namedef{align*}{\@verbatim\@salignverbatim
You are using the "align*" environment in a style in which it is not defined.}
\expandafter\let\csname endalign*\endcsname =\endtrivlist




\def\alignat{\@verbatim \frenchspacing\@vobeyspaces \@alignatverbatim
You are using the "alignat" environment in a style in which it is not defined.}
\let\endalignat=\endtrivlist
 
\@namedef{alignat*}{\@verbatim\@salignatverbatim
You are using the "alignat*" environment in a style in which it is not defined.}
\expandafter\let\csname endalignat*\endcsname =\endtrivlist




\def\xalignat{\@verbatim \frenchspacing\@vobeyspaces \@xalignatverbatim
You are using the "xalignat" environment in a style in which it is not defined.}
\let\endxalignat=\endtrivlist
 
\@namedef{xalignat*}{\@verbatim\@sxalignatverbatim
You are using the "xalignat*" environment in a style in which it is not defined.}
\expandafter\let\csname endxalignat*\endcsname =\endtrivlist




\def\gather{\@verbatim \frenchspacing\@vobeyspaces \@gatherverbatim
You are using the "gather" environment in a style in which it is not defined.}
\let\endgather=\endtrivlist
 
\@namedef{gather*}{\@verbatim\@sgatherverbatim
You are using the "gather*" environment in a style in which it is not defined.}
\expandafter\let\csname endgather*\endcsname =\endtrivlist


\def\multiline{\@verbatim \frenchspacing\@vobeyspaces \@multilineverbatim
You are using the "multiline" environment in a style in which it is not defined.}
\let\endmultiline=\endtrivlist
 
\@namedef{multiline*}{\@verbatim\@smultilineverbatim
You are using the "multiline*" environment in a style in which it is not defined.}
\expandafter\let\csname endmultiline*\endcsname =\endtrivlist


\def\arrax{\@verbatim \frenchspacing\@vobeyspaces \@arraxverbatim
You are using a type of "array" construct that is only allowed in AmS-LaTeX.}
\let\endarrax=\endtrivlist

\def\tabulax{\@verbatim \frenchspacing\@vobeyspaces \@tabulaxverbatim
You are using a type of "tabular" construct that is only allowed in AmS-LaTeX.}
\let\endtabulax=\endtrivlist

 
\@namedef{arrax*}{\@verbatim\@sarraxverbatim
You are using a type of "array*" construct that is only allowed in AmS-LaTeX.}
\expandafter\let\csname endarrax*\endcsname =\endtrivlist

\@namedef{tabulax*}{\@verbatim\@stabulaxverbatim
You are using a type of "tabular*" construct that is only allowed in AmS-LaTeX.}
\expandafter\let\csname endtabulax*\endcsname =\endtrivlist

% macro to simulate ams tag construct


% This macro is a fix to the equation environment
 \def\endequation{%
     \ifmmode\ifinner % FLEQN hack
      \iftag@
        \addtocounter{equation}{-1} % undo the increment made in the begin part
        $\hfil
           \displaywidth\linewidth\@taggnum\egroup \endtrivlist
        \global\tag@false
        \global\@ignoretrue   
      \else
        $\hfil
           \displaywidth\linewidth\@eqnnum\egroup \endtrivlist
        \global\tag@false
        \global\@ignoretrue 
      \fi
     \else   
      \iftag@
        \addtocounter{equation}{-1} % undo the increment made in the begin part
        \eqno \hbox{\@taggnum}
        \global\tag@false%
        $$\global\@ignoretrue
      \else
        \eqno \hbox{\@eqnnum}% $$ BRACE MATCHING HACK
        $$\global\@ignoretrue
      \fi
     \fi\fi
 } 

 \newif\iftag@ \tag@false
 
 \def\TCItag{\@ifnextchar*{\@TCItagstar}{\@TCItag}}
 \def\@TCItag#1{%
     \global\tag@true
     \global\def\@taggnum{(#1)}}
 \def\@TCItagstar*#1{%
     \global\tag@true
     \global\def\@taggnum{#1}}

  \@ifundefined{tag}{
     \def\tag{\@ifnextchar*{\@tagstar}{\@tag}}
     \def\@tag#1{%
         \global\tag@true
         \global\def\@taggnum{(#1)}}
     \def\@tagstar*#1{%
         \global\tag@true
         \global\def\@taggnum{#1}}
  }{}
% Do not add anything to the end of this file.  
% The last section of the file is loaded only if 
% amstex has not been.



\makeatother
\endinput




\documentclass[thmsb,11pt]{article}
%%%%%%%%%%%%%%%%%%%%%%%%%%%%%%%%%%%%%%%%%%%%%%%%%%%%%%%%%%%%%%%%%%%%%%%%%%%%%%%%%%%%%%%%%%%%%%%%%%%%%%%%%%%%%%%%%%%%%%%%%%%%%%%%%%%%%%%%%%%%%%%%%%%%%%%%%%%%%%%%%%%%%%%%%%%%%%%%%%%%%%%%%%%%%%%%%%%%%%%%%%%%%%%%%%%%%%%%%%%%%%%%%%%%%%%%%%%%%%%%%%%%%%%%%%%%
\usepackage{amsfonts}
\usepackage{appendix}
\usepackage[pagewise,displaymath, mathlines]{lineno}
\usepackage{amssymb}
\usepackage{amsmath}
\usepackage{graphicx}
\usepackage{color}
\usepackage{refcount}
\usepackage{natbib}
\usepackage{bm}
\usepackage{hyperref}
\usepackage{epstopdf}
\setcounter{MaxMatrixCols}{10}
%TCIDATA{TCIstyle=article/art4.lat,lart,article}

%TCIDATA{OutputFilter=LATEX.DLL}
%TCIDATA{Version=5.50.0.2953}
%TCIDATA{Codepage=1251}
%TCIDATA{<META NAME="SaveForMode" CONTENT="1">}
%TCIDATA{BibliographyScheme=Manual}
%TCIDATA{Created=Sat Jul 31 12:49:26 1999}
%TCIDATA{LastRevised=Wednesday, June 12, 2013 12:37:50}
%TCIDATA{<META NAME="GraphicsSave" CONTENT="32">}
%TCIDATA{Language=American English}
%TCIDATA{CSTFile=LaTeX article (bright).cst}

\newtheorem{theorem}{Theorem}
\newtheorem{acknowledgement}[theorem]{Acknowledgement}
\newtheorem{algorithm}[theorem]{Algorithm}
\newtheorem{assumption}{Assumption}
\newtheorem{axiom}{Axiom}
\newtheorem{case}[theorem]{Case}
\newtheorem{claim}[theorem]{Claim}
\newtheorem{conclusion}[theorem]{Conclusion}
\newtheorem{condition}[theorem]{Condition}
\newtheorem{conjecture}{Conjecture}
\newtheorem{corollary}{Corollary}
\newtheorem{criterion}[theorem]{Criterion}
\newtheorem{definition}{Definition}
\newtheorem{lemma}{Lemma}
\newtheorem{problem}[theorem]{Problem}
\newtheorem{proposition}{Proposition}
\newtheorem{solution}[theorem]{Solution}
\newtheorem{summary}[theorem]{Summary}
\newtheorem{example}{Example}
\newtheorem{exercise}{Exercise}
\newtheorem{notation}{Notation}
\newtheorem{remark}{Remark}
\newcommand{\bmat}{\begin{matrix}}
\newcommand{\emat}{\end{matrix}}
\newcommand{\ov}{\overline}
\newcommand{\un}{\underline}
\newcommand{\EE}{\mathbb E}
\newcommand{\var}{\mathrm{var}}
\newcommand{\cov}{\mathrm{cov}}
\newcommand{\corr}{\mathrm{corr}}
\newcommand{\dd}{\displaystyle}
\newcommand{\ZZ}{\mathbb{Z}}
\newcommand{\RR}{\mathbb{R}}
\newcommand{\FF}{\mathbb F}
\newcommand{\LL}{\mathbb L}
\newcommand{\MM}{\mathbb M}
\newcommand{\KK}{\mathbb K}
\newcommand{\HH}{\mathbb H}
\newcommand{\QQ}{\mathbb Q}
\newcommand{\CC}{\mathbb{C}}
\newcommand{\Pin}{P_{\text{in}}}
\newcommand{\bx}{\mathbf{x}}
\newcommand{\bp}{\mathbf{p}}
\newcommand{\by}{\mathbf{y}}
\newcommand{\cP}{{\cal P}}
\newcommand{\bB}{\mathbf B}
\newcommand{\bM}{\mathbf M}
\newcommand{\bS}{\mathbf S}
\newcommand{\phis}{\varphi}
\newcommand{\barphis}{\overline\phis}
\newcommand{\se}{\text{se}}
\newcommand{\daga}{a^\dagger}
\newcommand{\devides}{\bigl |}
\newcommand{\eval}{\biggl |}
\newcommand{\ybar}{\overline y}
\newcommand{\bWhat}{\hat{\mathbf W}}
\newcommand{\bW}{\mathbf W}
\newcommand{\bz}{\mathbf z}
\newcommand{\bs}{\mathbf s}
\newcommand{\rightas}{\stackrel{a.s.}{\rightarrow}}
\newcommand{\rightp}{\stackrel{p}{\rightarrow}}
\newcommand{\rightd}{\stackrel{d}{\rightarrow}}
\newcommand{\bI}{\mathbf I}
\newcommand{\barB}{{\overline B}}
\newcommand{\barC}{{\overline C}}
\newcommand{\pbar}{{\overline p}}
\newcommand{\bbar}{{\overline b}}
\newcommand{\mubar}{{\overline \mu}}
\newenvironment{proof}[1][Proof]{\noindent \textbf{#1.} }{\  \rule{0.5em}{0.5em}}
\topmargin=-1cm
\oddsidemargin=-0cm
\textheight=22.2cm
\textwidth=16cm
\setcounter{secnumdepth}{2}
\pagestyle{plain}
\setcounter{figure}{0}
\sloppy
\input{sw20elba.sty}
% Macros for Scientific Word 3.5 documents saved with the LaTeX filter.
% Copyright (C) 2000 Mackichan Software, Inc.

\typeout{TCILATEX Macros for Scientific Word 3.5 <19 July 2000>.}
\typeout{NOTICE:  This macro file is NOT proprietary and may be 
freely copied and distributed.}
%
\makeatletter

%%%%%%%%%%%%%%%%%%%%%
% FMTeXButton
% This is used for putting TeXButtons in the 
% frontmatter of a document. Add a line like
% \QTagDef{FMTeXButton}{101}{} to the filter 
% section of the cst being used. Also add a
% new section containing:
%     [f_101]
%     ALIAS=FMTexButton
%     TAG_TYPE=FIELD
%     TAG_LEADIN=TeX Button:
%
% It also works to put \defs in the preamble after 
% the \input tcilatex
\def\FMTeXButton#1{#1}
%
%%%%%%%%%%%%%%%%%%%%%%
% macros for time
\newcount\@hour\newcount\@minute\chardef\@x10\chardef\@xv60
\def\tcitime{
\def\@time{%
  \@minute\time\@hour\@minute\divide\@hour\@xv
  \ifnum\@hour<\@x 0\fi\the\@hour:%
  \multiply\@hour\@xv\advance\@minute-\@hour
  \ifnum\@minute<\@x 0\fi\the\@minute
  }}%

%%%%%%%%%%%%%%%%%%%%%%
% macro for hyperref
%%% \@ifundefined{hyperref}{\def\hyperref#1#2#3#4{#2\ref{#4}#3}}{}

\def\x@hyperref#1#2#3{%
   % Trun off various catcodes before reading parameter 4
   \catcode`\~ = 12
   \catcode`\$ = 12
   \catcode`\_ = 12
   \catcode`\# = 12
   \catcode`\& = 12
   \y@hyperref{#1}{#2}{#3}%
}

\def\y@hyperref#1#2#3#4{%
   #2\ref{#4}#3
   \catcode`\~ = 13
   \catcode`\$ = 3
   \catcode`\_ = 8
   \catcode`\# = 6
   \catcode`\& = 4
}

\@ifundefined{hyperref}{\let\hyperref\x@hyperref}{}


% macro for external program call
\@ifundefined{qExtProgCall}{\def\qExtProgCall#1#2#3#4#5#6{\relax}}{}
%%%%%%%%%%%%%%%%%%%%%%
%
% macros for graphics
%
\def\FILENAME#1{#1}%
%
\def\QCTOpt[#1]#2{%
  \def\QCTOptB{#1}
  \def\QCTOptA{#2}
}
\def\QCTNOpt#1{%
  \def\QCTOptA{#1}
  \let\QCTOptB\empty
}
\def\Qct{%
  \@ifnextchar[{%
    \QCTOpt}{\QCTNOpt}
}
\def\QCBOpt[#1]#2{%
  \def\QCBOptB{#1}%
  \def\QCBOptA{#2}%
}
\def\QCBNOpt#1{%
  \def\QCBOptA{#1}%
  \let\QCBOptB\empty
}
\def\Qcb{%
  \@ifnextchar[{%
    \QCBOpt}{\QCBNOpt}%
}
\def\PrepCapArgs{%
  \ifx\QCBOptA\empty
    \ifx\QCTOptA\empty
      {}%
    \else
      \ifx\QCTOptB\empty
        {\QCTOptA}%
      \else
        [\QCTOptB]{\QCTOptA}%
      \fi
    \fi
  \else
    \ifx\QCBOptA\empty
      {}%
    \else
      \ifx\QCBOptB\empty
        {\QCBOptA}%
      \else
        [\QCBOptB]{\QCBOptA}%
      \fi
    \fi
  \fi
}
\newcount\GRAPHICSTYPE
%\GRAPHICSTYPE 0 is for TurboTeX
%\GRAPHICSTYPE 1 is for DVIWindo (PostScript)
%%%(removed)%\GRAPHICSTYPE 2 is for psfig (PostScript)
\GRAPHICSTYPE=\z@
\def\GRAPHICSPS#1{%
 \ifcase\GRAPHICSTYPE%\GRAPHICSTYPE=0
   \special{ps: #1}%
 \or%\GRAPHICSTYPE=1
   \special{language "PS", include "#1"}%
%%%\or%\GRAPHICSTYPE=2
%%%  #1%
 \fi
}%
%
\def\GRAPHICSHP#1{\special{include #1}}%
%
% \graffile{ body }                                  %#1
%          { contentswidth (scalar)  }               %#2
%          { contentsheight (scalar) }               %#3
%          { vertical shift when in-line (scalar) }  %#4

\def\graffile#1#2#3#4{%
%%% \ifnum\GRAPHICSTYPE=\tw@
%%%  %Following if using psfig
%%%  \@ifundefined{psfig}{\input psfig.tex}{}%
%%%  \psfig{file=#1, height=#3, width=#2}%
%%% \else
  %Following for all others
  % JCS - added BOXTHEFRAME, see below
    \bgroup
	   \@inlabelfalse
       \leavevmode
       \@ifundefined{bbl@deactivate}{\def~{\string~}}{\activesoff}%
        \raise -#4 \BOXTHEFRAME{%
           \hbox to #2{\raise #3\hbox to #2{\null #1\hfil}}}%
    \egroup
}%
%
% A box for drafts
\def\draftbox#1#2#3#4{%
 \leavevmode\raise -#4 \hbox{%
  \frame{\rlap{\protect\tiny #1}\hbox to #2%
   {\vrule height#3 width\z@ depth\z@\hfil}%
  }%
 }%
}%
%
\newcount\draft
\draft=\z@
\let\nographics=\draft
\newif\ifwasdraft
\wasdraftfalse

%  \GRAPHIC{ body }                                  %#1
%          { draft name }                            %#2
%          { contentswidth (scalar)  }               %#3
%          { contentsheight (scalar) }               %#4
%          { vertical shift when in-line (scalar) }  %#5
\def\GRAPHIC#1#2#3#4#5{%
   \ifnum\draft=\@ne\draftbox{#2}{#3}{#4}{#5}%
   \else\graffile{#1}{#3}{#4}{#5}%
   \fi
}
%
\def\addtoLaTeXparams#1{%
    \edef\LaTeXparams{\LaTeXparams #1}}%
%
% JCS -  added a switch BoxFrame that can 
% be set by including X in the frame params.
% If set a box is drawn around the frame.

\newif\ifBoxFrame \BoxFramefalse
\newif\ifOverFrame \OverFramefalse
\newif\ifUnderFrame \UnderFramefalse

\def\BOXTHEFRAME#1{%
   \hbox{%
      \ifBoxFrame
         \frame{#1}%
      \else
         {#1}%
      \fi
   }%
}


\def\doFRAMEparams#1{\BoxFramefalse\OverFramefalse\UnderFramefalse\readFRAMEparams#1\end}%
\def\readFRAMEparams#1{%
 \ifx#1\end%
  \let\next=\relax
  \else
  \ifx#1i\dispkind=\z@\fi
  \ifx#1d\dispkind=\@ne\fi
  \ifx#1f\dispkind=\tw@\fi
  \ifx#1t\addtoLaTeXparams{t}\fi
  \ifx#1b\addtoLaTeXparams{b}\fi
  \ifx#1p\addtoLaTeXparams{p}\fi
  \ifx#1h\addtoLaTeXparams{h}\fi
  \ifx#1X\BoxFrametrue\fi
  \ifx#1O\OverFrametrue\fi
  \ifx#1U\UnderFrametrue\fi
  \ifx#1w
    \ifnum\draft=1\wasdrafttrue\else\wasdraftfalse\fi
    \draft=\@ne
  \fi
  \let\next=\readFRAMEparams
  \fi
 \next
 }%
%
%Macro for In-line graphics object
%   \IFRAME{ contentswidth (scalar)  }               %#1
%          { contentsheight (scalar) }               %#2
%          { vertical shift when in-line (scalar) }  %#3
%          { draft name }                            %#4
%          { body }                                  %#5
%          { caption}                                %#6


\def\IFRAME#1#2#3#4#5#6{%
      \bgroup
      \let\QCTOptA\empty
      \let\QCTOptB\empty
      \let\QCBOptA\empty
      \let\QCBOptB\empty
      #6%
      \parindent=0pt
      \leftskip=0pt
      \rightskip=0pt
      \setbox0=\hbox{\QCBOptA}%
      \@tempdima=#1\relax
      \ifOverFrame
          % Do this later
          \typeout{This is not implemented yet}%
          \show\HELP
      \else
         \ifdim\wd0>\@tempdima
            \advance\@tempdima by \@tempdima
            \ifdim\wd0 >\@tempdima
               \setbox1 =\vbox{%
                  \unskip\hbox to \@tempdima{\hfill\GRAPHIC{#5}{#4}{#1}{#2}{#3}\hfill}%
                  \unskip\hbox to \@tempdima{\parbox[b]{\@tempdima}{\QCBOptA}}%
               }%
               \wd1=\@tempdima
            \else
               \textwidth=\wd0
               \setbox1 =\vbox{%
                 \noindent\hbox to \wd0{\hfill\GRAPHIC{#5}{#4}{#1}{#2}{#3}\hfill}\\%
                 \noindent\hbox{\QCBOptA}%
               }%
               \wd1=\wd0
            \fi
         \else
            \ifdim\wd0>0pt
              \hsize=\@tempdima
              \setbox1=\vbox{%
                \unskip\GRAPHIC{#5}{#4}{#1}{#2}{0pt}%
                \break
                \unskip\hbox to \@tempdima{\hfill \QCBOptA\hfill}%
              }%
              \wd1=\@tempdima
           \else
              \hsize=\@tempdima
              \setbox1=\vbox{%
                \unskip\GRAPHIC{#5}{#4}{#1}{#2}{0pt}%
              }%
              \wd1=\@tempdima
           \fi
         \fi
         \@tempdimb=\ht1
         %\advance\@tempdimb by \dp1
         \advance\@tempdimb by -#2
         \advance\@tempdimb by #3
         \leavevmode
         \raise -\@tempdimb \hbox{\box1}%
      \fi
      \egroup%
}%
%
%Macro for Display graphics object
%   \DFRAME{ contentswidth (scalar)  }               %#1
%          { contentsheight (scalar) }               %#2
%          { draft label }                           %#3
%          { name }                                  %#4
%          { caption}                                %#5
\def\DFRAME#1#2#3#4#5{%
 \begin{center}
     \let\QCTOptA\empty
     \let\QCTOptB\empty
     \let\QCBOptA\empty
     \let\QCBOptB\empty
	 \vbox\bgroup
        \ifOverFrame 
           #5\QCTOptA\par
        \fi
        \GRAPHIC{#4}{#3}{#1}{#2}{\z@}
        \ifUnderFrame 
           \par#5\QCBOptA
        \fi
	 \egroup
 \end{center}%
 }%
%
%Macro for Floating graphic object
%   \FFRAME{ framedata f|i tbph x F|T }              %#1
%          { contentswidth (scalar)  }               %#2
%          { contentsheight (scalar) }               %#3
%          { caption }                               %#4
%          { label }                                 %#5
%          { draft name }                            %#6
%          { body }                                  %#7
\def\FFRAME#1#2#3#4#5#6#7{%
 %If float.sty loaded and float option is 'h', change to 'H'  (gp) 1998/09/05
  \@ifundefined{floatstyle}
    {%floatstyle undefined (and float.sty not present), no change
     \begin{figure}[#1]%
    }
    {%floatstyle DEFINED
	 \ifx#1h%Only the h parameter, change to H
      \begin{figure}[H]%
	 \else
      \begin{figure}[#1]%
	 \fi
	}
  \let\QCTOptA\empty
  \let\QCTOptB\empty
  \let\QCBOptA\empty
  \let\QCBOptB\empty
  \ifOverFrame
    #4
    \ifx\QCTOptA\empty
    \else
      \ifx\QCTOptB\empty
        \caption{\QCTOptA}%
      \else
        \caption[\QCTOptB]{\QCTOptA}%
      \fi
    \fi
    \ifUnderFrame\else
      \label{#5}%
    \fi
  \else
    \UnderFrametrue%
  \fi
  \begin{center}\GRAPHIC{#7}{#6}{#2}{#3}{\z@}\end{center}%
  \ifUnderFrame
    #4
    \ifx\QCBOptA\empty
      \caption{}%
    \else
      \ifx\QCBOptB\empty
        \caption{\QCBOptA}%
      \else
        \caption[\QCBOptB]{\QCBOptA}%
      \fi
    \fi
    \label{#5}%
  \fi
  \end{figure}%
 }%
%
%
%    \FRAME{ framedata f|i tbph x F|T }              %#1
%          { contentswidth (scalar)  }               %#2
%          { contentsheight (scalar) }               %#3
%          { vertical shift when in-line (scalar) }  %#4
%          { caption }                               %#5
%          { label }                                 %#6
%          { name }                                  %#7
%          { body }                                  %#8
%
%    framedata is a string which can contain the following
%    characters: idftbphxFT
%    Their meaning is as follows:
%             i, d or f : in-line, display, or floating
%             t,b,p,h   : LaTeX floating placement options
%             x         : fit contents box to contents
%             F or T    : Figure or Table. 
%                         Later this can expand
%                         to a more general float class.
%
%
\newcount\dispkind%

\def\makeactives{
  \catcode`\"=\active
  \catcode`\;=\active
  \catcode`\:=\active
  \catcode`\'=\active
  \catcode`\~=\active
}
\bgroup
   \makeactives
   \gdef\activesoff{%
      \def"{\string"}
      \def;{\string;}
      \def:{\string:}
      \def'{\string'}
      \def~{\string~}
      %\bbl@deactivate{"}%
      %\bbl@deactivate{;}%
      %\bbl@deactivate{:}%
      %\bbl@deactivate{'}%
    }
\egroup

\def\FRAME#1#2#3#4#5#6#7#8{%
 \bgroup
 \ifnum\draft=\@ne
   \wasdrafttrue
 \else
   \wasdraftfalse%
 \fi
 \def\LaTeXparams{}%
 \dispkind=\z@
 \def\LaTeXparams{}%
 \doFRAMEparams{#1}%
 \ifnum\dispkind=\z@\IFRAME{#2}{#3}{#4}{#7}{#8}{#5}\else
  \ifnum\dispkind=\@ne\DFRAME{#2}{#3}{#7}{#8}{#5}\else
   \ifnum\dispkind=\tw@
    \edef\@tempa{\noexpand\FFRAME{\LaTeXparams}}%
    \@tempa{#2}{#3}{#5}{#6}{#7}{#8}%
    \fi
   \fi
  \fi
  \ifwasdraft\draft=1\else\draft=0\fi{}%
  \egroup
 }%
%
% This macro added to let SW gobble a parameter that
% should not be passed on and expanded. 

\def\TEXUX#1{"texux"}

%
% Macros for text attributes:
%
\def\BF#1{{\bf {#1}}}%
\def\NEG#1{\leavevmode\hbox{\rlap{\thinspace/}{$#1$}}}%
%
%%%%%%%%%%%%%%%%%%%%%%%%%%%%%%%%%%%%%%%%%%%%%%%%%%%%%%%%%%%%%%%%%%%%%%%%
%
%
% macros for user - defined functions
\def\limfunc#1{\mathop{\rm #1}}%
\def\func#1{\mathop{\rm #1}\nolimits}%
% macro for unit names
\def\unit#1{\mathop{\rm #1}\nolimits}%

%
% miscellaneous 
\long\def\QQQ#1#2{%
     \long\expandafter\def\csname#1\endcsname{#2}}%
\@ifundefined{QTP}{\def\QTP#1{}}{}
\@ifundefined{QEXCLUDE}{\def\QEXCLUDE#1{}}{}
\@ifundefined{Qlb}{\def\Qlb#1{#1}}{}
\@ifundefined{Qlt}{\def\Qlt#1{#1}}{}
\def\QWE{}%
\long\def\QQA#1#2{}%
\def\QTR#1#2{{\csname#1\endcsname #2}}%(gp) Is this the best?
\long\def\TeXButton#1#2{#2}%
\long\def\QSubDoc#1#2{#2}%
\def\EXPAND#1[#2]#3{}%
\def\NOEXPAND#1[#2]#3{}%
\def\PROTECTED{}%
\def\LaTeXparent#1{}%
\def\ChildStyles#1{}%
\def\ChildDefaults#1{}%
\def\QTagDef#1#2#3{}%

% Constructs added with Scientific Notebook
\@ifundefined{correctchoice}{\def\correctchoice{\relax}}{}
\@ifundefined{HTML}{\def\HTML#1{\relax}}{}
\@ifundefined{TCIIcon}{\def\TCIIcon#1#2#3#4{\relax}}{}
\if@compatibility
  \typeout{Not defining UNICODE  U or CustomNote commands for LaTeX 2.09.}
\else
  \providecommand{\UNICODE}[2][]{\protect\rule{.1in}{.1in}}
  \providecommand{\U}[1]{\protect\rule{.1in}{.1in}}
  \providecommand{\CustomNote}[3][]{\marginpar{#3}}
\fi

%
% Macros for style editor docs
\@ifundefined{StyleEditBeginDoc}{\def\StyleEditBeginDoc{\relax}}{}
%
% Macros for footnotes
\def\QQfnmark#1{\footnotemark}
\def\QQfntext#1#2{\addtocounter{footnote}{#1}\footnotetext{#2}}
%
% Macros for indexing.
%
\@ifundefined{TCIMAKEINDEX}{}{\makeindex}%
%
% Attempts to avoid problems with other styles
\@ifundefined{abstract}{%
 \def\abstract{%
  \if@twocolumn
   \section*{Abstract (Not appropriate in this style!)}%
   \else \small 
   \begin{center}{\bf Abstract\vspace{-.5em}\vspace{\z@}}\end{center}%
   \quotation 
   \fi
  }%
 }{%
 }%
\@ifundefined{endabstract}{\def\endabstract
  {\if@twocolumn\else\endquotation\fi}}{}%
\@ifundefined{maketitle}{\def\maketitle#1{}}{}%
\@ifundefined{affiliation}{\def\affiliation#1{}}{}%
\@ifundefined{proof}{\def\proof{\noindent{\bfseries Proof. }}}{}%
\@ifundefined{endproof}{\def\endproof{\mbox{\ \rule{.1in}{.1in}}}}{}%
\@ifundefined{newfield}{\def\newfield#1#2{}}{}%
\@ifundefined{chapter}{\def\chapter#1{\par(Chapter head:)#1\par }%
 \newcount\c@chapter}{}%
\@ifundefined{part}{\def\part#1{\par(Part head:)#1\par }}{}%
\@ifundefined{section}{\def\section#1{\par(Section head:)#1\par }}{}%
\@ifundefined{subsection}{\def\subsection#1%
 {\par(Subsection head:)#1\par }}{}%
\@ifundefined{subsubsection}{\def\subsubsection#1%
 {\par(Subsubsection head:)#1\par }}{}%
\@ifundefined{paragraph}{\def\paragraph#1%
 {\par(Subsubsubsection head:)#1\par }}{}%
\@ifundefined{subparagraph}{\def\subparagraph#1%
 {\par(Subsubsubsubsection head:)#1\par }}{}%
%%%%%%%%%%%%%%%%%%%%%%%%%%%%%%%%%%%%%%%%%%%%%%%%%%%%%%%%%%%%%%%%%%%%%%%%
% These symbols are not recognized by LaTeX
\@ifundefined{therefore}{\def\therefore{}}{}%
\@ifundefined{backepsilon}{\def\backepsilon{}}{}%
\@ifundefined{yen}{\def\yen{\hbox{\rm\rlap=Y}}}{}%
\@ifundefined{registered}{%
   \def\registered{\relax\ifmmode{}\r@gistered
                    \else$\m@th\r@gistered$\fi}%
 \def\r@gistered{^{\ooalign
  {\hfil\raise.07ex\hbox{$\scriptstyle\rm\text{R}$}\hfil\crcr
  \mathhexbox20D}}}}{}%
\@ifundefined{Eth}{\def\Eth{}}{}%
\@ifundefined{eth}{\def\eth{}}{}%
\@ifundefined{Thorn}{\def\Thorn{}}{}%
\@ifundefined{thorn}{\def\thorn{}}{}%
% A macro to allow any symbol that requires math to appear in text
\def\TEXTsymbol#1{\mbox{$#1$}}%
\@ifundefined{degree}{\def\degree{{}^{\circ}}}{}%
%
% macros for T3TeX files
\newdimen\theight
\@ifundefined{Column}{\def\Column{%
 \vadjust{\setbox\z@=\hbox{\scriptsize\quad\quad tcol}%
  \theight=\ht\z@\advance\theight by \dp\z@\advance\theight by \lineskip
  \kern -\theight \vbox to \theight{%
   \rightline{\rlap{\box\z@}}%
   \vss
   }%
  }%
 }}{}%
%
\@ifundefined{qed}{\def\qed{%
 \ifhmode\unskip\nobreak\fi\ifmmode\ifinner\else\hskip5\p@\fi\fi
 \hbox{\hskip5\p@\vrule width4\p@ height6\p@ depth1.5\p@\hskip\p@}%
 }}{}%
%
\@ifundefined{cents}{\def\cents{\hbox{\rm\rlap/c}}}{}%
\@ifundefined{miss}{\def\miss{\hbox{\vrule height2\p@ width 2\p@ depth\z@}}}{}%
%
\@ifundefined{vvert}{\def\vvert{\Vert}}{}%  %always translated to \left| or \right|
%
\@ifundefined{tcol}{\def\tcol#1{{\baselineskip=6\p@ \vcenter{#1}} \Column}}{}%
%
\@ifundefined{dB}{\def\dB{\hbox{{}}}}{}%        %dummy entry in column 
\@ifundefined{mB}{\def\mB#1{\hbox{$#1$}}}{}%   %column entry
\@ifundefined{nB}{\def\nB#1{\hbox{#1}}}{}%     %column entry (not math)
%
\@ifundefined{note}{\def\note{$^{\dag}}}{}%
%
\def\newfmtname{LaTeX2e}
% No longer load latexsym.  This is now handled by SWP, which uses amsfonts if necessary
%
\ifx\fmtname\newfmtname
  \DeclareOldFontCommand{\rm}{\normalfont\rmfamily}{\mathrm}
  \DeclareOldFontCommand{\sf}{\normalfont\sffamily}{\mathsf}
  \DeclareOldFontCommand{\tt}{\normalfont\ttfamily}{\mathtt}
  \DeclareOldFontCommand{\bf}{\normalfont\bfseries}{\mathbf}
  \DeclareOldFontCommand{\it}{\normalfont\itshape}{\mathit}
  \DeclareOldFontCommand{\sl}{\normalfont\slshape}{\@nomath\sl}
  \DeclareOldFontCommand{\sc}{\normalfont\scshape}{\@nomath\sc}
\fi

%
% Greek bold macros
% Redefine all of the math symbols 
% which might be bolded	 - there are 
% probably others to add to this list

\def\alpha{{\Greekmath 010B}}%
\def\beta{{\Greekmath 010C}}%
\def\gamma{{\Greekmath 010D}}%
\def\delta{{\Greekmath 010E}}%
\def\epsilon{{\Greekmath 010F}}%
\def\zeta{{\Greekmath 0110}}%
\def\eta{{\Greekmath 0111}}%
\def\theta{{\Greekmath 0112}}%
\def\iota{{\Greekmath 0113}}%
\def\kappa{{\Greekmath 0114}}%
\def\lambda{{\Greekmath 0115}}%
\def\mu{{\Greekmath 0116}}%
\def\nu{{\Greekmath 0117}}%
\def\xi{{\Greekmath 0118}}%
\def\pi{{\Greekmath 0119}}%
\def\rho{{\Greekmath 011A}}%
\def\sigma{{\Greekmath 011B}}%
\def\tau{{\Greekmath 011C}}%
\def\upsilon{{\Greekmath 011D}}%
\def\phi{{\Greekmath 011E}}%
\def\chi{{\Greekmath 011F}}%
\def\psi{{\Greekmath 0120}}%
\def\omega{{\Greekmath 0121}}%
\def\varepsilon{{\Greekmath 0122}}%
\def\vartheta{{\Greekmath 0123}}%
\def\varpi{{\Greekmath 0124}}%
\def\varrho{{\Greekmath 0125}}%
\def\varsigma{{\Greekmath 0126}}%
\def\varphi{{\Greekmath 0127}}%

\def\nabla{{\Greekmath 0272}}
\def\FindBoldGroup{%
   {\setbox0=\hbox{$\mathbf{x\global\edef\theboldgroup{\the\mathgroup}}$}}%
}

\def\Greekmath#1#2#3#4{%
    \if@compatibility
        \ifnum\mathgroup=\symbold
           \mathchoice{\mbox{\boldmath$\displaystyle\mathchar"#1#2#3#4$}}%
                      {\mbox{\boldmath$\textstyle\mathchar"#1#2#3#4$}}%
                      {\mbox{\boldmath$\scriptstyle\mathchar"#1#2#3#4$}}%
                      {\mbox{\boldmath$\scriptscriptstyle\mathchar"#1#2#3#4$}}%
        \else
           \mathchar"#1#2#3#4% 
        \fi 
    \else 
        \FindBoldGroup
        \ifnum\mathgroup=\theboldgroup % For 2e
           \mathchoice{\mbox{\boldmath$\displaystyle\mathchar"#1#2#3#4$}}%
                      {\mbox{\boldmath$\textstyle\mathchar"#1#2#3#4$}}%
                      {\mbox{\boldmath$\scriptstyle\mathchar"#1#2#3#4$}}%
                      {\mbox{\boldmath$\scriptscriptstyle\mathchar"#1#2#3#4$}}%
        \else
           \mathchar"#1#2#3#4% 
        \fi     	    
	  \fi}

\newif\ifGreekBold  \GreekBoldfalse
\let\SAVEPBF=\pbf
\def\pbf{\GreekBoldtrue\SAVEPBF}%
%

\@ifundefined{theorem}{\newtheorem{theorem}{Theorem}}{}
\@ifundefined{lemma}{\newtheorem{lemma}[theorem]{Lemma}}{}
\@ifundefined{corollary}{\newtheorem{corollary}[theorem]{Corollary}}{}
\@ifundefined{conjecture}{\newtheorem{conjecture}[theorem]{Conjecture}}{}
\@ifundefined{proposition}{\newtheorem{proposition}[theorem]{Proposition}}{}
\@ifundefined{axiom}{\newtheorem{axiom}{Axiom}}{}
\@ifundefined{remark}{\newtheorem{remark}{Remark}}{}
\@ifundefined{example}{\newtheorem{example}{Example}}{}
\@ifundefined{exercise}{\newtheorem{exercise}{Exercise}}{}
\@ifundefined{definition}{\newtheorem{definition}{Definition}}{}


\@ifundefined{mathletters}{%
  %\def\theequation{\arabic{equation}}
  \newcounter{equationnumber}  
  \def\mathletters{%
     \addtocounter{equation}{1}
     \edef\@currentlabel{\theequation}%
     \setcounter{equationnumber}{\c@equation}
     \setcounter{equation}{0}%
     \edef\theequation{\@currentlabel\noexpand\alph{equation}}%
  }
  \def\endmathletters{%
     \setcounter{equation}{\value{equationnumber}}%
  }
}{}

%Logos
\@ifundefined{BibTeX}{%
    \def\BibTeX{{\rm B\kern-.05em{\sc i\kern-.025em b}\kern-.08em
                 T\kern-.1667em\lower.7ex\hbox{E}\kern-.125emX}}}{}%
\@ifundefined{AmS}%
    {\def\AmS{{\protect\usefont{OMS}{cmsy}{m}{n}%
                A\kern-.1667em\lower.5ex\hbox{M}\kern-.125emS}}}{}%
\@ifundefined{AmSTeX}{\def\AmSTeX{\protect\AmS-\protect\TeX\@}}{}%
%

% This macro is a fix to eqnarray
\def\@@eqncr{\let\@tempa\relax
    \ifcase\@eqcnt \def\@tempa{& & &}\or \def\@tempa{& &}%
      \else \def\@tempa{&}\fi
     \@tempa
     \if@eqnsw
        \iftag@
           \@taggnum
        \else
           \@eqnnum\stepcounter{equation}%
        \fi
     \fi
     \global\tag@false
     \global\@eqnswtrue
     \global\@eqcnt\z@\cr}


\def\TCItag{\@ifnextchar*{\@TCItagstar}{\@TCItag}}
\def\@TCItag#1{%
    \global\tag@true
    \global\def\@taggnum{(#1)}}
\def\@TCItagstar*#1{%
    \global\tag@true
    \global\def\@taggnum{#1}}
%
%%%%%%%%%%%%%%%%%%%%%%%%%%%%%%%%%%%%%%%%%%%%%%%%%%%%%%%%%%%%%%%%%%%%%
%
\def\tfrac#1#2{{\textstyle {#1 \over #2}}}%
\def\dfrac#1#2{{\displaystyle {#1 \over #2}}}%
\def\binom#1#2{{#1 \choose #2}}%
\def\tbinom#1#2{{\textstyle {#1 \choose #2}}}%
\def\dbinom#1#2{{\displaystyle {#1 \choose #2}}}%
\def\QATOP#1#2{{#1 \atop #2}}%
\def\QTATOP#1#2{{\textstyle {#1 \atop #2}}}%
\def\QDATOP#1#2{{\displaystyle {#1 \atop #2}}}%
\def\QABOVE#1#2#3{{#2 \above#1 #3}}%
\def\QTABOVE#1#2#3{{\textstyle {#2 \above#1 #3}}}%
\def\QDABOVE#1#2#3{{\displaystyle {#2 \above#1 #3}}}%
\def\QOVERD#1#2#3#4{{#3 \overwithdelims#1#2 #4}}%
\def\QTOVERD#1#2#3#4{{\textstyle {#3 \overwithdelims#1#2 #4}}}%
\def\QDOVERD#1#2#3#4{{\displaystyle {#3 \overwithdelims#1#2 #4}}}%
\def\QATOPD#1#2#3#4{{#3 \atopwithdelims#1#2 #4}}%
\def\QTATOPD#1#2#3#4{{\textstyle {#3 \atopwithdelims#1#2 #4}}}%
\def\QDATOPD#1#2#3#4{{\displaystyle {#3 \atopwithdelims#1#2 #4}}}%
\def\QABOVED#1#2#3#4#5{{#4 \abovewithdelims#1#2#3 #5}}%
\def\QTABOVED#1#2#3#4#5{{\textstyle 
   {#4 \abovewithdelims#1#2#3 #5}}}%
\def\QDABOVED#1#2#3#4#5{{\displaystyle 
   {#4 \abovewithdelims#1#2#3 #5}}}%
%
% Macros for text size operators:
%
\def\tint{\mathop{\textstyle \int}}%
\def\tiint{\mathop{\textstyle \iint }}%
\def\tiiint{\mathop{\textstyle \iiint }}%
\def\tiiiint{\mathop{\textstyle \iiiint }}%
\def\tidotsint{\mathop{\textstyle \idotsint }}%
\def\toint{\mathop{\textstyle \oint}}%
\def\tsum{\mathop{\textstyle \sum }}%
\def\tprod{\mathop{\textstyle \prod }}%
\def\tbigcap{\mathop{\textstyle \bigcap }}%
\def\tbigwedge{\mathop{\textstyle \bigwedge }}%
\def\tbigoplus{\mathop{\textstyle \bigoplus }}%
\def\tbigodot{\mathop{\textstyle \bigodot }}%
\def\tbigsqcup{\mathop{\textstyle \bigsqcup }}%
\def\tcoprod{\mathop{\textstyle \coprod }}%
\def\tbigcup{\mathop{\textstyle \bigcup }}%
\def\tbigvee{\mathop{\textstyle \bigvee }}%
\def\tbigotimes{\mathop{\textstyle \bigotimes }}%
\def\tbiguplus{\mathop{\textstyle \biguplus }}%
%
%
%Macros for display size operators:
%
\def\dint{\displaystyle \int}%
\def\diint{\displaystyle \iint}%
\def\diiint{\displaystyle \iiint}%
\def\diiiint{\mathop{\displaystyle \iiiint }}%
\def\didotsint{\mathop{\displaystyle \idotsint }}%
\def\doint{\mathop{\displaystyle \oint}}%
\def\dsum{\mathop{\displaystyle \sum }}%
\def\dprod{\mathop{\displaystyle \prod }}%
\def\dbigcap{\mathop{\displaystyle \bigcap }}%
\def\dbigwedge{\mathop{\displaystyle \bigwedge }}%
\def\dbigoplus{\mathop{\displaystyle \bigoplus }}%
\def\dbigodot{\mathop{\displaystyle \bigodot }}%
\def\dbigsqcup{\mathop{\displaystyle \bigsqcup }}%
\def\dcoprod{\mathop{\displaystyle \coprod }}%
\def\dbigcup{\mathop{\displaystyle \bigcup }}%
\def\dbigvee{\mathop{\displaystyle \bigvee }}%
\def\dbigotimes{\mathop{\displaystyle \bigotimes }}%
\def\dbiguplus{\mathop{\displaystyle \biguplus }}%

%%%%%%%%%%%%%%%%%%%%%%%%%%%%%%%%%%%%%%%%%%%%%%%%%%%%%%%%%%%%%%%%%%%%%%%
% NOTE: The rest of this file is read only if amstex has not been
% loaded.  This section is used to define amstex constructs in the
% event they have not been defined.
%
%


\def\ExitTCILatex{\makeatother\endinput}

\bgroup
\ifx\ds@amstex\relax
   \message{amstex already loaded}\aftergroup\ExitTCILatex
\else
   \@ifpackageloaded{amsmath}%
      {\message{amsmath already loaded}\aftergroup\ExitTCILatex}
      {}
   \@ifpackageloaded{amstex}%
      {\message{amstex already loaded}\aftergroup\ExitTCILatex}
      {}
   \@ifpackageloaded{amsgen}%
      {\message{amsgen already loaded}\aftergroup\ExitTCILatex}
      {}
\fi
\egroup


%%%%%%%%%%%%%%%%%%%%%%%%%%%%%%%%%%%%%%%%%%%%%%%%%%%%%%%%%%%%%%%%%%%%%%%%
%%
%
%
%  Macros to define some AMS LaTeX constructs when 
%  AMS LaTeX has not been loaded
% 
% These macros are copied from the AMS-TeX package for doing
% multiple integrals.
%
\typeout{TCILATEX defining AMS-like constructs}
\let\DOTSI\relax
\def\RIfM@{\relax\ifmmode}%
\def\FN@{\futurelet\next}%
\newcount\intno@
\def\iint{\DOTSI\intno@\tw@\FN@\ints@}%
\def\iiint{\DOTSI\intno@\thr@@\FN@\ints@}%
\def\iiiint{\DOTSI\intno@4 \FN@\ints@}%
\def\idotsint{\DOTSI\intno@\z@\FN@\ints@}%
\def\ints@{\findlimits@\ints@@}%
\newif\iflimtoken@
\newif\iflimits@
\def\findlimits@{\limtoken@true\ifx\next\limits\limits@true
 \else\ifx\next\nolimits\limits@false\else
 \limtoken@false\ifx\ilimits@\nolimits\limits@false\else
 \ifinner\limits@false\else\limits@true\fi\fi\fi\fi}%
\def\multint@{\int\ifnum\intno@=\z@\intdots@                          %1
 \else\intkern@\fi                                                    %2
 \ifnum\intno@>\tw@\int\intkern@\fi                                   %3
 \ifnum\intno@>\thr@@\int\intkern@\fi                                 %4
 \int}%                                                               %5
\def\multintlimits@{\intop\ifnum\intno@=\z@\intdots@\else\intkern@\fi
 \ifnum\intno@>\tw@\intop\intkern@\fi
 \ifnum\intno@>\thr@@\intop\intkern@\fi\intop}%
\def\intic@{%
    \mathchoice{\hskip.5em}{\hskip.4em}{\hskip.4em}{\hskip.4em}}%
\def\negintic@{\mathchoice
 {\hskip-.5em}{\hskip-.4em}{\hskip-.4em}{\hskip-.4em}}%
\def\ints@@{\iflimtoken@                                              %1
 \def\ints@@@{\iflimits@\negintic@
   \mathop{\intic@\multintlimits@}\limits                             %2
  \else\multint@\nolimits\fi                                          %3
  \eat@}%                                                             %4
 \else                                                                %5
 \def\ints@@@{\iflimits@\negintic@
  \mathop{\intic@\multintlimits@}\limits\else
  \multint@\nolimits\fi}\fi\ints@@@}%
\def\intkern@{\mathchoice{\!\!\!}{\!\!}{\!\!}{\!\!}}%
\def\plaincdots@{\mathinner{\cdotp\cdotp\cdotp}}%
\def\intdots@{\mathchoice{\plaincdots@}%
 {{\cdotp}\mkern1.5mu{\cdotp}\mkern1.5mu{\cdotp}}%
 {{\cdotp}\mkern1mu{\cdotp}\mkern1mu{\cdotp}}%
 {{\cdotp}\mkern1mu{\cdotp}\mkern1mu{\cdotp}}}%
%
%
%  These macros are for doing the AMS \text{} construct
%
\def\RIfM@{\relax\protect\ifmmode}
\def\text{\RIfM@\expandafter\text@\else\expandafter\mbox\fi}
\let\nfss@text\text
\def\text@#1{\mathchoice
   {\textdef@\displaystyle\f@size{#1}}%
   {\textdef@\textstyle\tf@size{\firstchoice@false #1}}%
   {\textdef@\textstyle\sf@size{\firstchoice@false #1}}%
   {\textdef@\textstyle \ssf@size{\firstchoice@false #1}}%
   \glb@settings}

\def\textdef@#1#2#3{\hbox{{%
                    \everymath{#1}%
                    \let\f@size#2\selectfont
                    #3}}}
\newif\iffirstchoice@
\firstchoice@true
%
%These are the AMS constructs for multiline limits.
%
\def\Let@{\relax\iffalse{\fi\let\\=\cr\iffalse}\fi}%
\def\vspace@{\def\vspace##1{\crcr\noalign{\vskip##1\relax}}}%
\def\multilimits@{\bgroup\vspace@\Let@
 \baselineskip\fontdimen10 \scriptfont\tw@
 \advance\baselineskip\fontdimen12 \scriptfont\tw@
 \lineskip\thr@@\fontdimen8 \scriptfont\thr@@
 \lineskiplimit\lineskip
 \vbox\bgroup\ialign\bgroup\hfil$\m@th\scriptstyle{##}$\hfil\crcr}%
\def\Sb{_\multilimits@}%
\def\endSb{\crcr\egroup\egroup\egroup}%
\def\Sp{^\multilimits@}%
\let\endSp\endSb
%
%
%These are AMS constructs for horizontal arrows
%
\newdimen\ex@
\ex@.2326ex
\def\rightarrowfill@#1{$#1\m@th\mathord-\mkern-6mu\cleaders
 \hbox{$#1\mkern-2mu\mathord-\mkern-2mu$}\hfill
 \mkern-6mu\mathord\rightarrow$}%
\def\leftarrowfill@#1{$#1\m@th\mathord\leftarrow\mkern-6mu\cleaders
 \hbox{$#1\mkern-2mu\mathord-\mkern-2mu$}\hfill\mkern-6mu\mathord-$}%
\def\leftrightarrowfill@#1{$#1\m@th\mathord\leftarrow
\mkern-6mu\cleaders
 \hbox{$#1\mkern-2mu\mathord-\mkern-2mu$}\hfill
 \mkern-6mu\mathord\rightarrow$}%
\def\overrightarrow{\mathpalette\overrightarrow@}%
\def\overrightarrow@#1#2{\vbox{\ialign{##\crcr\rightarrowfill@#1\crcr
 \noalign{\kern-\ex@\nointerlineskip}$\m@th\hfil#1#2\hfil$\crcr}}}%
\let\overarrow\overrightarrow
\def\overleftarrow{\mathpalette\overleftarrow@}%
\def\overleftarrow@#1#2{\vbox{\ialign{##\crcr\leftarrowfill@#1\crcr
 \noalign{\kern-\ex@\nointerlineskip}$\m@th\hfil#1#2\hfil$\crcr}}}%
\def\overleftrightarrow{\mathpalette\overleftrightarrow@}%
\def\overleftrightarrow@#1#2{\vbox{\ialign{##\crcr
   \leftrightarrowfill@#1\crcr
 \noalign{\kern-\ex@\nointerlineskip}$\m@th\hfil#1#2\hfil$\crcr}}}%
\def\underrightarrow{\mathpalette\underrightarrow@}%
\def\underrightarrow@#1#2{\vtop{\ialign{##\crcr$\m@th\hfil#1#2\hfil
  $\crcr\noalign{\nointerlineskip}\rightarrowfill@#1\crcr}}}%
\let\underarrow\underrightarrow
\def\underleftarrow{\mathpalette\underleftarrow@}%
\def\underleftarrow@#1#2{\vtop{\ialign{##\crcr$\m@th\hfil#1#2\hfil
  $\crcr\noalign{\nointerlineskip}\leftarrowfill@#1\crcr}}}%
\def\underleftrightarrow{\mathpalette\underleftrightarrow@}%
\def\underleftrightarrow@#1#2{\vtop{\ialign{##\crcr$\m@th
  \hfil#1#2\hfil$\crcr
 \noalign{\nointerlineskip}\leftrightarrowfill@#1\crcr}}}%
%%%%%%%%%%%%%%%%%%%%%

\def\qopnamewl@#1{\mathop{\operator@font#1}\nlimits@}
\let\nlimits@\displaylimits
\def\setboxz@h{\setbox\z@\hbox}


\def\varlim@#1#2{\mathop{\vtop{\ialign{##\crcr
 \hfil$#1\m@th\operator@font lim$\hfil\crcr
 \noalign{\nointerlineskip}#2#1\crcr
 \noalign{\nointerlineskip\kern-\ex@}\crcr}}}}

 \def\rightarrowfill@#1{\m@th\setboxz@h{$#1-$}\ht\z@\z@
  $#1\copy\z@\mkern-6mu\cleaders
  \hbox{$#1\mkern-2mu\box\z@\mkern-2mu$}\hfill
  \mkern-6mu\mathord\rightarrow$}
\def\leftarrowfill@#1{\m@th\setboxz@h{$#1-$}\ht\z@\z@
  $#1\mathord\leftarrow\mkern-6mu\cleaders
  \hbox{$#1\mkern-2mu\copy\z@\mkern-2mu$}\hfill
  \mkern-6mu\box\z@$}


\def\projlim{\qopnamewl@{proj\,lim}}
\def\injlim{\qopnamewl@{inj\,lim}}
\def\varinjlim{\mathpalette\varlim@\rightarrowfill@}
\def\varprojlim{\mathpalette\varlim@\leftarrowfill@}
\def\varliminf{\mathpalette\varliminf@{}}
\def\varliminf@#1{\mathop{\underline{\vrule\@depth.2\ex@\@width\z@
   \hbox{$#1\m@th\operator@font lim$}}}}
\def\varlimsup{\mathpalette\varlimsup@{}}
\def\varlimsup@#1{\mathop{\overline
  {\hbox{$#1\m@th\operator@font lim$}}}}

%
%Companion to stackrel
\def\stackunder#1#2{\mathrel{\mathop{#2}\limits_{#1}}}%
%
%
% These are AMS environments that will be defined to
% be verbatims if amstex has not actually been 
% loaded
%
%
\begingroup \catcode `|=0 \catcode `[= 1
\catcode`]=2 \catcode `\{=12 \catcode `\}=12
\catcode`\\=12 
|gdef|@alignverbatim#1\end{align}[#1|end[align]]
|gdef|@salignverbatim#1\end{align*}[#1|end[align*]]

|gdef|@alignatverbatim#1\end{alignat}[#1|end[alignat]]
|gdef|@salignatverbatim#1\end{alignat*}[#1|end[alignat*]]

|gdef|@xalignatverbatim#1\end{xalignat}[#1|end[xalignat]]
|gdef|@sxalignatverbatim#1\end{xalignat*}[#1|end[xalignat*]]

|gdef|@gatherverbatim#1\end{gather}[#1|end[gather]]
|gdef|@sgatherverbatim#1\end{gather*}[#1|end[gather*]]

|gdef|@gatherverbatim#1\end{gather}[#1|end[gather]]
|gdef|@sgatherverbatim#1\end{gather*}[#1|end[gather*]]


|gdef|@multilineverbatim#1\end{multiline}[#1|end[multiline]]
|gdef|@smultilineverbatim#1\end{multiline*}[#1|end[multiline*]]

|gdef|@arraxverbatim#1\end{arrax}[#1|end[arrax]]
|gdef|@sarraxverbatim#1\end{arrax*}[#1|end[arrax*]]

|gdef|@tabulaxverbatim#1\end{tabulax}[#1|end[tabulax]]
|gdef|@stabulaxverbatim#1\end{tabulax*}[#1|end[tabulax*]]


|endgroup
  

  
\def\align{\@verbatim \frenchspacing\@vobeyspaces \@alignverbatim
You are using the "align" environment in a style in which it is not defined.}
\let\endalign=\endtrivlist
 
\@namedef{align*}{\@verbatim\@salignverbatim
You are using the "align*" environment in a style in which it is not defined.}
\expandafter\let\csname endalign*\endcsname =\endtrivlist




\def\alignat{\@verbatim \frenchspacing\@vobeyspaces \@alignatverbatim
You are using the "alignat" environment in a style in which it is not defined.}
\let\endalignat=\endtrivlist
 
\@namedef{alignat*}{\@verbatim\@salignatverbatim
You are using the "alignat*" environment in a style in which it is not defined.}
\expandafter\let\csname endalignat*\endcsname =\endtrivlist




\def\xalignat{\@verbatim \frenchspacing\@vobeyspaces \@xalignatverbatim
You are using the "xalignat" environment in a style in which it is not defined.}
\let\endxalignat=\endtrivlist
 
\@namedef{xalignat*}{\@verbatim\@sxalignatverbatim
You are using the "xalignat*" environment in a style in which it is not defined.}
\expandafter\let\csname endxalignat*\endcsname =\endtrivlist




\def\gather{\@verbatim \frenchspacing\@vobeyspaces \@gatherverbatim
You are using the "gather" environment in a style in which it is not defined.}
\let\endgather=\endtrivlist
 
\@namedef{gather*}{\@verbatim\@sgatherverbatim
You are using the "gather*" environment in a style in which it is not defined.}
\expandafter\let\csname endgather*\endcsname =\endtrivlist


\def\multiline{\@verbatim \frenchspacing\@vobeyspaces \@multilineverbatim
You are using the "multiline" environment in a style in which it is not defined.}
\let\endmultiline=\endtrivlist
 
\@namedef{multiline*}{\@verbatim\@smultilineverbatim
You are using the "multiline*" environment in a style in which it is not defined.}
\expandafter\let\csname endmultiline*\endcsname =\endtrivlist


\def\arrax{\@verbatim \frenchspacing\@vobeyspaces \@arraxverbatim
You are using a type of "array" construct that is only allowed in AmS-LaTeX.}
\let\endarrax=\endtrivlist

\def\tabulax{\@verbatim \frenchspacing\@vobeyspaces \@tabulaxverbatim
You are using a type of "tabular" construct that is only allowed in AmS-LaTeX.}
\let\endtabulax=\endtrivlist

 
\@namedef{arrax*}{\@verbatim\@sarraxverbatim
You are using a type of "array*" construct that is only allowed in AmS-LaTeX.}
\expandafter\let\csname endarrax*\endcsname =\endtrivlist

\@namedef{tabulax*}{\@verbatim\@stabulaxverbatim
You are using a type of "tabular*" construct that is only allowed in AmS-LaTeX.}
\expandafter\let\csname endtabulax*\endcsname =\endtrivlist

% macro to simulate ams tag construct


% This macro is a fix to the equation environment
 \def\endequation{%
     \ifmmode\ifinner % FLEQN hack
      \iftag@
        \addtocounter{equation}{-1} % undo the increment made in the begin part
        $\hfil
           \displaywidth\linewidth\@taggnum\egroup \endtrivlist
        \global\tag@false
        \global\@ignoretrue   
      \else
        $\hfil
           \displaywidth\linewidth\@eqnnum\egroup \endtrivlist
        \global\tag@false
        \global\@ignoretrue 
      \fi
     \else   
      \iftag@
        \addtocounter{equation}{-1} % undo the increment made in the begin part
        \eqno \hbox{\@taggnum}
        \global\tag@false%
        $$\global\@ignoretrue
      \else
        \eqno \hbox{\@eqnnum}% $$ BRACE MATCHING HACK
        $$\global\@ignoretrue
      \fi
     \fi\fi
 } 

 \newif\iftag@ \tag@false
 
 \def\TCItag{\@ifnextchar*{\@TCItagstar}{\@TCItag}}
 \def\@TCItag#1{%
     \global\tag@true
     \global\def\@taggnum{(#1)}}
 \def\@TCItagstar*#1{%
     \global\tag@true
     \global\def\@taggnum{#1}}

  \@ifundefined{tag}{
     \def\tag{\@ifnextchar*{\@tagstar}{\@tag}}
     \def\@tag#1{%
         \global\tag@true
         \global\def\@taggnum{(#1)}}
     \def\@tagstar*#1{%
         \global\tag@true
         \global\def\@taggnum{#1}}
  }{}
% Do not add anything to the end of this file.  
% The last section of the file is loaded only if 
% amstex has not been.



\makeatother
\endinput


%\setpagewiselinenumbers
%\linenumbers
\renewcommand{\thetable}{\Roman{table}}
\renewcommand{\thefigure}{\Roman{figure}}
\begin{document}

\author{\textbf{Anmol Bhandari}\\apb296@nyu.edu \and \textbf{David Evans} \\ \texttt{dgevans@nyu.edu} \and \textbf{Mikhail Golosov}\\\texttt{golosov@princeton.edu} \and \textbf{Thomas J. Sargent} \\ \texttt{thomas.sargent@nyu.edu}
}
\title{\textbf{Taxes, debts,  and redistributions with aggregate shocks%
\thanks{%
We thank Mark Aguiar, Stefania Albanesi, Manuel Amador,  Andrew Atkeson, Marco Bassetto, V.V. Chari, Harold
L. Cole, Guy Laroque, Francesco Lippi, Robert E. Lucas, Jr., Ali Shourideh, Pierre Yared and seminar
participants at Bocconi, Chicago, EIEF, the Federal Reserve Bank of
Minneapolis, IES, Princeton, Stanford, UCL, Universidade Cat\'{o}lica, 2012
Minnesota macro conference, Monetary Policy Workshop NY Fed for helpful
comments.}}}
\date{November 2014}
\maketitle
\newpage
\appendix
\section{Appendix}


\subsection{Proof of Theorem\ref{prop: affine eqm nec and suff}}
\label{appndx: affine eqm nec and stuff}
\smallskip

We prove a slight more general version of our result. Consider an infinite
horizon, incomplete markets economy in which an agent maximizes utility
function $U:\mathbb{R}_{+}^{n}\rightarrow \mathbb{R}$ subject to an infinite
sequence of budget constraints. We assume that $U$ is concave and
differentiable. Let $\mathbf{x}(s^t)$ be a vector of $n$ goods and let $%
\mathbf{p}(s^{t})$ be a price vector in state $s^{t}$ with $p_{i}(s^{t})$
denoting the price of good $i.$ We use a normalization $p_{1}\left(
s^{t}\right) =1$ for all $s^{t}.$ 
Let $b(s^{t})$ be the agent's asset holdings, and let $\mathbf{e}\left(
s^{t}\right) $ be a stochastic vector of endowments.

\textbf{Consumer maximization problem}

\begin{equation}
\max_{\mathbf{x}_{t},b_{t}}\sum_{t=0}^{\infty }\beta ^{t}\Pr \left(
s^{t}\right) U(\mathbf{x}\left( s^{t}\right) )
\label{Tech appendix: consumer maximization}
\end{equation}%
subject to%
\begin{equation}
\mathbf{p}\left( s^{t}\right) \mathbf{x}\left( s^{t}\right) +q(s^{t})b\left(
s^{t}\right) =\mathbf{p}\left( s^{t}\right) \mathbf{e}\left( s^{t}\right)
+P(s_t)b\left( s^{t-1}\right)  \label{Tech appendix: budget constraint}
\end{equation}%
and $\left \{ b\left( s^{t}\right) \right \} $ is bounded and $\left \{
q(s^{t})\right \} $ is the price of the risk-free bond.

The Euler conditions are%
\begin{eqnarray}
\mathbf{U}_{x}(s^{t}) &=&U_{1}(s^{t})\mathbf{p}(s^{t})
\label{Tech appendix: Euler} \\
\Pr \left( s^{t}\right) U_{1}\left( s^{t}\right) q(s^{t}) &=&\beta
\sum_{s^{t+1}>s^{t}}\Pr \left( s^{t+1}\right) U_{1}\left( s^{t+1}\right) .
\notag
\end{eqnarray}

\begin{theorem}

\smallskip Consider an allocation $\left \{ \mathbf{x}_{t},b_{t}\right \} $
that satisfies (\ref{Tech appendix: budget constraint}), (\ref{Tech
appendix: Euler}) and $\left \{ b_{t}\right \} _{t}$ is bounded. Then $%
\left
\{ \mathbf{x}_{t},b_{t}\right \} $ is a solution to (\ref{Tech
appendix: consumer maximization}).

\end{theorem}
\begin{proof}
The proof follows closely Constantinides and Duffie (1996). Suppose there is
another budget feasible allocation $\mathbf{x}+\mathbf{h}$ that maximizes (%
\ref{Tech appendix: consumer maximization}). Since $U$ is strictly concave,
\begin{eqnarray}
&&\mathbb{E}_{0}\sum_{t=0}^{\infty }\beta ^{t}U(\mathbf{x}_{t}+\mathbf{h}%
_{t})-\mathbb{E}_{0}\sum_{t=0}^{\infty }\beta ^{t}U(\mathbf{x}_{t})
\label{Tech appendix: gain from h} \\
&\leq &\mathbb{E}_{0}\sum_{t=0}^{\infty }\beta ^{t}\mathbf{U}_{x}(\mathbf{x}%
_{t})\mathbf{h}_{t}  \notag
\end{eqnarray}

To attain $\mathbf{x}+\mathbf{h}$, the agent must deviate by $\varphi _{t}$
from his original portfolio $b_{t}$ such that $\left\{ \varphi _{t}\right\}
_{t}$ is bounded$,$ $\varphi _{-1}=0$ and
\begin{equation*}
\mathbf{p}(s^{t})\mathbf{h}\left( s^{t}\right) =P(s_t)\varphi
(s^{t-1})-q(s^{t})\varphi (s^{t})
\end{equation*}%
Multiply by $\beta ^{t}\Pr \left( s^{t}\right) U_{1}(s^{t})$ to get:%
\begin{eqnarray*}
\beta ^{t}\Pr \left( s^{t}\right) U_{1}(s^{t})\mathbf{p}(s^{t})\mathbf{h}%
\left( s^{t}\right)  &=&\beta ^{t}\Pr \left( s^{t}\right)
U_{1}(s^{t})\varphi (s^{t-1})-q(s^{t})\beta ^{t}\Pr \left( s^{t}\right)
U_{1}(s^{t})\varphi (s^{t}) \\
&=&\beta ^{t}\Pr \left( s^{t}\right) U_{1}(s^{t})\varphi (s^{t-1})-\beta
^{t+1}\sum_{s^{t+1}>s^{t}}\Pr \left( s^{t+1}\right) U_{1}\left(
s^{t+1}\right) \varphi (s^{t})
\end{eqnarray*}%
where we used the second part of (\ref{Tech appendix: Euler}) in the second
equality. Sum over the first $T$ periods and use the first part of (\ref%
{Tech appendix: Euler}) to eliminate $\mathbf{U}_{x}(\mathbf{x}%
_{t})=U_{1}(s^{t})\mathbf{p}(s^{t})$%
\begin{equation*}
\sum_{t=0}^{T}\beta ^{t}\Pr \left( s^{t}\right) \mathbf{U}_{x}(\mathbf{x}%
_{t})\mathbf{h}\left( s^{t}\right) =-\sum_{s^{T+1}>s^{T}}\beta ^{T+1}\Pr
\left( s^{T+1}\right) U_{1}\left( s^{T+1}\right) \varphi (s^{T}).
\end{equation*}%
Since $\left\{ \varphi _{t}\right\} _{t}$ is bounded there must exist $\bar{%
\varphi}$ s.t. $|\varphi _{t}|\leq \bar{\varphi}$ for all $t$. By Theorem
5.2 of Magill and Quinzii (1994), this equilibrium with debt constraints
implies a transversality condition on the right hand side of the last
equation, so by transitivity we have%
\begin{equation*}
\lim_{T\rightarrow \infty }\sum_{t=0}^{T}\beta ^{t}\Pr \left( s^{t}\right)
\mathbf{U}_{x}(\mathbf{x}_{t})\mathbf{h}\left( s^{t}\right) =0.
\end{equation*}%
Substitute this into (\ref{Tech appendix: gain from h}) to show that $%
\mathbf{h}$ does not improve utility of consumer.
\end{proof}
\newpage

\subsection{Proof of Theorem \protect\ref{thm: rep agent general theorem}}


\begin{proof}
The optimal Ramsey plan solves the following Bellman equation. Let $V(b\_)$ be the maximum ex-ante value the government can achieve with debt $b\_$. 

\begin{equation}
  \label{eq-QLRA obj}
    V(b\_)=\max_{c(s),l(s),b(s)} \sum_{s}\pi(s)\left\{c(s)-\frac{l(s)^{1+\gamma}}{1+\gamma}+\beta V(b(s)) \right\}
\end{equation}   
subject to

   \begin{subequations}
   \label{sys- QLRA constraint}
    \begin{equation}
    \label{rep agent implementability constraint }
    c(s)+b(s)=l(s)^{1+\gamma}+\beta^{-1} P(s)b\_
    \end{equation}
 

 
\begin{equation}
  \label{eq-resoruces}
c(s)+g(s)\leq\theta l(s)
\end{equation}   
Let $\bar{b}=-\underline{B}$
\begin{equation}
  \label{ndl}
\underline{b}\leq b(s)\leq \bar{b}
\end{equation}   


   \end{subequations}
%Let $\mu(s)\pi(s),\phi(s)\pi(s)$ be the Lagrange multipliers on the respective constraints. 

{Part 1 of Theorem \textbf{\ref{thm: rep agent general theorem}}}

\begin{lemma}
There exists  a $\overline{b}$ such that $b_t\leq\overline{b}$. This is the natural debt limit for the government.
\end{lemma}
\begin{proof}
As we drive $\mu$ to $-\infty$, the tax rate approaches a maximum limit, $\bar{\tau}=\frac{\gamma}{1+\gamma}$. In state $s$, the government surplus,
\[
  S(s,\tau) = \theta^\frac\gamma{1+\gamma}(1-\tau)^\frac1\gamma\tau - g(s),
\]  which  is  maximized at $\tau = \frac\gamma{1+\gamma}$ when $(1-\tau)^\frac1\gamma\tau$ is also maximized. This would impose a natural borrowing limit for the government. 

\end{proof}

From now we assume that $\overline{b}$ represents the natural borrowing limit. We begin with some useful lemmas

let $L\equiv l^{1+\gamma }.$, To make this problem convex, 

Substitute for $c\left( s\right) $%
\[
V\left( b\_\right) =\max_{L(s),b(s)}\sum_{s\in S}\pi \left(
s\right) \left[ \frac{1}{1+\gamma }L\left( s\right) +\frac{1}{\beta }%
P\left( s\right) b\_-b\left( s\right) +\beta V\left( b\left( s\right)
\right) \right] 
\]%
s.t.%
\begin{eqnarray*}
\frac{1}{\beta }P\left( s\right) b-b\left( s\right) +g\left( s\right) &\leq
&\theta L^{\frac{1}{1+\gamma} }\left( s\right) -L\left( s\right) \\
b\left( s\right) &\leq &\bar{b} \\
L\left( s\right) &\geq &0.
\end{eqnarray*}

\begin{lemma}
\smallskip $V\left( b\right) $ is stictly concave, continuous,
differentiable and $V\left( b\right) <\beta ^{-1}$ for all $%
b<\bar{b}.$ The feasibility constraint binds for all $b\in (-\infty ,\bar{b}%
],\ s\in S$ and $\left( L^{\ast }\left( s\right) \right) ^{1-\frac{1}{1+\gamma} }\geq
\frac{1}{1+\gamma} .\footnote{
This last condition simply means that we do not tax to the right of the peak
of the Laffer curve. The revenue maximizing tax is $1-\bar{\tau}=\frac{1}{%
1+\gamma }.$ At the same time $1-\tau =l^{\gamma}$ so if taxes are always
to the left of the peak, $\frac{1}{1+\gamma }\leq l^{\gamma }=\left(
L^{\frac{1}{1+\gamma} }\right) ^{\gamma}=L^{1-\frac{1}{1+\gamma} }$.}$ 
\end{lemma}

\begin{proof}
\smallskip \textit{Concavity}

$V\left( b\right) $ is concave because we maximize linear objective
function over convex set.

\textit{Binding feasibility}

Suppose that feasibility does not bind for some $b,s.$ Then the optimal $%
L\left( s\right) $ solve $\max_{L\left( s\right) \geq 0}\pi \left( s\right) 
\frac{\gamma}{1+\gamma }L\left( s\right) $ which sets $L\left( s\right)
=\infty .$ This violates feasility for any finite $b,b\left( s\right) .$

\textit{Bounds on }$L$

Let $\phi\left( s\right) >0$ be a Lagrange multiplier on the
feasibility. \ The FOC for $L\left( s\right) $ is 
\[
\frac{1}{1+\gamma }+\phi(s) \left( \frac{1}{1+\gamma }L(s)^{\frac{1}{1+\gamma}
-}-\theta \right) =0. 
\]%
This gives%
\[
\frac{1}{1+\gamma }L^{\frac{1}{1+\gamma} -1}-\theta =-\frac{1}{\lambda }\frac{\gamma}{%
1+\gamma }<0 
\]%
or%
\[
L^{1-\frac{1}{1+\gamma} }\geq \frac{\theta }{1+\gamma }. 
\]

\textit{Continuity}

For any $L$ that satisfy $L^{1-\frac{}{1+\gamma} }\geq \frac{\theta }{1+\gamma} ,$ define function $%
\Psi $ that satisfies $\Psi \left( L^{\frac{1}{1+\gamma} }-\theta L\right) =L.$ Since $%
L^{\frac{1}{1+\gamma} }-L$ is strictly decreasing in $L$ for $L^{1-\frac{1}{1+\gamma} }\geq
\frac{1}{1+\gamma} $, this function is well defined. Note that $\Psi \left(
{}\right) \underbrace{\left( \frac{1}{1+\gamma }L^{\frac{1}{1+\gamma} -1}-\theta \right) }%
_{<0}=1$ (so that $\Psi >0$, i.e. $\Psi $ is strictly decreasing)
and $\Psi ^{\prime \prime }\underbrace{\left( \frac{1}{1+\gamma }L^{\frac{1}{1+\gamma}
-1}-1\right) ^{2}}_{>0}+\underbrace{\Psi }_{<0}\underbrace{\frac{1%
}{1+\gamma }\frac{\gamma }{1+\gamma }L^{\frac{1}{1+\gamma} -2}}_{<0}=0$ (so that $\Psi
^{\prime \prime }\geq 0$, $\Psi ^{\prime \prime }>0,$ i.e. $\Psi $ is
strictly concave on the interior). $\Psi $ is also continuous. When $%
L^{1-\frac{1}{1+\gamma} }=\frac{1}{1+\gamma} ,$ $L=(1+\gamma) ^{-\frac{(1+\gamma)} {\left( \gamma \right)} }.$
Let $D\equiv (1+\gamma) ^{\frac{-1}{ \gamma }-(1+\gamma) ^{-\frac{1+\gamma }{\left(\gamma\right)} }}.$ Then the objective is 
\[
V\left( b\_\right) =\max_{b\left( s\right) }\sum_{s\in S}\pi \left(
s\right) \left[ \Psi \left( \frac{1}{\beta }P\left( s\right) b-b\left(
s\right) +g\left( s\right) \right) +\frac{1}{\beta }P\left( s\right)
b\_-b\left( s\right) +\beta V\left( b\left( s\right) \right) \right] 
\]%
s.t.%
\begin{eqnarray*}
b\left( s\right) &\leq &\bar{b} \\
\frac{1}{\beta }P\left( s\right) b\_-b\left( s\right) +g\left( s\right) &\leq
&D.
\end{eqnarray*}

This function is continuous so $V$ is also continuous.

\textit{Differentiability}

Continuity and convexity implies differentiability everywhere, including the
boundaries.

\textit{Strict concavity}

$\Psi $ is strictly concave, so on the interior $V$ is strictly
concave.

\end{proof}

\smallskip Next we characterize policy functions

\begin{lemma}
\label{lem increasing b}
$b\left( b\_,s\right) $ is an increasing function of $b$ for all $s$ for all $%
\left( b\_,s\right) $ where $b\left( s\right) $ is interior.
\end{lemma}

\begin{proof}
Take the FOCs for $b\left( s\right) $ from the condition in the previous
problem. If $b\left( s\right) $ is interior%
\[
\Psi \left( \frac{1}{\beta }P\left( s\right) b\_-b\left( s\right)
+g\left( s\right) \right) =\beta V\left( b\left( s\right)
\right) . 
\]

Suppose $b_{1}<b_{2}$ but $b_{2}\left( s\right) <b_{1}\left( s\right) .$
Then from stict concavity%
\begin{eqnarray*}
V\left( b_{2}\left( s\right) \right) &<&V^{\prime
}\left( b_{1}\left( s\right) \right) \\
\Psi \left( \frac{1}{\beta }P\left( s\right) b_{2}-b_{2}\left(
s\right) +g\left( s\right) \right) &>&\Psi \left( \frac{1}{\beta }%
P\left( s\right) b_{1}-b_{1}\left( s\right) +g\left( s\right) \right) .
\end{eqnarray*}
\end{proof}

\begin{lemma}
There exists an invariant distribution of the stochastic process $b_{t+1}=b(s_{t+1},b_t)$ 
\end{lemma}
\begin{proof}
The state spaces for $b_t$ and $s_t$ are compact. Further the transition function on $s_{t+1}|s_{t}$ is trivially increasing under i.i.d shocks. We can apply standard arguments as in \cite{Prescott1992}(see corollary 3) to argue that there exists invariant distribution of assets.  
\end{proof}

Now we characterize the support of this distribution using further properties of the policy rules for $b(s|b\_)$


\begin{lemma}
\label{prop: b(s) relative to b}For any $b\_\in (\underline{b} ,\bar{b}),$ there are $s,s^{\prime \prime }$ s.t. $b\left( s\right) \geq b\_\geq b\left( s^{\prime \prime }\right) .$ Moreover, if there are any states $s^{\prime \prime },s^{\prime \prime \prime }$ s.t. $b\left( s^{\prime \prime
}\right) \neq b\left( s^{\prime \prime \prime }\right) ,$ those inequalities
are strict.
\end{lemma}


\begin{proof}
The FOCs together with the envelope theorem imply that $\mathbb{E}P(s)V'(b(s))=V'(b\_)+\kappa(s)$
We can rewrite this as $\mathbb{\tilde{E}}V'(b(s))=b+\kappa(s)$ with $\tilde{\pi}(s)=P(s)\pi(s)$

Now if there is at least one $b\left( s\right) $ s.t. $b\left(
s\right) >b\_,$ by strict concavity of $V$ there must be some $%
s^{\prime \prime }$ s.t. $b\left( s^{\prime \prime }\right) <b.$

If there is at least one $b\left( s\right) $ s.t. $b\left(
s\right) <b\_,$ the inequality above is strictly only if $b\left(
s^{\prime \prime \prime }\right) =\bar{b}$ for some $s^{\prime \prime \prime
}.$ But $V\left( \bar{b}\right) <V\left(
b\right) $ so there must be some $s^{\prime \prime }$ s.t. $b\left(
s^{\prime \prime }\right) >b.$ Equality is possible only if $b\_=b\left(
s\right) $ for all $s.$
\end{proof}


\begin{lemma}  Let $\mu(b,s)$ be the optimal policy function for the Lagrange multiplier $\mu(s)$.  If $P(s') > P(s'')$ then there exists a $b^*_{s',s''}$ such that for all $b < (>) \; b_{1,s',s''}$ we have $\mu(b,s') > (<) \;\mu(b,s'')$.  If $\underline b < b^*_{s',s''} < \overline b$ then $\mu(b^*_{s',s''},s') = \mu(b^*_{s',s''},s'')$.
\label{lem.order}
\end{lemma}
\begin{proof} 
Suppose that $\mu(b,s')\leq \mu(b,s'')$.  Subtracting the implementability for $s''$ from the implementability constraint for $s'$ we have 
\begin{align*}
	\frac{P(s')-P(s'')}{\beta}b &= S_{s'}(\mu(b,s'))-S_{s''}(\mu(b,s'')) + b'(b,s')-b'(b,s'')\\
						&\geq S_{s'}(\mu(b,s')) -S_{s''}(\mu(b,s')) + b'(b,s')-b'(b,s'')\\
						&\geq  S_{s'}(\mu(b,s')) -S_{s''}(\mu(b,s')) = g(s'')-g(s')
\end{align*}  We get the first inequality from noting that $S_s(\mu')\geq S_s(\mu'')$ if $\mu' \leq \mu''$.  We obtain the second inequality by noting that $\mu(b,s')\leq \mu(b,s'')$ implies $b'(b,s')\geq b'(b,s'')$ (which comes directly from the concavity of $V$).
Thus, $\mu(b,s')\leq \mu(b,s'')$ implies that 
\begin{equation}
\label{pair-wise ss}
b \geq \frac{\beta(g(s'')-g(s'))}{P(s')-P(s'')} = b^*_{s',s''} 
\end{equation}

The converse of this statement is that if $b<b^*_{s',s''}$ then $\mu(b,s') > \mu(b,s'')$.  The reverse statement that $\mu(b,s') \geq \mu(b,s'')$ implies $b \leq b^*_{s,s'}$ follows by symmetry.   Again, the converse implies that if $b > b^*_{s',s''}$ then $\mu(b,s') < \mu(b,s'')$.    Finally, if $\underline b < b^*_{s',s''} <\overline b$ then continuity of the policy functions implies that there must exist a root of $\mu(b,s')-\mu(b,s'')$ and that root can only be at $b^*_{s',s''}$.
\end{proof}


\begin{lemma}
\label{lem: existence and uniqueness of ss}
$P \in \mathcal{P}^*$ is necessary and sufficient for existence of $b^*$ such that $b(s,b^*)=b*$ for all $s$s
\end{lemma}

\begin{proof}
The necessary part follows from taking differences of the \eqref{rep agent implementability constraint } for $s'$,$s''$. We have 
\[[P(s)-P(s'')]\frac{b^*}{\beta}=g(s)-g(s'')\]
Thus $P\in\mathcal{P}^*$. The sufficient part follows from the Lemma \ref{lem.order}. If $P \not \in \mathcal{P}^*$, equation \ref{pair-wise ss} that defines $b^*_{s',s''}$ will not be same across all pairs. Thus $b^*$ that satisfies $b(s;b^*)$ independent of $s$ will not exist.
\end{proof}


Lemma  \ref{lem: existence and uniqueness of ss}  implies that under the hypothesis of part 1 of the Theorem \ref{thm: rep agent general theorem} there cannot exist an interior absorbing point for the dynamics of debt. This allows us to construct a sequences $\{b_t\}_t $ such that $b_t<b_{t+1}$ with the property that $\lim_tb_t=\underline{b}$.
Thus, for any $\epsilon>0$, there exists a finite history of shocks that can take us arbitrarily close to $\underline{b}$. Since the shocks are i.i.d this finite sequence will repeat i.o. With a symmetric argument we can show that $b_t$ will come arbitrarily close to its upper limit i.o too





{Part 2 of Theorem \textbf{\ref{thm: rep agent general theorem}}}


In this first section we will show that there exists $b_1$, and if $p$ is sufficiently volatile a $b_2$, such that if $b_t\leq b_1$ then 
\[
	\mu_t \geq \EE_t \mu_{t+1}
\] and if $b_t \geq b_2$ then
\[
	\mu_t \leq \EE_t \mu_{t+1}.
\]  Recall that $b$ is decreasing in $\mu$, so this implies that if $b_t$ is low (large) enough then there will exist a drift away from the lower (upper) limit of government debt.

With Lemma \ref{lem.order} we can order the policy functions $\mu(b,\cdot)$ for particular regions of the state space.  Take $b_1$ to be
\[
	b_1 = \min\left\{b^*_{s',s''}\right\}
\] and WLOG choose $\underline b < b_1$.  For all $b < b_1$ we have shown that $P(s) > P(s')$ implies that $\mu(b,s) > \mu(b,s')$.  The FOC for the problem imply,

\begin{equation}\label{eq.mart}
	\mu_t = \EE_t p_{t+1}\mu_{t+1}+\underline{\kappa}_t
\end{equation}  The inequality in the resource constraint implies that $\xi(s)\geq 0$ implying that $\mu(s) \leq 1$.  With some minor algebra algebra we obtain

By decomposing $\EE \mu_{t+1}p_{t+1}$ in equation \eqref{eq.mart}, we obtain (using $\EE_t p_{t+1} = 1$)
\begin{equation}
	\mu_t = \EE\mu_{t+1} +\cov_t(\mu_{t+1},p_{t+1}) + \kappa_t
\end{equation}Our analysis has just shown that for $b_t < b_1$ we have $\cov_t (\mu_{t+1},p_{t+1})  >0$ so 
\[
	\mu_t > \EE_t\mu_{t+1}.
\]  If $p$ is sufficiently volatile:
\[
	P(s') - P{s''} > \frac{\beta(g_{s''}-g_{s'})}{\overline b}
\] then 
\[
	b_2 = \max\left\{b^*_{s',s''}\right\} <\overline b
\] and through a similar argument  we can conclude that $\cov_t(\mu_{t+1},p_{t+1}) < 0$ 
\[
	\mu_t < \EE_t \mu_{t+1}
\] for $b_t > b_2$ (note $b_t >\underline b$ implies $\kappa_t =0$) which gives us a drift away from the upper-bound. 


{Part 3 of Theorem \textbf{\ref{thm: rep agent general theorem}}}


When $P\in\mathcal{P}^*$, Lemma \ref{lem: existence and uniqueness of ss} implies existence of $b^*$ as the steady state debt level.
\begin{lemma}
\label{sub super martingale}
There exists $\mu^*$ such that $\mu_t$  is a sub-martingale bounded above in the region $(-\infty,\mu^*)$ and super-martingale bounded below in the region $(\mu^*,\frac{1}{1+\gamma})$


\end{lemma}
\begin{proof}
Let $\mu^*$ be the associated multiplier, i.e $V_b(b^*)=\mu^*$.  Using the results of the previous section, we have that $b_1 = b_2 = b^*$, implying that $\mu_t < (>) \EE_t\mu_{t+1}$ for $b_t  < (>) b^*$. 
\end{proof}

Lastly we show that $\lim_t \mu_t=\mu^*$. Suppose $b_{t}<b^*$, we know that $\mu_t>\mu^*$. The previous lemma implies that in this region, $\mu_t$ is a super martingale. The lemma $\ref{lem increasing b}$ shows that $b(b\_,s)$ is continuous and increasing. This translates into $\mu(\mu(b\_),s)$ to be continuous and increasing as well.
 Thus 
 \[\mu_{t}>\mu^* \implies \mu(\mu_{t},s_{t+1})>\mu(\mu^*,s_{t+1}) \]
or
 \[\mu_{t+1}>\mu^*\]
Thus $\mu*$  provides a lower bound to this super martingale. Using standard martingale convergence theorem converges. The uniqueness of steady state implies that it can only converge to $\mu^*$. For $\mu<\mu^*$, the argument is symmetric.


\end{proof}

\newpage

\subsection{Proof of Theorem \protect\ref{thm: rep agent linear policies}}

Working with the first order conditions of problem \ref{eq-QLRA obj}, we obtain
\[
	l(s)^\gamma = \frac{\mu(s)-1}{(1+\gamma)\mu(s) - 1} = 1-\tau(\mu(s)),
\]
implying the relationship between tax rate $\tau$ and multiplier $\mu$ given by

\begin{equation}\label{eq.tau}
	\tau(\mu) = \frac{\gamma\mu}{(1+\gamma)\mu-1}
\end{equation}

The rest of the first order conditions  are summarized below
\begin{align*}
	\frac{b\_ P(s)}{\beta }= S(\mu(s),s) + b(s)\\
	V'(b) = \EE P(s) \mu(s)\\
	\mu(s) = V'(b(s))
\end{align*} 



where $S(\mu,s)$ is the government surplus in state $s$ given by
\[
	S(\mu,s) = (1-\tau(\mu))^\frac1\gamma \tau(\mu)-g(s) = I(\mu) - g(s)
\]


The proof of the theorem will have four steps:

\textbf{Step 1:} Obtaining a recursive representation of the optimal allocation in the incomplete markets economy with payoffs $P(s)$ with state variable $\mu\_$


Given a pair $\{P(s),g(s)\}$, since $V'(b)$ is one-to-one, so we can re-characterize these equations as searching for a function $b(\mu)$ and $\mu(s|\mu\_)$such that the following two equations can be solved for all $\mu\_$.
\begin{align}\label{eq.lin_imp}
	\frac{b(\mu\_)P(s)}{\beta } = I(\mu(s)) - g(s) +b(\mu(s))\\
	\mu = \EE\mu(s)P(s)\label{eq.lin_mart}
\end{align}  



\textbf{Step 2:} Describe how the policy rules are approximated

Usually perturbation approaches  to solve equilibrium conditions as above look for the solutions to $\{\mu(s|\mu\_)\}$ and $b(\mu\_)$ around deterministic steady state of the model. Thus for any $b^{ss}$, there exists a $\mu^{ss}$ that will solve

\[	\frac{b^{SS}}{\beta } = I(\mu^{SS}) - \bar{g} +b^{SS}\\
\]


For example if we set the perturbation parameter $q$ to scale the shocks, $g(s)=\mathbb{E}g(s)+q\Delta_g(s)$ and $P(s)=1+q\Delta_{P}(s)$, the first order expansion of $\mu(s|\mu\_)$ will imply that it is a martingale. Such approximations are not informative about the ergodic distribution. \footnote{One can do higher order approximations, but part 3 of theorem \ref{thm: rep agent general theorem} hints that for economies with payoffs close to $\mathcal{P}^*$, the stochastic steady state in general is far away from $\mu^{SS}$.}



In contrast we will approximate the functions $\mu(s|\mu\_)$ around 
around economy with payoffs in $\bar{P}\in \mathcal{P}^*$. 

In contrast we a) explicitly recognize that policy rules depend on payoffs: $\mu(s|\mu\_,\{P(s)\}_s)$ and $b(\mu\_,\{P(s)\}_s)$ and then take take the first order expansion with respect to both $\mu\_$ and $\{P(s)\}$ around the vector $(\bar{\mu},\{\bar{P}(s)\}_s)$ where $\bar{P}(s) \in \mathcal{P}^*$: these payoffs support  an allocation such that limiting distribution of debt is degenerate around the some value $\bar{b}$; and  $\overline {\mu}$ is the corresponding limiting value of multiplier. The next two expression make the link between $\bar{\mu}$ and $\bar{b}$ explicit. 
\begin{subequations}
\label{eq.ss.} 
\begin{equation}
\overline b = \frac{\beta}{1-\beta}\left( I(\overline\mu) - \overline g\right) 
\end{equation}
where $\overline g = \EE g$ and $\overline p$ as 
\begin{equation}
	\overline P(s) = 1+ \frac\beta{\overline b}(g(s) - \overline g) 
\end{equation}
\end{subequations}

As noted before $b(\overline\mu;\overline p) = \overline b$ solves the the system of equations (\ref{eq.lin_imp}-\ref{eq.lin_mart}) for $\mu'(s) = \overline \mu$ when the payoffs are $\overline{P}(s)$ 

We next obtain the expressions that characterize the linear approximation of $\mu(s|\mu\_,\{P(s)\})$ and $\b(\mu\_,\{P(s)\})$ around some arbitrary point $(\bar{\mu},\{\bar{P}(s)\}_s)$ where $\bar{P}(s) \in \mathcal{P}^*$. We will use these expressions to compute the mean and variance of the ergodic distribution associated with the approximated policy rules. Finally as a last step we propose a particular choice of the point of approximation.

The derivatives $\frac{\delta \mu(s|\mu\_,\{P(s)\}}{\delta \mu\_}$,$\frac{\delta \mu(s|\mu\_,\{P(s)\}}{\delta P(s)}$ and similarly for  $b(\mu\_,\{P(s)\}$ are obtained below:


Differentiating equation \eqref{eq.lin_imp} with respect to $\mu$ around $(\overline \mu,\overline P)$ we obtain
\[
	\frac{\overline{P}(s)}{\beta}\frac{\partial b}{\partial \mu\_} = \left[I'(\mubar)+\frac{\partial b}{\partial \mu\_}\right]\frac{\partial \mu(s)}{\partial \mu\_}.
\]Differentiating equation \eqref{eq.lin_mart} with respect to $\mu\_$ we obtain
\[
	1 = \sum_{s} \pi(s) \overline P(s) \frac{\partial \mu'(s)}{\partial \mu\_}
\]combining these two equations we see that 
\[
	\frac1\beta\left(\sum_s\pi(s)\overline P(s)^2\right)\frac{\partial b}{\partial \mu\_} = I'(\mubar) + \frac{\partial b}{\partial \mu\_}
\]Noting that $\EE\overline P^2(s) = 1 + \frac{\beta^2}{\bbar^2}\sigma^2_g$ we obtain
\begin{equation}
	\frac{\partial b}{\partial \mu\_} = \frac{I'(\mubar)}{\frac{\beta}{\bbar^2}\sigma_g^2 +\frac{1-\beta}{\beta}} < 0
\end{equation}as $I'(\mubar) < 0$.  We then have directly that 
\begin{equation}
	\frac{\partial \mu'(s)}{\partial \mu} = \frac{\overline P(s)}{\frac{\beta^2}{\overline b^2}\sigma^2_g +1} = \frac{\overline P(s)}{\EE\overline P(s)^2}
\end{equation}  We can perform the same procedure for $P(s)$.  Differentiating equation \eqref{eq.lin_imp} with respect to $P(s)$ we around $(\mubar,\pbar)$ we obtain
\begin{equation}\label{eq.dimp_dps}
\frac{\pbar(s')}{\beta}\frac{\partial b}{\partial P(s)} + 1_{s,s'}\frac{\bbar}{\beta} - \frac{\pi(s)\bbar\pbar(s')}{\beta} = \left[I'(\mubar) + \frac{\partial b}{\partial \mu}\right]\frac{\partial \mu(s')}{\partial P(s)}
\end{equation} Here $1_{s,s'}$ is $1$ if $s = s'$ and zero otherwise.  Differentiating equation \eqref{eq.lin_mart} with respect to $P(s)$ we obtain
\[
	0 = \pi(s)\overline \mu - \pi(s)\overline \mu + \sum_{s'} \pi(s) \pbar(s')\frac{\partial \mu(s')}{\partial P(s)} =  \sum_{s'} \pi(s')\pbar(s')\frac{\partial \mu(s')}{\partial P(s)}
\]  Again we can combine these two equations to give us
\[
	\frac{\EE\pbar(s)^2}{\beta}\frac{\partial b}{\partial P(s)} + \frac{\pi(s)\bbar}{\beta}(\pbar(s)-\EE\pbar(s)^2) = 0
\] or
\begin{equation}
	\frac{\partial b}{\partial P(s)} = \pi(s)\bbar \frac{\EE\pbar^2-\pbar(s)}{\EE\pbar^2}
\end{equation}Going back to equation \eqref{eq.dimp_dps} we have
\begin{equation}
	\frac{\partial \mu(s')}{\partial P(s)} = \frac{\bbar}{\beta\left[I '(\mubar) + \frac{\partial b}{\partial \mu}\right]}\left(1_{s,s'}-\frac{\pi(s)\pbar(s)\pbar(s')}{\EE\pbar^2}\right)
\end{equation}


\textbf{Step 3:} Getting expressions for the mean and variance of the ergodic distribution around some arbitrary point of approximation

For an arbitrary $\left (\overline \mu,\{\overline {P}(s )\}_{s}\right)$, using the derivatives that we computed, we can characterize the dynamics of $\hat{\mu}\equiv\mu_t-\mubar$ using our approximated policies.


\[
	\hat \mu_{t+1} = B \hat\mu_t + C,
\]  where $B(s)$ and $C(s)$ are respective derivatives. Note that both are random variables and let us denote their means $\barB$ and $\barC$, and variances $\sigma_B^2$ and $\sigma_C^2$ .  Suppose that $\hat\mu$ is distributed according to the ergodic distribution of this linear system with mean $\EE\hat\mu$ and variance $\sigma^2_\mu$.  Since 
\[
	B\hat\mu +C,
\]has the same distribution we can compute the mean of this distribution as
\[
\begin{split}
	\EE\hat\mu &= \EE\left[ B\hat\mu+C\right]\\
			  &= \EE\left[\EE_{\hat\mu}\left[B\hat\mu+C\right]\right]\\
			  &= \EE\left[\barB\hat\mu +\barC\right]\\
			  &=\barB\EE\hat\mu+\barC
\end{split}
\]solving for $\EE\hat\mu$ we get
\begin{equation}
	\EE\hat\mu = \frac{\barC}{1-\barB}
\end{equation}For the variance $\sigma^2_{\hat\mu}$ we know that 
\[
	\sigma^2_{\hat\mu} = \var(B\hat\mu+C) = \var(B\hat\mu) + \sigma_C^2 + 2\cov(B\hat\mu,C)
\]Computing the variance of $B\hat \mu$ we have
\[
\begin{split}
	\var(B\hat\mu) &=\EE\left[(B\hat\mu - \barB\EE\hat\mu)^2\right]\\
			       &=\EE\left[(B\hat\mu-\barB\hat\mu +\barB\hat\mu -\barB\EE\hat\mu)^2\right]\\
			      &=\EE\left[\EE_{\hat\mu}\left[(B-\barB)^2\hat\mu^2 +2(B-\barB)(\hat\mu-\EE\hat\mu)\barB\EE\hat\mu + (\hat\mu-\EE\hat\mu)^2\bar B^2\right]\right]\\
			&=\EE\left[\sigma_B^2\hat\mu^2 +(\hat\mu-\EE\hat\mu)^2\barB\right]\\
			& = \sigma_B^2(\sigma_{\hat\mu}^2+(\EE\hat\mu)^2) + \sigma_{\hat\mu}^2\barB^2
\end{split}
\]while for the covariance of $B\hat\mu$ and $C$
\[
	\cov(B\hat\mu,C) = \sigma_{BC}\EE\hat\mu
\]Putting this all together we have
\begin{equation}
	\sigma_{\hat\mu}^2 = \frac{\sigma_B^2(\EE\hat\mu)^2 + \sigma_{BC}\EE\hat\mu + \sigma_C^2}{1-\barB^2-\sigma_B^2}
\end{equation}


\textbf{Step 4:} Choice of the point of approximation

To get the expressions in Theorem \ref{thm: rep agent general theorem}, we finally choose a particular $\overline{P}=P^*(s) \in \mathcal{P}^*$. This will be the closest complete market economy to our the given $P(s)$ in $L^2$ sense. Formally,

\[
\min_{\tilde{P}\in \mathcal{P}^*} \sum_{s}\pi(s)( P(s)-\tilde{P}(s))^2.
\]
Since all payoffs in $\mathcal{P}^*$ are associated with some $b^*$ and $\mu^*$ via equations \eqref{eq.ss.},  we can re write the above problem as choosing $\mubar$ so as to minimize the variance of the difference between $ P(s)$ and the set of steady state payoffs.  Let $P^*$ be the solution to this minimization problem. The first order condition for this linearization gives us 
\[
	2\sum_{s'}\pi( P(s')-P^*(s',\mu^*)) \frac{\delta P^*(s,\mu^*)}{\delta \mu^*} = 0
\]as noted before 
\[
	P^*(s) =  1 -\frac{\beta}{b^*(\mu^*)}\left(g(s) - \EE g\right)
\]thus
\[
\frac{\delta P^*}{\delta \mu^*}\propto P^*-1
\]Thus we can see the the optimal choice of $\mubar$ is equivalent to choosing $\mubar$ such that 
\begin{equation}
	\begin{split}
		0 &= \sum_{s'}\Pi_{s'}( P(s') - P^*(s',\mu^*))(P^*(s',\mu^*)-1)\\
		&= -\sum_{s'}\Pi_{s'}( P(s')-P^*(s',\mu^*)) + \sum_{s'}\Pi_{s'}( P(s')-P^*(s',\mu^*))P^*(s',\mu^*)\\
		&= \sum_{s'}\Pi_{s'}( P(s')-P^*(s',\mu^*))P^*(s',\mu^*)\\
		&=\EE\left[( P-P^*)P^*\right]
	\end{split}
\end{equation}  At these values of $\pbar=P^*$ and $\mubar=\mu^*$ we have that $C$ for our linearized system is

\[
	C(s') = \sum_s\left\{\frac{{b^{*}}}{\beta\left[I'(\mubar)+\frac{\partial b}{\partial\mu}\right]}\left(1_{s,s'}-\frac{\pi(s) P^*(s)P^*(s')}{\EE\pbar^2}\right)(P(s)-P^*(s)) \right\}
\]Taking expectations we have that 
\begin{equation}
\begin{split}
	\barC &= \sum_s\left\{\frac{{b^{*}}}{\beta\left[I'(\mubar)+\frac{\partial b}{\partial\mu}\right]}\left(\pi(s) - \frac{\pi(s)P^*(s)}{\EE\pbar^2}\right)( P(s)-P^*(s))\right\}\\
	&=\frac{{b^{*}}}{\beta\left[I'(\mubar)+\frac{\partial b}{\partial\mu}\right]}\left(\EE( P-\pbar) -\frac{\EE\left[( P-\pbar)\pbar\right]}{\EE\pbar^2}\right)\\
	&= 0 
\end{split}
\end{equation}  Thus the linearized system will have the same mean for $\mu$, $\mubar$, as the closest approximating steady state payoff structure.

We can also compute the variance of the ergodic distribution for $\mu$.  Note 
\[
\begin{split}
	C(s') &= \sum_s\left\{\frac{{b^{*}}}{\beta\left[I'(\mubar)+\frac{\partial b}{\partial\mu}\right]}\left(1_{s,s'}-\frac{\pi(s) P^*(s)P^*(s')}{\EE{P^*}^2}\right)(P(s)-P^*(s)) \right\}\\
		 &=\frac{{b^{*}}}{\beta\left[I'(\mubar)+\frac{\partial b}{\partial\mu}\right]}\left( P(s')-P^*(s') -P^*(s')\frac{\sum_s\pi(s)P^*(s)( p_s-P^*(s))}{\EE{P^*}^2}\right)\\
		&= \frac{{b^{*}}}{\beta\left[I'(\mubar)+\frac{\partial b}{\partial\mu}\right]}( P(s')-P^*(s))
\end{split}
\]  As noted before
\[
	\sigma_{\mu}^2 = \frac{{b^{*}}^ 2}{\beta^2\left[I'(\mubar)+\frac{\partial b}{\partial\mu}\right]^2\left(1-\barB^2-\sigma_B^2\right)}\| P-{P^*}\|^2
\]  The variance of government debt in the linearized system is 
\[
	\sigma_b^2 = \frac{{b^{*}}^2\left(\frac{\partial b}{\partial\mu}\right)^2}{\beta^2\left[I'(\mubar)+\frac{\partial b}{\partial\mu}\right]^2\left(1-\barB^2-\sigma_B^2\right)}\| P-{P^*}\|^2
\]  This can be simplified using the following expressions: 
\[
	I'(\mubar)+\frac{\partial b}{\partial \mu} = \frac{\EE{P^*}^2}{\beta}\frac{\partial b}{\partial\mu},
\]
\[
	\barB = \frac{1}{\EE{P^*}^2}
\]and
\[
	\sigma_B^2 = \frac{\var({P^*})}{(\EE{P^*}^2)^2}
\] to
\begin{equation}
	\sigma^2_b = \frac{{b^{*}}^2}{\EE{P^*}^2\var({P^*})}\|P-{P^*}\|^2
\end{equation}Noting that $\EE{P^*}^2 = 1 +\var({P^*}) > 1$, we have immediately that up to first order the relative spread of debt is bounded by
\begin{equation}
	\frac{\sigma_b}{{b^{*}}} \leq \sqrt\frac{\|P-{P^*}\|^2}{\var({P^*})}
\end{equation}  


\newpage





\newpage

\subsection{Proof of Theorem \protect\ref{thm heterogeneous agents}}
\begin{proof}
 
Using Theorem \ref{theorem: main} let $\tilde{b}=b_1-b_2$. Under the normalization that $b_2=0$, the variable $\tilde{b}$ represents public debt government or the assets of the productive agent. Specializing the formulations in section \ref{} we have the optimal plan solves the following Bellman equation.

\begin{equation}
	\label{eq-2 agent QL obj}
   	V(\tilde{b}\_)=\max_{c_1(s),c_2(s),b'(s)} \sum_{s}\pi(s)\left\{\omega\left[u(c_1(s),l_1(s))\right]+(1-\omega)\left[c_2(s)\right]+\beta V(\tilde{b}(s)) \right\}
\end{equation}   
subject to

   \begin{subequations}
   \label{sys-2 agent QL constraint}
   	\begin{equation}
   	\label{eq-implementability constraint}
   	c_1(s)-c_2(s)+\tilde{b}(s)=l(s)^{1+\gamma}+\beta^{-1} P(s)\tilde{b}\_
   	\end{equation}

 
\begin{equation}
	\label{eq-resoruces}
   	n c_1(s)+(1-n)c_2(s)+g(s)\leq\theta_2 l(s)n
\end{equation}   


\begin{equation}
	\label{eq-non negativity of consumption}
   	c_2(s)\geq0
\end{equation}   

\begin{equation}
	\label{eq-debt limits}
   	\overline{b}\geq\tilde{b}(s)\geq \underline{b}
\end{equation}   
   \end{subequations}

Let $\mu(s),\phi(s),\lambda(s),\underline{\kappa}(s),\overline \kappa(s) $ be the Lagrange multipliers on the respective constraints. The FOC are summarized below

\begin{subequations}
\begin{equation}
\label{eq.het.agent.foc.c1}
\omega-\mu(s) =n \phi(s) 
\end{equation}

\begin{equation}
\label{eq.het.agent.foc.c2}
1-\omega+\mu(s)-\phi(s)(1-n)+\lambda(s)=0
\end{equation}


\begin{equation}
\label{eq.het.agent.foc.l}
-\omega l^\gamma(s)+\mu(s)(1+\gamma)l^{\gamma}(s)+n\phi(s)\theta=0
\end{equation}

\begin{equation}
\label{eq.het.agent.foc.tilde_b}
\beta V'(\tilde{b}(s))-\mu(s)-\overline \kappa(s)+\underline \kappa(s)=0
\end{equation}
and the envelope condition

\begin{equation}
\label{eq.het.agent.foc.envelope}
V'(\tilde{b}\_)=\sum_{s}\pi(s)\mu(s)\beta^{-1}P(s)
\end{equation}

\end{subequations}


To show part 1 of Theorem \ref{thm heterogeneous agents}, we show that $\frac{\omega}{n}>\frac{1+\gamma}{\gamma}$ is sufficient for the Lagrange multiplier $\lambda(s)$ on the non-negativity constraint to bind. 

\begin{lemma} The multiplier on the budget constraint $\mu(s)$ is bounded above
\[\mu(s)\leq \min \left\{\omega-n,\frac{\omega}{1+\gamma}\right\}\]
Similiarly the multiplier of the resource constraint is bounded below,
\[\phi(s)\geq \max \left\{1,\frac{\omega}{n}\left[\frac{\gamma}{1+\gamma}\right]\right\}  \]
\end{lemma}
\begin{proof}

Notice that the labor choice of the productive household implies $\frac{1}{1-\tau}=\frac{\theta_2}{l^{\gamma}(s)}$. 

As taxes go to $-\infty$ \eqref{eq.het.agent.foc.l} implies that $\mu(s)$ approaches $\frac{\omega}{1+\gamma}$ from below. Similiarly the non-negativity of $c_2(s)$ imposes a lower bound of $1$ on $\phi(s)$. This translates into an upper bound of $\omega-n$ on $\mu$. 
\end{proof}



\begin{lemma} 
There exists a $\bar{\omega}$ such that $\omega>\bar{\omega}$ implies $c_2(s)=0$ for all $b$
\end{lemma}
\begin{proof}
 
By the KKT conditions $c_2(s)=0$ if $\lambda(s)>0$. Now \eqref{eq.het.agent.foc.c2} implies this is true if $\mu(s)<\omega-n$.  The previous lemma bounds $\mu(s)$ by $\frac{\omega}{1+\gamma}$. 

We can thus define $\bar{\omega} = n \left(\frac{1+\gamma}{\gamma}\right)$ as the required threshold Pareto weight to ensure that the unproductive agent has zero consumption forever.

\end{proof}




Now for the rest of the parts $\omega<n\frac{1+\gamma}{\gamma}$, we can have postive transfers for low enough public debt. In particular, we can define a maximum level of debt $\mathcal{B}$ that is consistent with an interior solution for the unproductive agents' consumption.

Guess an interior solution $c_{2,t}>0$ or $\lambda_t=0$ for all $t$. This gives us $l(s)=l^*$ defined below:

\begin{equation}
\label{eq.opt.int.labor}
l^*=\left[\frac{n\theta}{\omega-(\omega-n)(1+\gamma)}\right]^{\frac{1}{\gamma}}
\end{equation}
As long as $\omega<n \left(\frac{1+\gamma}{\gamma}\right)$
At the interior solution $\tilde{b}(s)=\tilde b\_$ and using the implementability  constraint and resource constraints \eqref{eq-implementability constraint} and \eqref{eq-resoruces} respectively, we can obtain the expression for $c_{2}(s)$

\[c_{2}(s)=n \theta l^{*}-n {l^*}^{1+\gamma}-\tilde{b}\_(1-P(s)\beta^{-1})-g(s)\]

Non-negativity of $c_2$ implies,


\[\tilde{b}\_\leq \frac{g(s)-n\theta l^*+n{l^*}^{1+\gamma}}{\beta^{-1}P(s)-1}\]


We can also express this as 
\[\tilde{b}\_\leq \frac{g(s)-\tau^*y^* }{\beta^{-1}P(s)-1},\]


where the right hand side of the previous equation is just the present discounted value of the primary deficit of the government at the constant taxes $\tau^*$ associated with $l^*$ defined in \eqref{eq.opt.int.labor}.
As long as $\beta^{-1}P(s)-1>0$, this object is well defined, we define $\mathcal{B}=\min_{s}\left[\frac{g(s)-n\theta l^*+n{l^*}^{1+\gamma}}{\beta^{-1}P(s)-1}\right]$.  Thus for $\tilde{b}\_<\mathcal{B}$ the optimal allocation has constant taxes given by $\tau^*$ and debt $\tilde{b}\_$, while  transfers are given by 

\[T(s)=n \theta l^{*}-n {l^*}^{1+\gamma}-\tilde{b}\_(1-P(s)\beta^{-1})-g(s),\]

and are strictly positive.


For initial debt greater than $\mathcal{B}$, we distinguish cases when payoffs are perfectly aligned with $g(s)$ i.e belong to the set $\mathcal{P}^*$ and when they are not. For part 2 case b, let $P\not \in \mathcal{P}^*$. 

\begin{lemma}
\label{lem.limit.in.one.step}
There exists a $\check b>\mathcal{B}$ such that there are two shocks $\underline{s}$ and $\overline{s}$ and the optimal choice of debt starting from $\tilde{b}\_\leq \check b$ satisfies the following two inequalities:
\[\tilde{b}(\underline s,\tilde{b}\_)>\mathcal{B}\]
\[\tilde{b}(\overline s,\tilde{b}\_)\leq \mathcal{B}\]
\end{lemma}

\begin{proof}
At $\mathcal{B}$, there exist some $\overline s$ such that $T(\overline{s},\mathcal{B})=\epsilon>0$. Now define $\check b$ as follows:

\[\check b =\mathcal{B}	+ \frac{\epsilon \beta }{2 P(\overline{s})}\]

Now suppose to the contrary $\tilde{b}(\overline s, \tilde{b}\_)>\mathcal{B}$ for some $\tilde{b}\_\leq \check b$. This implies that $\tau(s,\tilde b\_)>\tau^*$ and $T(\overline{s}, \tilde b\_)=0$.\footnote{Explain why}

The government budget constraint implies


\[\frac{P(\overline s)\tilde b\_}{\beta}+g(s)=\tilde{b}(\overline s,\tilde{b}\_)+(1-\tau(\overline s, \tilde{b})\_)l(\overline s, \tilde{b}\_).\]


As,
\[\frac{P(\overline s)\tilde{b}\_}{\beta}+g(\overline s)\leq \frac{P(\overline s)\mathcal{B}}{\beta}+g(\overline s)+\frac{\epsilon}{2}< \frac{P(\overline s)\mathcal{B}}{\beta}+g(\overline s)+\epsilon\]

This further implies,
\[\tilde b(\overline s, \tilde b\_)+(1-\tau(\overline s, \tilde b\_))l(\tau(\overline s, \tilde b\_))>	[\tilde b(\overline s, \tilde b\_)+(1-\tau^*)l^*>\mathcal{B}+(1-\tau^*)l^*>\frac{P(\overline s)\tilde{b}\_}{\beta}+g(\overline s)+T(\overline s,\tilde b\_)=\frac{P(\overline s)\tilde{b}\_}{\beta}+g(\overline s)+\epsilon.\]

Combining the previous two inequalities yields a contradiction. The second inequality, $\tilde{b}(\underline s,\tilde{b}\_)>\mathcal{B}$ follows from the definition of $\mathcal{B}$. 


Now define $\overline \mu (\tilde b(s,\tilde{b}\_))$ as $\max_{s} \mu (s,\tilde{b}\_)$ and $\hat{s}(\tilde{b}\_)$ as the shock that achieves this maximum. Now we show that $\hat \mu(\tilde b(s,\tilde{b}\_))$ is finite for all $b\_\leq\overline{b}$. We show the claim for the natural debt limit.


Let $b^{n}(s)=(\beta ^{-1}P(s)-1)^{-1}\left[\theta^{\frac{\gamma}{1+\gamma}}\left(\frac{1}{1+\gamma} \right)^{\frac{1}{\gamma}}\left(\frac{\gamma}{1+\gamma}\right)-g(s)\right]$ be the maximum debt supported by a particular shock $s$. The natural debt limit is defined as $\overline b^{n}=\min_{s}b^{n}(s)$. Note that $\lim_{b\to \overline b^{n}}\mu(\tilde{b}\_)=\infty$

Now choose $s$ such that $b^{n}(s)>\overline b^{n}$ and consider the debt choice next period for the same shock $s$ when it comes in with debt $\overline b^{n}$. 


Suppose it chooses $\tilde{b}(s,\overline {b}^{n})=\overline b^{n}$, then taxes will have to be set to $\frac{\gamma}{1+\gamma}$ and the tax income will be $\frac{\gamma}{1+\gamma}l(\frac{\gamma}{1+\gamma})=\theta^{\frac{\gamma}{1+\gamma}}\left(\frac{1}{1+\gamma}\right)^{\frac{1}{\gamma}}\left(\frac{\gamma}{1+\gamma} \right)$. The budget constraint will then imply that,
\[\frac{\overline  b^{n} P(s)}{\beta}+g(s)=\theta^{\frac{\gamma}{1+\gamma}}\left(\frac{1}{1+\gamma}\right)^{\frac{1}{\gamma}}\left(\frac{\gamma}{1+\gamma} \right)+\overline b^{n}\]
\[\overline b^{n}=(P(s)\beta^{-1}-1)^{-1}\left(\theta^{\frac{\gamma}{1+\gamma}}\left(\frac{1}{1+\gamma}\right)^{\frac{1}{\gamma}}\left(\frac{\gamma}{1+\gamma}\right) -g(s)\right) \]

However the  right hand side is the definition of $b^{n}(s)$ and,

\[b^{n}(s)>\overline b^{n}.\]

Thus we have a contradiction and the optimal choice of debt at the natural debt limit $\tilde{b}(s,\overline {b}^{n})<\overline b^{n}$. 

This inturn means that $\lim_{\tilde b\to \overline b^{n}}\overline \mu (\tilde b) <\infty$. 


Now note that $\overline \mu(\tilde b\_)-\mu(\tilde b\_)$ is continuous on $[\check b, \quad \overline b^{n}]$ and is bounded below by zero, therefore attains a minimum at $\tilde b^{min}$. Let $\delta=\hat \mu(\tilde b^{min})-\mu(\tilde b^{min})>\eta>0$. If this was not true then $P(s)\in \mathcal {P}^*$ as $\mu$ will have an absorbing state. 


Let $\mu(\omega,n)=\omega-n$. This is the value of $\mu$ when debt falls below $\mathcal{B}$. 

Now consider any initial $\tilde b\_ \in [\mathcal{B},\overline b^{n}]$. If $\tilde{b}\_\leq \check b$, then by lemma \ref{lem.limit.in.one.step}, we know that $\mathcal{B}$ will be reached in one shock. Otherwise if $\tilde b\_>\check b$, we can construct a sequence of shocks $s_t=\hat s (\tilde b_{t-1})$ of length $N=\frac{\mu(\omega,n)-\mu(\tilde b\_)}{\delta}$. There exits $t<N$ such that $\tilde b_t<\check b$, otherwise,
\[\mu_t>\mu(\tilde b\_)+N\delta>\mu(\omega,n)\]

Thus we can reach $\mathcal{B}$ in finite steps. Since shocks are i.i.d, this is an almost sure statement. At $\mathcal{B}$, transfers are strictly positive for some shocks $T_t>0$ a.s. and taxes are given by $\tau^*$.




Now consider the payoffs $P\in \mathcal {P}^*$ such that the associated steady state debt $b^*>\mathcal{B}$. Under the guess $T_t=0$, the same algebra as in Theorem \ref{thm: rep agent general theorem} goes through and we can show that $\tilde b\_=b^*$ is a steady state for the heterogeneous agent economy. Thus the heterogeneous agent economy for a given $P\in \mathcal{P}^*$ has a continuum of steady states given by the set $[\overline b, \quad \mathcal{B}] \cup \{b^*\}$.



In the region $\tilde b\_>b^*$, as before $\mu_t$ is supermartingale bounded below by $b^*$. Since there is a unique fixed point in the region $\tilde b\_ \in [b^*, \overline b^{n}]$, $\mu_t$ converges to $\mu^*$ associated with $b^*$. Transfers are zero and taxes are given by $\tau^{**}$


\begin{equation}
\tau^{**} = \frac{\gamma\mu^*}{(1+\gamma)\mu^*-1}
\end{equation}



In the region $[\mathcal{B},\quad b^*]$ the outcomes depend on the exact sequence of shocks we can show that $\mu_t$ is a submartingale. This follows from the observation that for all $\tilde b\_\geq \mathcal{B}$, we have $T(s)=0$ and the outcomes from the representative agent economy allow us to order $\mu(s)$ relative $P(s)$. At $\tilde{b}\_=\mathcal{B}$, $\mu(s)=\omega-n$ and is constant. Thus in the region $[\mathcal{B},\quad B^*]$, $\mu_t$ is sub martingale and it converges. However f $\tilde b_t$ gets sufficiently close to $\check b$, then it can converge to $\mathcal{B}$ and if it gets sufficiently close to $b^*$, it can converge to $b^*$. Either of this can happen with strictly positive probability.
\end{proof}


\newpage

\subsection{Proof of Theorem \protect\ref{prop: long run forces}}

The Bellman equation for the optimal planners problem with log quadratic
preferences and IID shocks can be written as
\begin{equation*}
V(x,\rho) = \max_{c_1,c_2,l_1,x^{\prime },\rho^{\prime }} \sum_s \pi(s)\left[%
\alpha_1\left(\log c_1(s) -\frac{l_1(s)^2}{2}\right)+\alpha_2\log
c_2(s)+\beta V(x^{\prime }(s),\rho^{\prime }(s))\right]
\end{equation*}%
subject to the constraints
\begin{align}
1+\rho^{\prime }(s)[l_1(s)^2-1]+\beta x^{\prime }(s) - \frac{x\frac{P(s)}{%
c_2(s)}}{\mathbb{E}[\frac{P(s)}{c_2(s)}]}=0  \label{eq.conStart} \\
\mathbb{E}\frac{P(s)}{c_1(s)}(\rho^{\prime }(s)-\rho) = 0 \\
\theta_1(s)l_1(s) -c_1(s)-c_2(s)-g = 0 \\
\rho^{\prime }(s) c_2(s)-c_1(s) = 0  \label{eq.conEnd}
\end{align}
where the $\pi(s)$ is the probability distribution of the aggregate state $s$. If
we let $\pi(s)\mu(s)$, $\lambda$, $\pi(s)\xi(s)$ and $\pi(s)\phi(s)$ be the
Lagrange multipliers for the constraints (\ref{eq.conStart})-(\ref{eq.conEnd}%
) respectively then we obtain the following FONC for the planners problem \footnote{Appendix \ref{apndx: numerical methods} discuses the associated second order conditions that ensure these policies are optimal}

\begin{enumerate}
\item[$c_1(s):$]
\begin{equation}
\frac{\alpha_1\pi(s)}{c_1(s)}-\frac{\lambda \pi(s)}{c_1(s)^2}(\rho^{\prime
}(s)-\rho)-\pi(s)\xi(s)-\pi(s)\phi(s) = 0  \label{eq.c1FOC}
\end{equation}

\item[$c_2(s):$]
\begin{equation}
\frac{\alpha_2 \pi(s)}{c_2(s)} + \frac{x P(s)\pi(s)}{c_2(s)^2\mathbb{E}[\frac{P}{c_2}]}%
\left[\mu(s)-\frac{\mathbb{E}[\mu\frac{P}{c_2}]}{\mathbb{E}[\frac{P}{c_2}]}%
\right]-\pi(s)\xi(s)+\pi(s)\rho^{\prime }(s)\phi(s)=0  \label{eq.c2FOC}
\end{equation}

\item[$l_1(s):$]
\begin{equation}
-\alpha_1\pi(s)l_1(s)+2\mu(s)\pi(s)\rho^{\prime
}(s)l_1(s)+\theta_1(s)\pi(s)\xi(s)=0
\end{equation}

\item[$x^{\prime }(s):$]
\begin{equation}
V_x(x^{\prime }(s),\rho^{\prime }(s)) + \mu(s) = 0
\label{eq.x'FOC}
\end{equation}

\item[$\rho^{\prime }(s):$]
\begin{equation}
 \beta V_{\rho}(x^{\prime }(s),\rho^{\prime }(s))+\frac{\lambda \pi(s)}{%
c_1(s)}+\mu(s)[l_1(s)^2-1]+\pi(s)\phi(s)c_2(s) = 0  \label{eq.FOCNend}
\end{equation}
\end{enumerate}

In addition there are two envelope conditions given by
\begin{align}
V_x(x,\rho) = -\sum_{s^{\prime }}\frac{\mu(s^{\prime })\pi(s^{\prime
})\frac{P(s)}{c_2(s^{\prime })}}{\mathbb{E}[\frac{P}{c_2}]} = -\frac{\mathbb{E}%
[\mu\frac{P}{c_2}]}{\mathbb{E}[\frac{P}{c_2}]} \\
V_{\rho}(x,\rho) = -\lambda\mathbb{E}[\frac{P}{c_1}]  \label{eq.rho_env}
\end{align}

In the steady state, we need to solve for a collection of allocations, initial conditions and Lagrange multipliers $%
\{c_1(s),c_2(s),l_1(s),x,\rho,\mu(s),\lambda,\xi(s),\phi(s)\}$ such that
equations (\ref{eq.conStart})-(\ref{eq.rho_env}) are satisfied when $%
\rho^{\prime }(s) = \rho$ and $x^{\prime }(s) = x$. It should be clear that if we replace $\mu(s) = \mu$, equation (\ref{eq.x'FOC})  and the envelope condition with respect to $x$ is
always satisfied. Additionally under this assumption equation (\ref{eq.c2FOC}%
) simplifies significantly,since
\begin{align*}
\frac{xP(s)\pi(s)}{c_2(s)^2\mathbb{E}[\frac{P}{c_2}]}\left[\mu(s)-\frac{\mathbb{E}%
[\mu\frac{P}{c_2}]}{\mathbb{E}[\frac{P}{c_2}]}\right] = 0
\end{align*}
The first order conditions for a
steady can then be written simply as
\begin{align}
1+\rho[l_1(s)^2-1]+\beta x-\frac{x P(s)}{ c_2(s)\mathbb{E}[\frac{P}{c_2}]} = 0
\label{eq.imp} \\
\theta_1(s) l_1(s) - c_1(s)-c_2(s)-g=0  \label{eq.res} \\
\ \rho c_2(s)-c_1(s) = 0  \label{eq.rhoFOC} \\
\frac{\alpha_1}{c_1(s)}-\xi(s)-\phi(s) = 0  \label{eq.c1} \\
\frac{\alpha_2}{c_2(s)}-\xi(s)+\rho\phi(s) = 0  \label{eq.c2} \\
[2\mu \rho-\alpha_1]l_1(s)+\theta_1(s)\xi(s) = 0  \label{eq.l1} \\
\lambda\left(\frac{P(s)}{c_1(s)}-\beta\mathbb{E}\left[\frac{P}{c_1}\right]\right)+\mu[%
l_1(s)^2-1]+\phi(s)c_2(s) = 0  \label{eq.R}
\end{align}
We can rewrite equation (\ref{eq.c1}) as
\begin{equation*}
\frac{\alpha_1}{c_2(s)} - \rho\xi(s) -\rho\phi(s) = 0
\end{equation*}%
by substituting $c_1(s) = \rho c_2(s)$. Adding this to equation (\ref{eq.c2}%
) and normalizing $\alpha_1+\alpha_2 = 1$ we obtain
\begin{equation}
\xi(s) = \frac1{\left(1+\rho\right)c_2(s)}\label{eq.xi}
\end{equation}%
which we can use to solve for $\phi(s)$ as
\begin{equation}
\phi(s) = \frac{\alpha_1-\rho\alpha_2}{\left(\rho(1+\rho)\right)c_2(s)}
\label{eq.phi}
\end{equation}
From equation (\ref{eq.imp}) we can solve for $l_1(s)^2 -1$ as
\begin{equation*}
l_1(s)^2-1 = \frac{x}{\rho\mathbb{E}[\frac{P}{c_2}]}\left(\frac
{P(s)}{c_2(s})-\beta\mathbb{E}\left[\frac{P}{c_2}\right]\right)-\frac1\rho
\end{equation*}%
This can be used along with equations (\ref{eq.R}) and (\ref{eq.phi}) to
obtain
\begin{equation*}
\left(\frac\lambda \rho+\frac{\mu x}{\rho\mathbb{E}[\frac{P}{c_2}]}%
\right)\left(\frac{P(s)}{c_2(s)}-\beta\mathbb{E}\left[\frac{P}{c_2}\right]%
\right) = \frac{\mu}{\rho}+\frac{\rho\alpha_2-\alpha_1}{\rho(1+\rho)}
\end{equation*}
Note that the LHS depends on $s$ while the RHS does not, hence the solution to
this equation is
\begin{equation}
\lambda = - \frac{\mu x}{\mathbb{E}[\frac{P}{c_2}]}
\end{equation}%
and
\begin{equation}
\mu = \frac{\alpha_1-\rho\alpha_2}{1+\rho}  \label{eq.mu}
\end{equation}
Combining these with equation (\ref{eq.l1}) we quickly obtain that
\begin{equation*}
\left[2\rho\frac{\alpha_1-\rho\alpha_2}{1+\rho}-\alpha_1\right]%
l_1(s)+\frac{\theta_1(s)}{\left(1+\rho\right)c_2(s)} = 0
\end{equation*}%
Then solving for $l_1(s)$ gives
\begin{equation*}
l_1(s) = \frac{\theta_1(s)}{\left(\alpha_1(1-\rho)+2\rho^2\alpha_2\right)c_2(s)}
\end{equation*}

\begin{remark}
Note that the labor tax rate is given by $1-\frac{c_1(s)l_1(s)}{\theta(s)}$. The previous expression shows that labor taxes are constant at the steady state. This property holds generally for CES preferences separable in consumption and leisure
\end{remark}

This we can plug into the aggregate resource constraint (\ref{eq.res}) to
obtain
\begin{equation*}
l_1(s) = \left(\frac{1+\rho}{\alpha_1(1-\rho)+2\rho^2\alpha_2}\right)\frac{1}{l_1(s)} +
\frac{g}{\theta_1(s)}
\end{equation*}%
letting $C(\rho) = \frac{1+\rho}{\alpha_1(1-\rho)+2\rho^2\alpha_2}$ we can
then solve for $l_1(s)$ as
\begin{equation*}
l_1(s) = \frac{g\pm\sqrt{g^2+4C(\rho)\theta_1(s)^2}}{2\theta_1(s)}
\end{equation*}
The marginal utility of agent 2 is then
\begin{equation*}
\frac1{c_2(s)} = \left(\frac{1+\rho}{C(\rho)}\right)\left(\frac{g\pm\sqrt{g^2+4C(\rho)%
\theta_1(s)^2}}{2\theta_1(s)^2}\right)
\end{equation*}%
Note that in order for either of these terms to be positive we need $%
C(\rho)\geq 0$ implying that there is only one economically meaningful root.
Thus
\begin{equation}
l_1(s) = \frac{g+\sqrt{g^2+4C(\rho)\theta_1(s)^2}}{2\theta_1(s)}
\end{equation}
and
\begin{equation}
\frac1{c_2(s)} = \left(\frac{1+\rho}{C(\rho)}\right)\left(\frac{g+\sqrt{g^2+4C(\rho)%
\theta_1(s)^2}}{2\theta_1(s)^2} \right) \label{eq.uc2}
\end{equation}
A steady state is then a value of $\rho$ such that
\begin{equation}
x(s) = \frac{1+\rho[l_1(\rho,s)^2-1]}{\frac{P(s)/c_2(\rho,s)}{\mathbb{E}%
[\frac{P}{c_2}](\rho)}-\beta}  \label{eq.xSS}
\end{equation}%
s independent of $s$.

The following lemma, which orders consumption and labor across states, will
be useful in proving the parts of proposition \ref{prop: long run forces}.  As a notational aside we
will often use $\theta_{1,l}$ and $\theta_{1,h}$ to refer to $%
\theta_{1}(s_l) $ and $\theta_{1}(s_h)$ respectively. Where $s_l$ refers to
the low TFP state and $s_h$ refers to the high TFP state.

\begin{lemma}
Suppose that $\theta_1(s_l) < \theta_2(s_h)$ and $\rho$ such that $C(\rho) >
0$ then
\begin{equation*}
l_{1,l} = \frac{g+\sqrt{g^2+4C(\rho)\theta_{1,l}^2}}{2\theta_{1,l}} > \frac{%
g+\sqrt{g^2+4C(\rho)\theta_{1,h}^2}}{2\theta_{1,h}} = l_{1,h}
\end{equation*}
and
\begin{equation*}
\frac1{c_{2,l}} = \frac{1+\rho}{C(\rho)}\frac{g+\sqrt{g^2+4C(\rho)%
\theta_{1,l}^2}}{2\theta_{1,l}^2} > \frac{1+\rho}{C(\rho)}\frac{g+\sqrt{%
g^2+4C(\rho)\theta_{1,h}^2}}{2\theta_{1,h}^2} = \frac1{c_{2.h}}
\end{equation*}%
\label{lem.1}
\end{lemma}

\begin{proof}
The results should follow directly from showing that
the function
\begin{equation*}
l_1(\theta) = \frac{g+\sqrt{g^2+4C(\rho)\theta}}{2\theta}
\end{equation*}%
is decreasing in $\theta$. Taking the derivative with respect to $\theta$
\begin{align*}
\frac{d l_1}{d\theta}(\theta) &=- \frac g{2\theta^2}-\frac{\sqrt{%
g+4C(\rho)\theta^2}}{2\theta^2} +\frac{4C(\rho)\theta}{2\theta\sqrt{%
g^2+4C(\rho)\theta^2}} \\
&=-\frac g{2\theta^2}-\frac{g+4C(\rho)\theta^2-4C(\rho)\theta^2}{2\theta^2%
\sqrt{g^2+4C(\rho)\theta^2}} \\
&=-\frac g{2\theta^2}-\frac{g}{2\theta^2\sqrt{g^2+4C(\rho)\theta^2}}<0
\end{align*}
That $\frac1{c_{2,l}}>\frac1{c_{2,h}}$ follows directly.
\end{proof}

Now we use these lemma to prove the part 1 and part 2 of theorem \ref{thm long run forces}



\begin{proof}

\textbf {[Part 1.]} For a riskfree bond when $P(s)=1$.
In order for there to exist a $\rho$ such that equation (\ref%
{eq.xSS}) is independent of the state (and hence have a steady state) we
need the existence of root for the following function
\begin{equation*}
f (\rho) = \frac{1+\rho[l_1(\rho,s_h)^2-1]}{1+\rho[l_1(\rho,s_l)^2-1]}-\frac{%
\frac{1/c_2(\rho,s_h)}{\mathbb{E}[\frac{P}{c_2}](\rho)}-\beta }{\frac{%
1/c_2(\rho,s_l)}{\mathbb{E}[\frac{P}{c_2}](\rho)}-\beta}
\end{equation*}
From lemma \ref{lem.1} we can conclude that
\begin{equation}
1+\rho[l_1(\rho,s_l)^2-1] > 1+\rho[l_1(\rho,s_h)^2-1]  \label{eq.inc_order}
\end{equation}%
and
\begin{equation}
\frac{1/c_2(\rho,s_l)}{\mathbb{E}[\frac{P}{c_2}](\rho)}-\beta >\frac{%
1/c_2(\rho,s_h)}{\mathbb{E}[\frac{P}{c_2}](\rho)}-\beta  \label{eq.int_order}
\end{equation}%
for all $\rho > 0$ such that $C(\rho)\geq 0$. To begin with we will define $%
\underline \rho$ such that $C(\rho) > 0$ for all $\rho > \underline\rho$.
Note that we will have to deal with two different cases.

\begin{description}
\item[$\protect\alpha_1(1-\protect\rho)+2\protect\rho^2\protect\alpha_2 > 0$
for all $\protect\rho\geq 0$:] In this case we know that $C(\rho)\geq 0$ for
all $\rho$ and is bounded above and thus we will let $\underline \rho =0$.

\item[$\protect\alpha_1(1-\protect\rho)+2\protect\rho^2\protect\alpha_2 = 0$
for some $\protect\rho>0$:] In this case let $\underline \rho$ be the
largest positive root of $\alpha_1(1-\rho)+2\rho^2\alpha_2$. Note that $%
\lim_{\rho\rightarrow \underline \rho^+}C(\rho) = \infty$
\end{description}

With this we note that\footnote{In the first case $\underline {\rho}=0$ and in the second case $l_1(\rho,s_l)=l_1(\rho,s_h)$ as $\rho \to \underline{\rho}^{+}$}
\begin{equation*}
\lim_{\rho\rightarrow \underline \rho^+} \frac{1+\rho[l_1(\rho,s_h)^2-1]}{1+%
\rho[l_1(\rho,s_l)^2-1]} = 1
\end{equation*}%
We can also show that
\begin{equation*}
\lim_{\rho\rightarrow\underline \rho^+} \frac{\frac{1/c_2(\rho,s_h)}{\mathbb{%
E}[\frac{P}{c_2}](\rho)}-\beta }{\frac{1/c_2(\rho,s_l)}{\mathbb{E}%
[\frac{P}{c_2}](\rho)}-\beta} < 1
\end{equation*}
which implies that $\lim_{\rho\rightarrow \underline \rho^+} f(\underline
\rho) > 0$.

Taking the limit as $\rho\rightarrow\infty$ we see that $C(\rho)\rightarrow
0 $, given that $\frac g{\theta(s)} <1$, we can then conclude that
\begin{equation*}
\lim_{\rho\rightarrow\infty } 1+ \rho[l_1(\rho,s)^2-1] = -\infty
\end{equation*}
Thus, there exists $\overline \rho$ such that $1+\overline \rho[%
l_1(\overline \rho,s_l)^2-1] = 0$. \footnote{This can be seen from the fact $\lim_{\rho\rightarrow \underline
\rho^+} 1+\rho[l_1(\rho,s_l)^2 -1] > 0$ and $\lim_{\rho\rightarrow \infty } 1+\rho[l_1(\rho,s_l)^2 -1] > -\infty$, thus $\overline{\rho}$ exists in $(\underline{\rho},\infty)$ } From equation (\ref{eq.inc_order}), we
know that
\begin{equation*}
0 = 1+\overline \rho[l_1(\overline \rho,s_l)^2-1] > 1+\overline \rho[%
l_1(\overline \rho,s_h)^2-1]
\end{equation*}
which implies in the limit
\begin{equation*}
\lim_{\rho\rightarrow \overline \rho^-}\frac{1+\rho[l_1(\rho,s_h)^2-1]}{1+%
\rho[l_1(\rho,s_l)^2-1]} = -\infty
\end{equation*}
which along with
\begin{equation*}
\frac{\frac{1/c_2(\rho,s_h)}{\mathbb{E}[\frac{P}{c_2}]}-\beta }{\frac{%
1/c_2(\rho,s_l)}{\mathbb{E}[\frac{P}{c_2}]}-\beta} \geq -1
\end{equation*}
allows us to conclude that $\lim_{\rho\rightarrow \overline \rho^-} f(\rho)
= -\infty$. The intermediate value theorem then implies that there exists $%
\rho_{SS}$ such that $f(\rho_{SS}) = 0$ and hence that $\rho_{SS}$ is a
steady state.

Finally, as $\rho_{SS} < \overline \rho$ we know that
\begin{equation*}
1+\rho_{SS}[l_1(\rho_{SS},s_l)-1] >0
\end{equation*}%
as $\frac{1/c_2(\rho,s_l)}{\mathbb{E}[\frac{P}{c_2}]} >1$ we can
conclude
\begin{equation*}
x_{SS} = \frac{1+\rho_{SS}[l_1(\rho_{SS},s_l)-1]}{\frac{1/c_2(\rho,s_l)}{%
\mathbb{E}[\frac{P}{c_2}](\rho)}-\beta} >0
\end{equation*}%
implying that the government will hold assets in the steady state (under the
normalization that agent 2 holds no assets).

\textbf {[Part 2]} As noted before, since $g/\theta (s)<1$ for all $s$ we have
\begin{equation*}
\lim_{\rho \rightarrow \infty }1+\rho \lbrack l_{1}(\rho ,s)^{2}-1]=-\infty
\end{equation*}%
Thus, there exists $\rho_{SS}$ such that
\begin{equation*}
0>1+\rho_{SS}[l_{1}(\rho_{SS},s_{l})^{2}-1]>1\rho_{SS}[l_{1}(\rho_{SS},s_{h})^{2}-1]
\end{equation*}%
It is then possible to choose $P (s)$ such that $\beta <\frac{P(s)/c_{2}(\rho_{SS},s)}{%
\mathbb{E}[\frac{P}{c_{2}}]}$ such that
\begin{equation}
1>\frac{1+\rho_{SS}[l_{1}(\rho_{SS},s_{l})^{2}-1]}{1+\rho_{SS}[l_{1}(\rho_{SS},s_{h})^{2}-1]}=\frac{\frac{P(s_l)/c_{2}(\rho_{SS},s_{l})%
}{\mathbb{E}[\frac{P}{c_{2}}]}-\beta}{\frac{%
P(s_h)/c_{2}(\rho_{SS},s_{h})}{\mathbb{E}[\frac{P}{c_{2}}]}%
-\beta }  \label{eq.beta_cond}
\end{equation}%
Implying that for Payoff shocks $P(s)$, $\rho_{SS}$ is a
steady state level for the ratio of marginal utilities, with steady state
marginal utility weighted government debt
\begin{equation*}
x_{SS}=\frac{1+\rho_{SS}[l_{1}(\rho_{SS},s_{l})^{2}-1]}{\frac{%
P(s_l)/c_{2}(\rho_{SS},s_{l})}{\mathbb{E}[\frac{P}{c_{2}}]}%
-\beta}<0
\end{equation*}%
Thus, in the steady state, the government is holding debt, under the
normalization that the unproductive worker holds no assets. Note this imposes a restriction of $\frac{P(s_l)}{P(s_h)}$.

\[\frac{P(s_l)c_2^{-1}(\rho_{SS},s_l)-\beta \mathbb{E}Pc_2^{-1}}{P(s_h)c_2^{-1}(\rho_{SS},s_h)-\beta \mathbb{E}Pc_2^{-1}}<1\]
or

\[\frac{P(s_l)}{P(s_h)}<\frac{c_2^{-1}(\rho_{SS},s_h)}{c_2^{-1}(\rho_{SS},s_l)}<1\]

or 

Thus $P(s_l)<P(s_h)$ i.e payoffs have to be sufficiently procyclical.

\end{proof}

    


\end{proof}
 






\end{document}
